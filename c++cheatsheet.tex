%
%    Another C++ cheat sheet but written in LaTeX to be printed out on A3 paper
%    Copyright (C) 2021 Michele Iarossi - michele@mathsophy.com
%
%    This program is free software: you can redistribute it and/or modify
%    it under the terms of the GNU General Public License as published by
%    the Free Software Foundation version 3 of the License.
%
%    This program is distributed in the hope that it will be useful,
%    but WITHOUT ANY WARRANTY; without even the implied warranty of
%    MERCHANTABILITY or FITNESS FOR A PARTICULAR PURPOSE.  See the
%    GNU General Public License for more details.
%
%    You should have received a copy of the GNU General Public License
%    along with this program.  If not, see <https://www.gnu.org/licenses/>.
%
%
% Example of document composed in LaTeX
%
%
% Source file ending with .tex
%

% Preamble starts here: includes further global commands and specifications
% In the following, the option 10pt is a global option passed to all
% the packages
\documentclass[10pt]{article}

% A3 paper size
\usepackage[a3paper]{geometry}

\usepackage{multicol}
\usepackage{listings}
\lstset{
    escapeinside={(*<}{>*)},          % if you want to add LaTeX within your code
}

% This encoding corresponds to the encoding used by TexShop for saving this file
% The option utf8 is a local option, valid only for the package inputenc
\usepackage[utf8]{inputenc}

% Layout package shows the layout of the page
% You can issue the command \layout for printing the applied layout of the pages
\usepackage{layout}

% Some length parameters are rubber lengths: they stretch or reduce as needed
% 1ex means one time the height of the letter x
% Customize page layout for more space
\setlength{\parskip}{1ex plus0.5ex minus0.2ex}
\setlength{\oddsidemargin}{0pt}
\setlength{\topmargin}{0pt}
\setlength{\headsep}{0pt}
\setlength{\headheight}{0pt}
\setlength{\marginparwidth}{0pt}
\setlength{\marginparsep}{0pt}
\setlength{\textwidth}{700pt}
\setlength{\textheight}{1020pt}
\setlength{\footskip}{50pt}

% 1cm separation between columns
\setlength{\columnsep}{1cm}

\usepackage{fancyhdr}
\pagestyle{fancy}
\renewcommand{\headrulewidth}{0pt}
\renewcommand{\footrulewidth}{0pt}
\fancyhead{}

\fancyfoot[L]{\small www.mathsophy.com}
\fancyfoot[C]{\small \thepage}
\fancyfoot[R]{\small GNU GPL v3.0}


% This is the body: text mixed with commands

% Commands have the following general syntax:
% \name[optional]{mandatory}
\begin{document}

% Set title and author
\title{\emph{C++ basics cheat sheet}}
\author{Michele Iarossi\thanks{\texttt{michele@mathsophy.com}}}
\date{\small \today~-~Version 1.01~-~GNU GPL v3.0}

%\layout

\maketitle

% Do not balance columns on the last page
% 3 columns
\begin{multicols*}{3}
%
% Type casting
%
\section*{Type casting}
\small
\lstset {language=C++}
\begin{lstlisting}
// (*<int 15>*) to (*<double 15.0>*)
double num;
num = static_cast<double>(15);
\end{lstlisting}
%
% Functions
%
\section*{Functions}
\small
Function declaration with
default trailing arguments
\lstset {language=C++}
\begin{lstlisting}
#include <iostream>
using namespace std;

// if (*<year>*) is omitted, 
// then (*<year = 2000>*)
void setBirthday(int day,
int month, int year=2000);
\end{lstlisting}
%
% Random numbers
%
\section*{Random numbers}
\small
\lstset {language=C++}
\begin{lstlisting}
#include <cstdlib>
#include <ctime>
using namespace std;

// seed the generator
srand( time(0) );
// integer random number between
// 0 and RAND_MAX
int n = rand();
\end{lstlisting}
%
% Pointers
%
\section*{Pointers}
\small
\lstset {language=C++}
\begin{lstlisting}
#include <iostream>
using namespace std;

// simple pointer to (*<double>*)
double *d = new double(5.123);

// dynamic array of (*<10 double>*)s
double *dd = new double[10];

// delete the storage
// on the freestore
delete d;
delete [] dd;
\end{lstlisting}
%
% Arrays
%
\section*{Arrays}
\small
\lstset {language=C++}
\begin{lstlisting}
#include <iostream>
using namespace std;

// range based for statement
int arr[] = {2, 4, 6, 8};

for (int& x : arr)
  x++;
// outputs 3579
for (auto x : arr)
  cout << x;
cout << endl;
\end{lstlisting}
%
% C-Strings
%
\section*{C-Strings}
\small
\lstset {language=C++}
\begin{lstlisting}
#include <cstring>
#include <cstdlib>

// (*<C-string>*) for max 10 characters
// long string + null char '\0'
const int SIZE = 10 + 1;
char msg[SIZE] = "Hello!";

// correct looping over (*<C-string>*)s
int i = 0;
while ( msg[i] != '\0' && i < SIZE)
{
   // process msg[i]
}

// safe string copy,
// at most 10 characters are copied
strncpy(msg, srcStr, 10);

// safe string compare,
// at most 10 characters
// are compared
strncmp(msg, srcStr, 10);

// safe string concatenation,
// at most 10 characters 
// are concatenated
strncat(msg, srcStr, 10);

// from (*<C-string>*) to (*<int>*),
// (*<long>*), (*<float>*)
int n = atoi("567");
long n = atol("1234567");
double n = atof("12.345");
\end{lstlisting}
%
% Standard I/O
%
\section*{Standard I/O}
\small
\lstset {language=C++}
\begin{lstlisting}
#include <iostream>
#include <iomanip>
using namespace std;

// set flag
cout.setf(ios::fixed);
// unset flag
cout.unsetf(ios::fixed);

// set (*<ios::fixed>*) or 
// (*<ios::scientific>*) notation
cout.setf(ios::fixed);
cout << fixed;

// set precision
cout.precision(4);
cout << setprecision(4);

// set character text width
cout.width(10);
cout << setw(10);

// set (*<ios::left>*) or
// (*<ios::right>*) alignment
cout.setf(ios::left);
cout << left;

// always show decimal
// point and zeros
cout.setf(ios::showpoint);
cout << showpoint;

// always show plus sign
cout.setf(ios::showpos);
cout << showpos;
\end{lstlisting}
%
% Character I/O
%
\section*{Character I/O}
\small
\lstset {language=C++}
\begin{lstlisting}
#include <iostream>
using namespace std;

// read any character from (*<cin>*)
// (doesn't skip spaces, 
// newlines, etc.)
char nextChar;
cin.get(nextChar);

// write a character to (*<cout>*)
cout.put(nextChar)

// read a whole line of 80 chars
char line[80+1];
cin.getline(line,81);

// put back a char to (*<cin>*)
// nextChar will be the next
// char read by cin.get()
cin.putback(nextChar);
\end{lstlisting}
%
% Files
%
\section*{Files}
\small
\lstset {language=C++}
\begin{lstlisting}
#include <fstream>
using namespace std;

// input file
ifstream inStream;
// output file
ofstream outStream;

// open
inStream.open("infile.dat");
outStream.open("outfile.dat");

// check for failure
if ( inStream.fail() ||
     outStream.fail() )
{
  // file opening failed
}

// read/write data
inStream >> data1 >> data2;
outStram << data1 << data2;

// checking for end of file
while ( ! inStream.eof() )
{
  inStream >> next;
}

// close file
inStream.close();
outStream.close()
\end{lstlisting}
%
% Strings
%
\section*{Strings}
\small
\lstset {language=C++}
\begin{lstlisting}
#include <string>
using namespace std;

// initialization
string s1 = "Hello";
string s2("World");

// concatenation
string s3 = s1 + ", " + s2;

// read a line
string line;
getline(cin,line);

// access to the ith character
// (no illegal index checking)
s1[i];

// access to the ith character
// (with illegal index checking)
s1.at(i);

// append
s1.append(s2);

// size and length
s1.size();
s1.length();

// substring from position 5
// and length 4 characters
s4.substr(5,4);

// find (returns string::npos
// if not found)
s3.find("World");

// find starting from position 5
s3.find("l",5);

// C-string
s3.c_str();

// from (*<string>*) to (*<int>*),
// (*<long>*), (*<float>*)
int n = stoi("456");
long n = stol("1234567");
double n = stod("12.345");

// from numeric type to (*<string>*)
string s = to_string(123.456);
\end{lstlisting}
%
% Vectors
%
\section*{Vectors}
\small
\lstset {language=C++}
\begin{lstlisting}
#include <iostream>
#include <vector>
using namespace std;

// vector with base type (*<int>*)
vector<int> v = {2, 4, 6, 8};

// vector with 10 elements
// all initialised to 0
vector<int> v(10);

for (auto x : v)
  cout << v << endl;

// access to ith element
cout << v[i];

// add an element
v.push_back(10);

// size
cout << v.size();

// capacity: number of 
// elements currently allocated
cout << v.capacity();

// reserve more capacity
// e.g. at least 64 (*<int>*)s
v.reserve(64);
\end{lstlisting}
%
% Classes
%
\section*{Classes}
\small
Note: If you give no constructor, the compiler will generate a default constructor that does nothing.
If you give at least one constructor, then the C++ compiler will generate no other constructors.
\lstset {language=C++}
\begin{lstlisting}
#include <iostream>
using namespace std;

class Car
{
public:
    // constructor
    Car(double);
    // mutators
    void setEngineSize(const
         double&);
    // accessors
    double getEngineSize() const;
    // friend function
    friend bool equal(const Car&,
         	const Car&);
private:	
    double engineLiter;
};

// constructor
// with initialization list
Car::Car(double engineSize) :
engineLiter(engineSize)
{
}

// parameter passed by
// reference for efficiency
void Car::setEngineSize(const 
		double &size)
{
    engineLiter = size;
}

// constant member function
// doesn't change the object
double Car::getEngineSize() const
{
    return engineLiter;
}

// friend function with
// direct access to
// private members
bool equal(const Car &car1,
	const Car &car2)
{
    return car1.engineLiter ==
    	 car2.engineLiter;
}
\end{lstlisting}
%
% Operator overloading
%
\section*{Operator overloading}
\small
Note: the behaviour is different if overloaded as class members or 
friend functions.
\\
\\
As class members:
\lstset {language=C++}
\begin{lstlisting}
#include <iostream>
using namespace std;

class Euro
{
    // constructor for  (*<euro>*)
    Euro(int);
    // constructor for  (*<euro>*) and
    // (*<cents>*)
    Euro(int,int);
    // works for (*<Euro(5) + 2>*),
    // equivalent to
    // (*<Euro(5).operator+( Euro(2) )>*)
    // doesn't work for (*<2 + Euro(5)>*)
    // (*<2>*) is not a calling object 
    // of type (*<Euro>*) !
    Euro operator+(const Euro&); 
    friend Euro 
        operator+(const Euro&,
    		const Euro&);
private:
    int euro;
    int cents;
};
\end{lstlisting}
As friend members:
\lstset {language=C++}
\begin{lstlisting}
class Euro
{
    // constructor for  (*<euro>*)
    Euro(int);
    // constructor for (*<euro>*) and
    // (*<cents>*)
    Euro(int,int);
    // works for every combination
    // int arguments are converted
    // by the constructor to (*<Euro>*)
    // objects
    friend Euro
        operator+(const Euro&,
             const Euro&);
    // insertion and extraction
    // operators
    friend ostream& 
        operator<<(ostream&,
            const Euro&);
    friend istream&
        operator>>(istream&, Euro&);
private:
    int euro;
    int cents;
};
\end{lstlisting}
%
% Copy constructor / Assignment operator
%
\section*{Copy constructor / Assignment operator}
\small
Note: If not defined, C++ automatically adds the default copy
constructor and the default assignment operator.
They might not be correct if dynamic variables are used,
because class members are simply copied.
\lstset {language=C++}
\begin{lstlisting}
#include <iostream>
using namespace std;

class IntList
{
    // constructor with 
    // size of the list
    IntList(int);
    // copy constructor
    IntList(IntList&);
    // assignment operator
    IntList& operator=(const IntList&);
private:
    int *p;
    int size;
}

// call the copy constructor
// (*<secondList>*) is initialised
// from (*<firstList>*)
IntList secondList(firstList); 

// call the assignment operator
thirdList = firstList;
\end{lstlisting}
%
% Inheritance
%
\section*{Inheritance}
\small
Note: Constructors, desctructor, private member functions, copy constructor
and assignment operator are not inherited! Derived classes get the default ones
if they are not explicitely provided but are present in the base class.
\lstset {language=C++}
\begin{lstlisting}
#include <iostream>
using namespace std;

// a simple book class
class Book
{
public:
    Book(string t,int p);
    void print(ostream& os);
protected:
    int pages;
    string title;
};
\end{lstlisting}
Redefinition of function members:
\lstset {language=C++}
\begin{lstlisting}
// a simple textbook class
class Textbook : public Book
{
public:
    Textbook(string t,int p,
          string s);
    // redefinition of (*<print()>*) 
    // from the base class
    void print(ostream& os);
protected:
    string subject;
};
\end{lstlisting}
\textbf{protecetd} members can be accessed
by derived function members:
\lstset {language=C++}
\begin{lstlisting}
// has access to (*<protected>*) 
// members of he base class
void Textbook::print(ostream& os)
{
    os << "The title of this "
         << "textbook is '" <<
       title << "' and the"
       << " textbook is " <<
       pages << " pages long." 
       << endl;
    os << "The subject is '" 
         << subject
         << "'" << endl;
}
\end{lstlisting}
Note: With redefinition, no polymorphism!
\lstset {language=C++}
\begin{lstlisting}
Book *abook = &aMathTextbook;
// call (*<Book::print()>*)
// not (*<Textbook::print()>*)!
abook->print(cout);
\end{lstlisting}
%
% Polymorphism
%
\section*{Polymorphism}
\small
\textbf{virtual} allows for late binding, i.e.
polymorphism. Function members are
overridden in the derived class.\\
Note: Destructors should also be
declared \textbf{virtual}. When derived
objects are referenced by base class
pointers, the destructor of the derived class
is called if it is declared virtual.
\lstset {language=C++}
\begin{lstlisting}
#include <iostream>
using namespace std;

// a simple book class
class Book
{
public:
    Book(string t,int p);
    virtual ~Book();
    void print(ostream& os);
protected:
    int *pages;
    string *title;
};

Book::Book(string t, int p)
{
    pages = new int(p);
    title = new string(t);
}

Book::~Book()
{
    delete pages;
    delete title;
}

// a simple textbook class
class Textbook : public Book
{
public:
    Textbook(string t,int p,
          string s);
    virtual ~Textbook();
    // overriding of (*<print()>*) 
    // from the base class
    virtual void print(ostream& os);
protected:
    string *subject;
};

Textbook::Textbook(string t,
     int p, string s) :
Book(t,p)
{
    subject = new string(s);
}

Textbook::~Textbook()
{
    delete subject;
}

Book *abook = &aMathTextbook;
// call (*<Textbook::print()>*)!
abook->print(cout);
\end{lstlisting}
%
% Exceptions
%
\section*{Exceptions}
\small
Note: The value thrown by
\textbf{throw} can be of any type.
\lstset {language=C++}
\begin{lstlisting}
#include <iostream>
using namespace std;

// exception class
class MyException
{
public:
    MyException(string s);
    virtual ~MyException();
    friend ostream& 
        operator<<(ostream&,
    	    const MyException& e);
protected:
    string msg;
};

try
{
    throw MyException("error");
}
catch (MyException e)
{
    cout << e;
}
// everything else
catch (...)
{
    exit(1);
}
\end{lstlisting}
Functions throwing exceptions should list the exceptions thrown in
the exception specification list. These exceptions are not 
caught by the function itself!

\lstset {language=C++}
\begin{lstlisting}
#include <iostream>
using namespace std;

// exceptions of type (*<DivideByZero>*) or 
// (*<OtherException>*) are
// to be caught outside the function.
// All other exceptions end the program
// if not caught inside the function.
void myFunction( ) throw (DivideByZero,
            OtherException);

// empty exception list; 
// all exceptions end the
// program if thrown but
// not caught inside the function.
void myFunction( ) throw ( );

// all exceptions of all
// types treated normally.
void myFunction( );
\end{lstlisting}
%
% Templates
%
\section*{Templates}
\small
Function templates:\\ \\
Note: C++ does not need 
the template declaration.
The template function
definition is included directly.
\lstset {language=C++}
\begin{lstlisting}
#include <iostream>
using namespace std;

// generic swap function
template<class T>
void swap(T& a, T& b)
{
    T temp = a;
    
    a = b;
    b = temp;
}

int a, b;
char c,d;

// swaps two (*<int>*)s
swap(a,b);

// swaps two (*<char>*)s
swap(c,d);
\end{lstlisting}
Class templates:\\ \\
Note: Methods are defined as 
template functions

\lstset {language=C++}
\begin{lstlisting}
#include <iostream>
using namespace std;

template<class T>
class AList
{
    // constructor with 
    // size of the list
    AList(int size);
    // destructor
    ~AList();
    // copy constructor
    AList(AList<T>& b);
    // assignment operator
    AList<T>& operator=(const
        AList<T>& b);
private:
    T *p;
    int size;
}

// constructor definition
template<class T>
AList<T>::AList(int size)
{
    p = new T[size];
}

// variable declaration
AList<double> list;
\end{lstlisting}
%
% Iterators
%
\section*{Iterators}
\small
An iterator is a generalization of a pointer. Different containers have
different iterators.
\lstset {language=C++}
\begin{lstlisting}
#include <iostream>
#include <vector>
using namespace std;

vector<int> v = {1,2,3,4,5};
// mutable iterator
vector<int>::iterator e;

// bidirectional access
e = v.begin();
++e;
// print (*<v[1]>*)
cout << *e << endl;
--e;
// print (*<v[0]>*)
cout << *e << endl;

// random access
e = v.begin();
// print (*<v[3]>*)
cout << e[3] << endl;

// change an element
e[3] = 9;

// constant iterator (only read)
vector<int>::constant_iterator c;

// print out the vector content
// (read only)   
for (c = v.begin(); c != v.end(); c++)
    cout << *c << endl;

// not allowed
// (*<c[2] = 2;>*)

// reverse iterator
vector<int>::reverse_iterator r;

// print out the vector content
// in reverse order
for (r = v.rbegin(); r != v.rend(); r++)
    cout << *r << endl;
\end{lstlisting}
%
% Containers
%
\section*{Containers}
\small
Sequential containers: \textbf{list}
\lstset {language=C++}
\begin{lstlisting}
#include <iostream>
#include <list>
using namespace std;

list<double> data = {1.32,-2.45,5.65};

// adds elements
data.push_back(9.23);
data.push_front(-3.94);

// bidirectional iterator
// no random access    
list<double>::iterator e;

// erase    
e = data.begin();
++e;
data.erase(e);

// print out the content    
for (e = data.begin();
    e != data.end(); e++)
    cout << *e << endl;
\end{lstlisting}
Adapter containers: \textbf{stack}
\lstset {language=C++}
\begin{lstlisting}
#include <iostream>
#include <stack>
using namespace std;

stack<double> numbers;

// push on the stack
numbers.push(5.65);
numbers.push(-3.95);
numbers.push(6.95);

// size
cout << numbers.size()

// read top data element
double d = numbers.top();

// pop top element
numbers.pop();
\end{lstlisting}
Associative containers: \textbf{set}
\lstset {language=C++}
\begin{lstlisting}
#include <iostream>
#include <set>
using namespace std;

set<char> letters;

// inserting elements    
letters.insert('a');
letters.insert('d');
// no duplicates!
letters.insert('d');
letters.insert('g');

// erase    
letters.erase('a');

// const iterator 
set<char>::const_iterator c;   
for (c = letters.begin();
    c != letters.end(); c++)
    cout << *c << endl;
\end{lstlisting}
Associative containers: \textbf{map}
\lstset {language=C++}
\begin{lstlisting}
#include <iostream>
#include <string>
#include <map>
#include <utility>
using namespace std;

// initialization
map<string,int> dict = 
    { {"one",1}, {"two",2} };
pair<string,int> three("three",3);

// insertion    
dict.insert(three);
dict["four"] = 4;
dict["five"] = 5;

// iterator    
map<string,int>::iterator two;

// find    
two = dict.find("two");

// erase    
dict.erase(two);

// ranged loop
for (auto n : dict)
    cout << "(" <<  n.first
            << "," <<  n.second 
            << ")" << endl;
\end{lstlisting}
%
% Algorithms
%
\section*{Algorithms}
\small
\lstset {language=C++}
\begin{lstlisting}
#include <iostream>
#include <vector>
#include <algorithm>
using namespace std;

vector<int> v = {6,2,7,13,4,3,1};
vector<int>::iterator p;
bool found;

// find
p = find(v.begin(),v.end(),13);

// merge sort
sort(v.begin(),v.end());

// binary search
found = binary_search(v.begin(),
    v.end(),3);

// reverse
reverse(v.begin(),v.end());
\end{lstlisting}
\end{multicols*}
\end{document}
