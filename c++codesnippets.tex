%
%    C++ code snippets reference written in LaTeX to be printed out on A4 paper
%    Copyright (C) 2021 Michele Iarossi - michele@mathsophy.com
%
%    This program is free software: you can redistribute it and/or modify
%    it under the terms of the GNU General Public License as published by
%    the Free Software Foundation version 3 of the License.
%
%    This program is distributed in the hope that it will be useful,
%    but WITHOUT ANY WARRANTY; without even the implied warranty of
%    MERCHANTABILITY or FITNESS FOR A PARTICULAR PURPOSE.  See the
%    GNU General Public License for more details.
%
%    You should have received a copy of the GNU General Public License
%    along with this program.  If not, see <https://www.gnu.org/licenses/>.
%
%
% Example of document composed in LaTeX
%
%
% Source file ending with .tex
%

% Preamble starts here: includes further global commands and specifications
% In the following, the option 10pt is a global option passed to all
% the packages
\documentclass[10pt]{article}

% Using Courier font
\renewcommand{\ttdefault}{pcr}

% A4 paper size
\usepackage[a4paper]{geometry}

% Change the color of the text
\usepackage{xcolor}
\definecolor{codegray}{gray}{0.80}

% multiple columns in the toc
\usepackage{multicol}

% Add filling dots for sections in the toc
\usepackage{tocloft}
\renewcommand{\cftsecleader}{\cftdotfill{\cftdotsep}}

% C++ code listings embedded in the document
\usepackage{listings}
\lstset{
    language = [11]C++,
    basicstyle = \ttfamily,
    keywordstyle=\bfseries,
    escapeinside={(*}{*)},          % if you want to add LaTeX within your code
    backgroundcolor = \color{codegray},
    xleftmargin = 1.2cm,
    framexleftmargin = 1em
}

% This encoding corresponds to the encoding used by TexShop for saving this file
% The option utf8 is a local option, valid only for the package inputenc
\usepackage[utf8]{inputenc}

\usepackage[T1]{fontenc}

% Layout package shows the layout of the page
% You can issue the command \layout for printing the applied layout of the pages
\usepackage{layout}

% Some length parameters are rubber lengths: they stretch or reduce as needed
% 1ex means one time the height of the letter x
% Customize page layout for more space
\setlength{\parskip}{1ex plus0.5ex minus0.2ex}

% No header, only footer
\usepackage{fancyhdr}
\fancypagestyle{plain}{
    \fancyfoot[L]{\small www.mathsophy.com}
    \fancyfoot[C]{\small \thepage}
    \fancyfoot[R]{\small GNU GPL v3.0}}
\pagestyle{fancy}
\renewcommand{\headrulewidth}{0pt}
\renewcommand{\footrulewidth}{0pt}
\fancyhead{}
\fancyfoot[L]{\small www.mathsophy.com}
\fancyfoot[C]{\small \thepage}
\fancyfoot[R]{\small GNU GPL v3.0}

\setcounter{secnumdepth}{0} % sections are level 1

% for british hyphenation
\usepackage[british]{babel}

% for making the index
\usepackage{imakeidx}

\makeindex
\indexsetup{othercode=\fontsize{9}{10}\selectfont}

% This is the body: text mixed with commands

% Commands have the following general syntax:
% \name[optional]{mandatory}
\begin{document}

% Set title and author
\title{\emph{C++ code snippets}}
\author{Michele Iarossi\thanks{\texttt{michele@mathsophy.com}}}
\date{\small \today~-~Version 1.75~-~GNU GPL v3.0}

%\layout

\maketitle

\small

\begin{frame}{}
\setlength\columnsep{1cm}
\begin{multicols}{2}
\tableofcontents
\end{multicols}
\end{frame}

\noindent
In the following code snippets, the standard I/O library and namespace are always used:
\begin{lstlisting}
#include <iostream>(*\index{\texttt{<iostream>}}*)
using namespace std;(*\index{\texttt{std}}*)(*\index{Namespaces!\textbf{using namespace} directives}*)(*\index{\textbf{namespace}|see{Namespaces}}*)
\end{lstlisting}

\newpage
%
% assert
%
\section{Assertions}
\small
\begin{enumerate}
\item[$\Rightarrow$] The first argument of a \textbf{static\_assert} is a constant expression that must be true:
\end{enumerate}
\begin{lstlisting}
static_assert(8<=sizeof(long),"longs are too small");(*\index{Assertions!\textbf{static\_assert}}*)(*\index{\textbf{static\_assert}}*)
\end{lstlisting}
%
% constants
%
\section{Constants}
\small
There are two options:
\begin{enumerate}
\item[$\Rightarrow$] \textbf{constexpr} must be known at compile time:
\end{enumerate}
\begin{lstlisting}
constexpr int max = 200;(*\index{\textbf{constexpr}|see{Constants}} *)(*\index{Constants!\textbf{constexpr}}*)
constexpr int c = max + 2;(*\index{Constants!\textbf{constexpr}}*)
\end{lstlisting}
\begin{enumerate}
\item[$\Rightarrow$] \textbf{constexpr} applied to functions instructs the compiler to try to evaluate the function at compile time:\index{Constants!constant functions}
\end{enumerate}
\begin{lstlisting}
constexpr int func(int n) { return n*2+5; }
constexpr int c = func(122);  // (*149*)
\end{lstlisting}
\begin{enumerate}
\item[$\Rightarrow$] Integer literal with single quotes for readability:\index{Integer literal!single quotes}
\end{enumerate}
\begin{lstlisting}
// (*1000000*)
constexpr int k = 1'000'000;
\end{lstlisting}
\begin{enumerate}
\item[$\Rightarrow$] \textbf{const} variables don't change at runtime. They cannot be declared as
\textbf{constexpr} because their value is not known at compile time:
\end{enumerate}
\begin{lstlisting}
// the value of (*n*) is not known at compile time
const int m = n + 1;(*\index{\textbf{const}|see{Constants}}*)(*\index{Constants!\textbf{const}}*)
\end{lstlisting}
%
% Type safety
%
\section{Type safety}
\small
\begin{enumerate}
\item[$\Rightarrow$] Universal and uniform initialization\index{Uniform initialization}\index{Braced initialization|see{Uniform initialization}} prevents narrowing conversions from happening:
\end{enumerate}
\begin{lstlisting}
// safe conversions (*\index{Conversions!safe}*)
double x {54.21};
int a {2342};

// unsafe conversions (compile error!) (*\index{Conversions!unsafe}*)
int y {x};
char b {a};
\end{lstlisting}
%
% Type casting
%
\section{Type casting}
\small
\begin{enumerate}
\item[$\Rightarrow$] Use \textbf{static\_cast} for normal casting\index{Type casting!\textbf{static\_cast}}, i.e. types that can be converted into each other:
\end{enumerate}
\begin{lstlisting}
// (*int 15*) to (*double 15.0*)
double num;
num = static_cast<double>(15);(*\index{\textbf{static\_cast}|see{Type casting}}*)
\end{lstlisting}
\begin{enumerate}
\item[$\Rightarrow$] Use \textbf{static\_cast} for casting a void pointer to the desired pointer type:
\end{enumerate}
\begin{lstlisting}
// (*void **) pointer can point to anything
double num;
void *p = &num;(*\index{\texttt{void $\ast$}|see{Pointers}}*)(*\index{Pointers!\texttt{void $\ast$}}*) 

// back to (*double*) type
double *pd = static_cast<double*>(p);(*\index{\textbf{static\_cast}|see{Type casting}}*)
\end{lstlisting}
\begin{enumerate}
\item[$\Rightarrow$] Use \textbf{reinterpret\_cast} for casting between unrelated pointer types\index{Type casting!\textbf{reinterpret\_cast}}:
\end{enumerate}
\begin{lstlisting}
// reinterprets a (*long*) value as a (*double*) one
long n = 53;
double *pd = reinterpret_cast<double *>(&n);(*\index{\textbf{reinterpret\_cast}|see{Casts}}*)

// prints out (*2.61855e-322*)
cout << *pd << endl;(*\index{Input-output streams!\texttt{cout}}*)
\end{lstlisting}
\begin{enumerate}
\item[$\Rightarrow$] Use user-defined type conversions\index{Type casting!user defined type conversions}:
\end{enumerate}
\begin{lstlisting}
// User defined type
class MyType
{
public:
    MyType(int y=1) : x{y} {}
    operator int() const { return x; }(*\index{Type casting!user defined type conversions!\texttt{operator int()}}*)(*\index{\texttt{operator int()}}*)
    explicit operator double() const { return double(x); }(*\index{Type casting!user defined type conversions!\texttt{explicit operator double()}}*)(*\index{\texttt{operator double()}}*)
private:
    int x{0};
};

MyType a{5};
MyType b{7};

// (*a*) and (*b*) are converted to (*int*) by (*operator int()*)
// (*c = 12*)
int c = a + b;

// (*b*) is converted to (*double*) by (*operator double()*)
// but requires explicit static cast
double =   static_cast<double>(b);(*\index{\textbf{static\_cast}|see{Type casting}}*)
\end{lstlisting}

%
% Limits
%
\section{Limits}
\small
\begin{enumerate}
\item[$\Rightarrow$] Use \texttt{numeric\_limits<T>}\index{\texttt{numeric\_limits<T>}} for checking against built-in type limits:
\end{enumerate}
\begin{lstlisting}
#include <limits>(*\index{\texttt{<limits>}}*)

// (*int*) type
cout << numeric_limits<int>::min() << endl; // (*-2147483648*)
cout << numeric_limits<int>::max() << endl; // (*2147483647*)
    
// (*double*) type
cout << numeric_limits<double>::min() << endl; // (*2.22507e-308*)
cout << numeric_limits<double>::max() << endl; // (*1.79769e+308*)
cout << numeric_limits<double>::lowest() << endl; // (*-1.79769e+308*)
cout << numeric_limits<double>::epsilon() << endl; // (*2.22045e-16*)
cout << numeric_limits<double>::round_error() << endl; // (*0.5*)
\end{lstlisting}
%
% Functions
%
\section{Functions}
\small
\begin{enumerate}
\item[$\Rightarrow$] With default trailing arguments only in the function declaration:
\index{Functions!arguments!default}
\end{enumerate}
\begin{lstlisting}
// if (*year*) is omitted, then (*year = 2000*)
void set_birthday(int day, int month, int year=2000);
\end{lstlisting}
\begin{enumerate}
\item[$\Rightarrow$] Omitting the name of an argument if not used anymore in the function definition:
\index{Functions!arguments!omitted}
\end{enumerate}
\begin{lstlisting}
// argument (*year*) is not used anymore in the function definition
// (doesn't break legacy code!)
void set_birthday(int day, int month, int) { ...}
\end{lstlisting}
\begin{enumerate}
\item[$\Rightarrow$] With read-only, read-write and copy-by-value parameters:
\index{Functions!arguments!read-only}
\index{Functions!arguments!read-write}
\index{Functions!arguments!copy-by-value}
\index{Functions!arguments!copy-by-reference}
\end{enumerate}
\begin{lstlisting}
// (*day*) input parameter passed by const reference (read-only)
// (*month*) output parameter to be changed by the function (read-write)
// (*year*) input parameter copied-by-value
void set_birthday(const int& day, int& month, int year);(*\index{Constants!\textbf{const}}*)
\end{lstlisting}
\begin{enumerate}
\item[$\Rightarrow$] Use a function for initializing an object with a complicated initializer (we might not
know exactly when the object gets initialized):
\index{Functions!object initialization}
\end{enumerate}
\begin{lstlisting}
const Object& default_value()(*\index{Constants!\textbf{const}}*)
{
  static const Object default{1,2,3};(*\index{Constants!\textbf{const}}*)
  return default;
}
\end{lstlisting}
\begin{enumerate}
\item[$\Rightarrow$] Rule of thumb for passing arguments to functions:
\index{Functions!arguments!rule of thumb}
\begin{itemize}
\item Pass-by-value for small objects
\item Pointer parameter type if \textbf{nullptr} means no object given\index{\textbf{nullptr}}
\item Pass-by-const-reference for large objects that are not changed
\item Pass-by-reference for large objects that are changed (output parameters)
\item Return error conditions of the function as return values
\end{itemize}
\end{enumerate}
\begin{enumerate}
\item[$\Rightarrow$] Function pointer type definition:
\index{Functions!pointer to function}
\end{enumerate}
\begin{lstlisting}
// pointer to a function returning a (*void*) and
// having parameters a pointer to a (*Fl\_Widget*) and a pointer to a (*void*)
typedef void ( *Callback_type )( Fl_Widget*, void* ); (*\index{\textbf{typedef}}*)

// (*cb*) is a callback defined as above
Callback_type cb;
\end{lstlisting}
%
% Lambda expressions
%
\section{Lambda expressions}
\small
An unnamed function that can be used where a function is needed as an argument or object.
It is introduced by \texttt{[ ]} which are called \emph{lambda introducers}.
\index{Lambda expressions!lambda introducers} 
\begin{enumerate}
\item[$\Rightarrow$] Without access to local variables:\index{Lambda expressions!without access to local variables}
\end{enumerate}
\begin{lstlisting}
// Instantiates a (*Function*) object where the first argument is
// an unnamed function having one (*double*) parameter (*x*)
// and returning a (*double*). The return type is inferred
Function e_gr{[](double x){return exp(x);},{-8.0,8.0},0.001,(*\index{\texttt{[ ]}}*)
        {-8.0,8.0},{320,240},400};
\end{lstlisting}
\begin{enumerate}
\item[$\Rightarrow$] With access to local variables:\index{Lambda expressions!with access to local variables}
\end{enumerate}
\begin{lstlisting}
// Same as above, but the variable (*n*) inside the lambda introducer
// is available for the function to be used
int n = 5;
Function ee_gr{[n](double x){return expe(x,n);},{-8.0,8.0},0.001,
        {-8.0,8.0},{320,240},400};
 \end{lstlisting}
%
% Namespaces
%
\section{Namespaces and aliases}
\small
\begin{enumerate}
\item[$\Rightarrow$] \textbf{using} declarations for avoiding fully qualified names:
\index{Namespaces!\textbf{using} declarations}
\end{enumerate}
\begin{lstlisting}
// use (*string*) instead of (*std::string*)
using std::string;(*\index{\texttt{std}}*)(*\index{Strings!\texttt{string}}*)(*\index{\texttt{string}|see{Strings}}*)(*\index{\textbf{using}|see{Namespaces}}*)

// use (*cin*), (*cout*) instead of (*std::cin*), (*std::cout*)
using std::cin;(*\index{Input-output streams!\texttt{cin}}*)(*\index{\texttt{std}}*)(*\index{\texttt{cin}|see{Input-output streams}}*)
using std::cout;(*\index{Input-output streams!\texttt{cout}}*)(*\index{\texttt{std}}*)(*\index{\texttt{cout}|see{Input-output streams}}*)
\end{lstlisting}
\begin{enumerate}
\item[$\Rightarrow$] \textbf{using namespace} directives for including the whole namespace:
\end{enumerate}
\begin{lstlisting}
using namespace std;(*\index{Namespaces!\textbf{using namespace} directives}*)(*\index{\texttt{std}}*)(*\index{\textbf{using}|see{Namespaces}}*)
\end{lstlisting}
\begin{enumerate}
\item[$\Rightarrow$] An \emph{alias} is a symbolic name that means exactly the same as what it refers to:
\end{enumerate}
\begin{lstlisting}
using value_type = int; // (*value\_type*) means (*int*)(*\index{Alias!{\textbf{using}}}*)(*\index{\textbf{using}|see{Alias}}*)
using pchar = char*; // (*pchar*) means (*char$\ast$*)(*\index{Alias!{\textbf{using}}}*)
\end{lstlisting}
%
% Random numbers
%
\section{Random numbers}
\small
\begin{enumerate}
\item[$\Rightarrow$] C-style:
\end{enumerate}
\begin{lstlisting}
#include <cstdlib>(*\index{\texttt{<cstdlib>}}*)
#include <ctime>(*\index{\texttt{<ctime>}}*)

// seed the generator(*\index{Random numbers!seed the generator}*)
srand( time(0) );(*\index{\texttt{srand}}*)(*\index{\texttt{time}}*)

// integer random number between (*0*) and (*RAND\_MAX*)(*\index{Random numbers!integer random number}*)
int n = rand();(*\index{\texttt{rand}}*)
\end{lstlisting}
\begin{enumerate}
\item[$\Rightarrow$] C++ style:
\end{enumerate}
\begin{lstlisting}
#include <random>(*\index{\texttt{<random>}}*)
#include <functional>(*\index{\texttt{<functional>}}*)

// (*bind*) constructs a function object that calls its first argument
// with its second argument

// normal distribution by using the default random engine
auto gen_def = bind(normal_distribution<double>{15,4.0},(*\index{\texttt{bind}}*)(*\index{Random numbers!normal distribution}*)
                    default_random_engine{});(*\index{Random numbers!default random engine}*)(*\index{\texttt{default\_random\_engine}}*)
                    
// normal distribution by using the Mersenne Twister engine
// with seed (*91586*)                  
auto gen_mt = bind(normal_distribution<double>{15,4.0},(*\index{\texttt{bind}}*)(*\index{Random numbers!normal distribution}*)(*\index{\texttt{normal\_distribution}}*)
                    mt19937_64{91586});(*\index{Random numbers!Mersenne Twister engine}*)(*\index{\texttt{mt19937\_64}}*)

// call the function objects to get random numbers
cout << gen_def() << gen_mt() << endl;
\end{lstlisting}
%
% Bitsets
%
\section{Bitsets}
\small
\begin{enumerate}
\item[$\Rightarrow$] Bits and bit operations:\index{Bitsets!bits and bit operations}
\index{range-for-loop}
\end{enumerate}
\begin{lstlisting}
#include <bitset>(*\index{\texttt{<bitset>}}*)

bitset<8> a{87};   // (*01010111*)(*\index{\texttt{bitset<N>}}*)
bitset<8> b{0x87}; // (*10000111*)
    
cout << a << endl << b << endl;
    
// boolean and
bitset<8> c = a & b;
    
cout << c << endl; // (*00000111*)
\end{lstlisting}
\begin{enumerate}
\item[$\Rightarrow$] Access to single bits:\index{Bitsets!access to single bits}
\index{range-for-loop}
\end{enumerate}
\begin{lstlisting}
// prints out single bits reverting the order
// (*11100000*)
for (auto i=0; i<8; ++i)
     cout << c[i];
\end{lstlisting}
%
%
% Arrays
%
\section{Arrays}
\small
\begin{enumerate}
\item[$\Rightarrow$] Declaration and initialization:
\end{enumerate}
\begin{lstlisting}
// array of length (*4*) initialized to all zeros
int array[4]{};(*\index{Arrays!declaration and initialization}*)

// array of length (*4*) initialized to (*2,4,6,8*)
int array[]{2, 4, 6, 8};(*\index{Arrays!declaration and initialization}*)
\end{lstlisting}
\begin{enumerate}
\item[$\Rightarrow$] Length of an array using \textbf{sizeof}:
\end{enumerate}
\begin{lstlisting}
// array of length (*4*) initialized to (*2,4,6,8*)
int array[]{2, 4, 6, 8};
size_t array_size = sizeof(array) / sizeof(int);(*\index{\textbf{sizeof}}*)(*\index{\texttt{size\_t}}*)(*\index{Arrays!length calculation using \textbf{sizeof}}*)
\end{lstlisting}
\begin{enumerate}
\item[$\Rightarrow$] Range-based \textbf{for} statement:
\index{range-for-loop}
\end{enumerate}
\begin{lstlisting}
// changes the values and outputs (*3579*)
int array[]{2, 4, 6, 8};(*\index{Arrays!declaration and initialization}*)

for (int& x : array)
  x++;(*\index{Arrays!modifying elements of an array}*)
\end{lstlisting}
\begin{enumerate}
\item[$\Rightarrow$] \textbf{auto} lets the compiler use the type of the elements in the container because it knows the type already:\index{range-for-loop}
\end{enumerate}
\begin{lstlisting}
for (auto x : arr)(*\index{\textbf{auto}}*)(*\index{range-for-loop}*)
  cout << x;(*\index{Arrays!printing elements of an array}*)(*\index{Input-output streams!\texttt{cout}}*)
\end{lstlisting}
%
% Pointers
%
\section{Pointers}
\small
\begin{enumerate}
\item[$\Rightarrow$] Simple object:
\end{enumerate}
\begin{lstlisting}
// simple pointer to (*double*)(*\index{Pointers!simple pointer}*)
double *d = new double{5.123};(*\index{\textbf{new}}*)

// read
double dd = *d;(*\index{Pointers!dereference operator \textbf{$\ast$}}*)

// write
*d = -11.234;

// delete the storage on the free store(*\index{Pointers!free store}*)
delete d;(*\index{\textbf{delete}}*)

// reassign: now (*d*) points to (*dd*)
d = &dd;(*\index{Pointers!address of operator \texttt{\&}}*)
\end{lstlisting}
\begin{enumerate}
\item[$\Rightarrow$] Dynamic array:\index{Dynamic array|see{Pointers}}
\end{enumerate}
\begin{lstlisting}
// dynamic array of (*10 double*)s
double *dd = new double[10] {0,1,2,3,4,5,6,7,8,9};(*\index{Pointers!dynamic array!allocation}*)(*\index{\textbf{new}}*)(*\index{\textbf{new}}*)

// delete the storage on the free store
delete [] dd;(*\index{Pointers!dynamic array!deallocation}*)(*\index{\textbf{delete}}*)
\end{lstlisting}
\begin{enumerate}
\item[$\Rightarrow$] Dynamic matrix:\index{Dynamic bidimensional array|see{Pointers}}
\end{enumerate}
\begin{lstlisting}
// dynamic matrix of (*5 x 5 double*)s memory allocation(*\index{Pointers!dynamic matrix!allocation}*)
double **m = new double*[5];(*\index{\textbf{new}}*)(*\index{\textbf{new}}*)
for (int i=0; i<5; i++)
    m[i] = new double[5];(*\index{\textbf{new}}*)

// memory initialization    
for (int i=0; i<5; i++)
    for (int j=0; j<5; j++)
        m[i][j] = i*j;(*\index{Pointers!subscript operator\texttt{[]}}*)

// memory deallocation(*\index{Pointers!dynamic matrix!deallocation}*)
for (int i=0; i<5; i++)
    delete[] m[i];(*\index{\textbf{delete}}*)
delete[] m;(*\index{\textbf{delete}}*)
\end{lstlisting}
\begin{enumerate}
\item[$\Rightarrow$] \texttt{unique\_ptr}:
Holds ownership of a dynamic object according to RAII, i.e. resource acquisition is initialization\index{Pointers!RAII}. It will 
automatically destroy the object if needed.
\end{enumerate}
\begin{lstlisting}
#include <memory>(*\index{\texttt{<memory>}}*)

MyVector<int>* my_function()
{
    unique_ptr<MyVector<int>*> p { new MyVector<int>};(*\index{\textbf{new}}*)(*\index{\texttt{unique\_ptr}|see{Pointers}}*)(*\index{Pointers!\texttt{unique\_ptr}}*)
    /* ... */
    /* if something goes wrong, deletes the object */
   /* ... */   
    return p.release(); // returns the pointer
}
\end{lstlisting}
%
% References
%
\section{References}
\small
\begin{enumerate}
\item[$\Rightarrow$] A variable reference must be initialized with a variable being referred to:
\end{enumerate}
\begin{lstlisting}
// an integer amount
int amount = 12;

// reference to (*amount*)
int& ref_amount = amount;(*\index{Reference}*)

// outputs (*12*)
cout << "amount = " << amount << endl;
cout << "ref_amount = " << ref_amount << endl;
\end{lstlisting}
\begin{enumerate}
\item[$\Rightarrow$] A variable reference cannot be made referring to another variable at runtime:
\end{enumerate}
\begin{lstlisting}
// a new integer amount
int new_amount = 24;

// (*ref\_amount*) still refers to (*amount*)
ref_amount = new_amount;

// outputs (*24*)
cout << "amount = " << amount << endl;
cout << "ref_amount = " << ref_amount << endl;
\end{lstlisting}
\begin{enumerate}
\item[$\Rightarrow$] A variable reference can be used as input and output parameter of a function:
\end{enumerate}
\begin{lstlisting}
// function definition taking a reference
void do_something(int& in_out_var)
{
    in_out_var *= 2;
}

int a_variable = 24;

do_something(a_variable);

// outputs (*48*)
cout << "a_variable = " << a_variable << endl;
\end{lstlisting}
%
% Storage classes
%
\section{Storage classes}
\small
The storage class\index{Storage class} defines the memory type where an object is stored. The lifetime\index{Lifetime} of an object is from the time it is first initialized until it is destroyed.
\begin{enumerate}
\item[$\Rightarrow$] A simple class for objects:
\end{enumerate}
\begin{lstlisting}
class Object(*\index{Classes!\textbf{class}!\texttt{Object}}*)
{
public:
    Object(string obj_name) : name{obj_name} {
        cout << "Created object: " << obj_name << endl;
    }
    ~Object() { cout << "Destroyed object: " << name << endl; }
private:
    string name{};
};
\end{lstlisting}
\begin{enumerate}
\item[$\Rightarrow$] Static storage with global scope, external or internal linkage, i.e. storage is allocated before the program starts and deallocated when the program ends:
\end{enumerate}
\begin{lstlisting}
// static storage, external linkage
Object a{"a"};(*\index{Storage class!static!external linkage}*)

// static storage, internal linkage
static Object b{"b"};(*\index{Storage class!static!internal linkage}*)
\end{lstlisting}
\begin{enumerate}
\item[$\Rightarrow$] Static storage with local scope, i.e. storage is allocated the first time the function is called and deallocated when the program ends:
\end{enumerate}
\begin{lstlisting}
void func(void)
{
    // static storage, local variable
    static Object c{"c"};(*\index{Storage class!static!local variable}*)
}

int main()
{
    // Object (*c*) is allocated
    func();
}
\end{lstlisting}
\begin{enumerate}
\item[$\Rightarrow$] Automatic storage with local scope, i.e. storage is allocated on the stack when the local scope is entered and deallocated after execution leaves the scope:
\end{enumerate}
\begin{lstlisting}
int main()
{
     // automatic storage object
    {
        Object d{"d"};(*\index{Storage class!automatic}*)
    }   
}
\end{lstlisting}
\begin{enumerate}
\item[$\Rightarrow$] Dynamic storage, i.e. storage is allocated dynamically on the heap with \textbf{new} and deallocated explicitly with \textbf{delete}:
\end{enumerate}
\begin{lstlisting}
int main()
{
    // dynamic storage object
    Object* e = new Object{"e"};(*\index{Storage class!dynamic}*)(*\index{\textbf{new}}*)
    delete(e);(*\index{\textbf{delete}}*)
}
\end{lstlisting}
\begin{enumerate}
\item[$\Rightarrow$] Example of object declarations:
\end{enumerate}
\begin{lstlisting}
// static storage object, external linkage
Object a{"a"};

// static storage object, internal linkage
static Object b{"b"};

void func(void)
{
    cout<< "Start of func()" << endl;
    
    // static storage, local variable
    static Object c{"c"};
    
    cout<< "End of func()" << endl;
}

int main()
{
    cout<< "Start of main()" << endl;
    
    func();
    
    // local scope
    {
        cout<< "Start of local scope" << endl;
        
        // automatic storage object
        Object d{"d"};
        
        cout<< "End of local scope" << endl;
    }
    
    // dynamic storage object
    Object* e = new Object{"e"};
    delete(e);
    
    cout<< "End of main()" << endl;
    return 0;
}
\end{lstlisting}
\begin{enumerate}
\item[$\Rightarrow$] Example result of the order of allocation and deallocation:
\end{enumerate}
\begin{lstlisting}
// Output printed
Created object: a
Created object: b
Start of main()
Start of func()
Created object: c
End of func()
Start of local scope
Created object: d
End of local scope
Destroyed object: d
Created object: e
Destroyed object: e
End of main()
Destroyed object: c
Destroyed object: b
Destroyed object: a
Program ended with exit code: 0
\end{lstlisting}

%
% C-Strings
%
\section{C-Strings}
\index{C-Strings}
\small
\begin{enumerate}
\item[$\Rightarrow$] Legacy strings from C:
\end{enumerate}
\begin{lstlisting}
#include <cstring>(*\index{\texttt{<cstring>}}*)
#include <cstdlib>(*\index{\texttt{<cstdlib>}}*)

// (*C-string*) for max (*10*) characters(*\index{C-Strings!definition}*)
// long string + null char '\0'
const int SIZE = 10 + 1;(*\index{Constants!\textbf{const}}*)
char msg[SIZE] = "Hello!";
\end{lstlisting}
\begin{enumerate}
\item[$\Rightarrow$] Checking for end of string when looping:\index{C-Strings!end of string}
\end{enumerate}
\begin{lstlisting}
// correct looping over (*C-string*)s(*\index{C-Strings!correct looping}*)
int i = 0;
while ( msg[i] != '\0' && i < SIZE)
{
   // process msg[i]
}
\end{lstlisting}
\begin{enumerate}
\item[$\Rightarrow$] Safe C-string operations:
\end{enumerate}
\begin{lstlisting}
// safe string copy, at most (*10*) characters are copied(*\index{C-Strings!safe looping}*)
strncpy(msg, srcStr, 10);(*\index{\texttt{strncpy}}*)

// safe string compare, at most (*10*) characters are compared(*\index{C-Strings!safe compare}*)
strncmp(msg, srcStr, 10);(*\index{\texttt{strncmp}}*)

// safe string concatenation, at most (*10*) characters are concatenated(*\index{C-Strings!safe concatenation}*)
strncat(msg, srcStr, 10);(*\index{\texttt{strncat}}*)
\end{lstlisting}
\begin{enumerate}
\item[$\Rightarrow$] Conversions:
\end{enumerate}
\begin{lstlisting}
// from (*C-string*) to (*int*), (*long*), (*float*)
int    n = atoi("567");(*\index{C-Strings!conversions!to integer}*)(*\index{\texttt{atoi}}*)
long   n = atol("1234567");(*\index{C-Strings!conversions!to long integer}*)(*\index{\texttt{atol}}*)
double n = atof("12.345");(*\index{C-Strings!conversions!to double}*)(*\index{\texttt{atof}}*)
\end{lstlisting}
%
% Input-output streams
%
\section{Input-output streams}
\index{Input-output streams}
\small
\begin{enumerate}
\item[$\Rightarrow$] Input stream \texttt{cin}, output stream \texttt{cout}, error stream \texttt{cerr}:
\index{Input-output streams!input stream|see{\texttt{cin}}}
\index{Input-output streams!output stream|see{\texttt{cout}}}
\index{Input-output streams!error stream|see{\texttt{cerr}}}
\end{enumerate}
\begin{lstlisting}
int number;
char ch;

// read a number followed by a character
// from standard input (keyboard)
// (ignores whitespaces, newlines, etc.)
cin >> number >> ch;(*\index{Input-output streams!reading from the keyboard}*)(*\index{Input-output streams!\texttt{cin}}*)

// write on standard output (display)
cout << number << " " <<  ch << endl;(*\index{Input-output streams!writing to the screen}*)(*\index{Input-output streams!\texttt{cout}}*)

// write error message on standard error (display)
cerr << "Wrong input!\n";(*\index{Input-output streams!writing error message to the screen}*)(*\index{Input-output streams!\texttt{cerr}}*)(*\index{\texttt{cerr}|see{Input-output streams}}*)
\end{lstlisting}
\begin{enumerate}
\item[$\Rightarrow$] Integer format manipulators\\ \\ Once a  manipulator is set, it stays until another one is set, i.e. manipulators are \emph{sticky}:
\index{Input-output streams!integer format manipulators}
\end{enumerate}
\begin{lstlisting}
#include <iomanip>(*\index{\texttt{<iomanip>}}*)

// set decimal, octal, or hexadecimal notation,
// and show the base, i.e. (*0*) for octal and (*0x*) for hexadecimal
cout << showbase;(*\index{Input-output streams!integer format manipulators!show the base}*)(*\index{Input-output streams!\texttt{cout}}*)(*\index{Input-output streams!\texttt{cout}!\texttt{showbase}}*)
cout << dec << 1974 << endl;(*\index{Input-output streams!integer format manipulators!decimal}*)(*\index{Input-output streams!\texttt{cout}!\texttt{dec}}*)
cout << oct << 1974 << endl;(*\index{Input-output streams!integer format manipulators!otctal}*)(*\index{Input-output streams!\texttt{cout}!\texttt{oct}}*)
cout << hex << 1974 << endl;(*\index{Input-output streams!integer format manipulators!hexadecimal}*)(*\index{Input-output streams!\texttt{cout}!\texttt{hex}}*)
cout << noshowbase;(*\index{Input-output streams!integer format manipulators!don't show the base}*)(*\index{Input-output streams!\texttt{cout}!\texttt{noshowbase}}*)

// values can be read from input in decimal, octal(*\index{Input-output streams!integer format manipulators!reading a value from the keyboard in any notation}*)
// or hexadecimal format previous unsetting
// of all the flags
cin.unsetf(ios::dec);(*\index{Input-output streams!\texttt{cin}!\texttt{unset}}*)(*\index{Input-output streams!\texttt{cin}!\texttt{unset}!\texttt{ios::dec}}*)
cin.unsetf(ios::oct);(*\index{Input-output streams!\texttt{cin}!\texttt{unset}!\texttt{ios::oct}}*)
cin.unsetf(ios::hex);(*\index{Input-output streams!\texttt{cin}!\texttt{unset}!\texttt{ios::hex}}*)

//  now (*val*) can be inserted in any format
cin >> val;(*\index{Input-output streams!\texttt{cin}}*)
\end{lstlisting}
\begin{enumerate}
\item[$\Rightarrow$] Floating point format manipulators\\ \\ Once a  manipulator is set, it stays until another one is set, i.e. manipulators are \emph{sticky}:
\index{Input-output streams!floating point format manipulators}
\end{enumerate}
\begin{lstlisting}
#include <iomanip>(*\index{\texttt{<iomanip>}}*)

// set default, fixed, or scientific notation
cout << defaultfloat << 1023.984;(*\index{Input-output streams!floating point format manipulators!default float notation}*)(*\index{Input-output streams!\texttt{cout}!\texttt{defaultfloat}}*)
cout << fixed << 1023.984;(*\index{Input-output streams!floating point format manipulators!fixed notation}*)(*\index{Input-output streams!\texttt{cout}!\texttt{fixed}}*)
cout << scientific << 1023.984;(*\index{Input-output streams!floating point format manipulators!scientific notation}*)(*\index{Input-output streams!\texttt{cout}!\texttt{scientific}}*)

// set precision
cout << setprecision(2) << 1023.984;(*\index{Input-output streams!floating point format manipulators!precision}*)(*\index{Input-output streams!\texttt{cout}!\texttt{setprecision}}*)

// set character text width
cout << setw(10);(*\index{Input-output streams!floating point format manipulators! text width}*)(*\index{Input-output streams!\texttt{cout}!\texttt{setw}}*)

// set left or right alignment
cout << left  << 1023.984;(*\index{Input-output streams!floating point format manipulators!left aligned }*)(*\index{Input-output streams!\texttt{cout}!\texttt{left}}*)
cout << right << 1023.984;(*\index{Input-output streams!floating point format manipulators!right aligned}*)(*\index{Input-output streams!\texttt{cout}!\texttt{right}}*)

// always show decimal point and zeros
cout << showpoint << 0.532;(*\index{Input-output streams!floating point format manipulators!always show decimal point}*)(*\index{Input-output streams!\texttt{cout}!\texttt{showpoint}}*)

// always show plus sign
cout << showpos << 3.64;(*\index{Input-output streams!floating point format manipulators!always show plus sign}*)(*\index{Input-output streams!\texttt{cout}!\texttt{showpos}}*)
\end{lstlisting}
\begin{enumerate}
\item[$\Rightarrow$] Single characters read and write:
\index{Input-output streams!reading and writing characters}
\end{enumerate}
\begin{lstlisting}
// read any character from (*cin*) (doesn't skip spaces, newlines, etc.)(*\index{Input-output streams!reading and writing characters!read any character}*)
char nextChar;
cin.get(nextChar);(*\index{Input-output streams!\texttt{cin}!\texttt{get}}*)

// loop for keeping reading
// stops when end of line control character (control-d)(*\index{end of line control character}*)
// is inserted
while ( cin.get(nextChar) )(*\index{Input-output streams!\texttt{cin}!\texttt{get}}*)
{
    // process character
}

// write a character to (*cout*)(*\index{Input-output streams!reading and writing characters!write a single character}*)
cout.put(nextChar)(*\index{Input-output streams!\texttt{cout}!\texttt{put}}*)

// read a whole line of 80 chars(*\index{Input-output streams!reading and writing characters!read a whole line}*)
char line[80+1];
cin.getline(line,81);(*\index{Input-output streams!\texttt{cin}!\texttt{getline}}*)

// put back (*nextChar*) to (*cin*), (*nextChar*) will be the next(*\index{Input-output streams!reading and writing characters!putting a character back into the input stream}*)
// char read by (*cin.get()*)
cin.putback(nextChar);(*\index{Input-output streams!\texttt{cin}!\texttt{putback}}*)

// put back the last char got from (*cin.get()*) to (*cin*)(*\index{Input-output streams!reading and writing characters!putting the last character back into the input stream}*)
cin.unget();(*\index{Input-output streams!\texttt{cin}!\texttt{unget}}*)
\end{lstlisting}
\begin{enumerate}
\item[$\Rightarrow$] If the input pattern is unexpected, it is possible to set the state of \texttt{cin} to failed:
\index{Input-output streams!handling of unexpected input}
\end{enumerate}
\begin{lstlisting}
try(*\index{\textbf{try-catch}|see{Exceptions}}*)(*\index{Exceptions!\textbf{try-catch}}*)
{
    // check for unexpected input
    char ch;
    if ( cin >> ch && ch != expected_char )(*\index{Input-output streams!\texttt{cin}}*)
    {
        // put back last character read
        cin.unget();(*\index{Input-output streams!\texttt{cin}!\texttt{unget}}*)
        
        // set failed bit(*\index{Input-output streams!handling of unexpected input!setting explicitly the failure bit}*)
        cin.clear(ios_base::failbit);(*\index{Input-output streams!\texttt{cin}!\texttt{clear}}*)
    
        // throw an exception or deal with failed stream
        throw runtime_error("Unexpected input");(*\index{\textbf{throw}}*)(*\index{Exceptions!\texttt{runtime\_error}}*)
    }
}
catch (runtime_error e)
{
    cerr << "Error! " << e.what() << "\n";(*\index{Input-output streams!\texttt{cerr}}*)(*\index{Exceptions!\texttt{runtime\_error}!\texttt{what}}*)
            
    // check for failure
    if (cin.fail())(*\index{Input-output streams!\texttt{cin}!\texttt{fail}}*)
    {
        // clear failed bit(*\index{Input-output streams!handling of unexpected input!clearing the failed state of the input stream}*)
        cin.clear();(*\index{Input-output streams!\texttt{cin}!\texttt{clear}}*)
                
        // read wrong input
        string wrong_input;(*\index{Strings!\texttt{string}}*)
        cin >> wrong_input;(*\index{Input-output streams!\texttt{cin}}*)
                    
        cerr << "Got '" << wrong_input[0] << "'\n";(*\index{Input-output streams!\texttt{cerr}}*)
    }
     // End of file (eof) or corrupted state (bad)
    else return 1;
}
\end{lstlisting}
%
% Files
%
\section{Files}
\small
\begin{enumerate}
\item[$\Rightarrow$] Accessed by means of \texttt{ifstream} (input) or \texttt{ofstream} (output) objects:
\end{enumerate}
\begin{lstlisting}
#include <fstream>(*\index{\texttt{<fstream>}}*)

// open input file(*\index{Files!opening as input}*)
ifstream in_stream {"infile.dat"};(*\index{\texttt{ifstream}}*)
// open output file(*\index{Files!opening as output}*)
ofstream out_stream {"outfile.dat"};(*\index{\texttt{ofstream}}*)
\end{lstlisting}
\begin{enumerate}
\item[$\Rightarrow$] Accessed both in input and output mode by means of \texttt{fstream}\index{\texttt{fstream}} objects (not recommended):
\end{enumerate}
\begin{lstlisting}
#include <fstream>(*\index{\texttt{<fstream>}}*)

// open file in both input and output mode(*\index{Files!opening both as input and output}*)
fstream fs{"inoutfile.dat", ios_base::in | ios_base::out};(*\index{\texttt{fstream}}*)
\end{lstlisting}
\begin{enumerate}
\item[$\Rightarrow$] Opened explicitly (not recommended):
\end{enumerate}
\begin{lstlisting}
#include <fstream>(*\index{\texttt{<fstream>}}*)

// input file 
ifstream in_stream;(*\index{\texttt{ifstream}}*)
// output file
ofstream out_stream;(*\index{\texttt{ofstream}}*)

// open files(*\index{Files!opening explicitly}*)
in_stream.open("infile.dat");
out_stream.open("outfile.dat");
\end{lstlisting}
\begin{enumerate}
\item[$\Rightarrow$] When checking for failure, the status flag needs to be cleared in order to continue working with the file:\index{Files!checking for failure}
\end{enumerate}
\begin{lstlisting}
// check for failure on input file
if ( !in_stream )
{
    if ( in_stream.bad() ) error("stream corrupted!");(*\index{Files!checking for failure!corrupted stream}*)
    
    if ( in_stream.eof() )(*\index{Files!checking for failure!end of file}*)
    { 
        // no more data available
    }
    
    if ( in_stream.fail() )(*\index{Files!checking for failure!format data error}*)
    {
        // some format data error, e.g. expected
        // an integer but a string was read
        // recovery is still possible
        
        // set back the state to good 
        // before attempting to read again
        in_stream.clear();(*\index{Files!checking for failure!setting back to good state}*)
        
        // read again
        string wrong_input;(*\index{Strings!\texttt{string}}*)
        in_stream >> wrong_input;
    }
}
\end{lstlisting}
\begin{enumerate}
\item[$\Rightarrow$] As for the standard input, if the input pattern is unexpected, it is possible to set the state of the file to failed and try to recover somewhere else, e.g. by throwing an exception:
\end{enumerate}
\begin{lstlisting}
try(*\index{Exceptions!\textbf{try-catch}}*)
{
    // check for unexpected input(*\index{Files!checking for unexpected input}*)
    char ch;
    if ( in_stream >> ch && ch != expected_char )
    {
        // put back last character read
        in_stream.unget();
        
        // set failed bit
        in_stream.clear(ios_base::failbit);
    
        // throw an exception or deal with failed stream
        throw runtime_error("Unexpected input");(*\index{\textbf{throw}}*)(*\index{Exceptions!\texttt{runtime\_error}}*)
    }
}
catch (runtime_error e)
{
    cerr << "Error! " << e.what() << "\n";(*\index{Input-output streams!\texttt{cerr}}*)(*\index{Exceptions!\texttt{runtime\_error}!\texttt{what}}*)
            
    // check for failure
    if (in_stream.fail())
    {
        // clear failed bit
        in_stream.clear();
                
        // read wrong input
        string wrong_input;(*\index{Strings!\texttt{string}}*)
        in_stream >> wrong_input;
                    
        cerr << "Got '" << wrong_input[0] << "'\n";(*\index{Input-output streams!\texttt{cerr}}*)
    }
    // end-of-file or bad state
    else return 1;
}
\end{lstlisting}
\begin{enumerate}
\item[$\Rightarrow$] Read and write:
\end{enumerate}
\begin{lstlisting}
// read/write data(*\index{Files!reading and writing}*)
in_stream >> data1 >> data2;
out_stream << data1 << data2;
\end{lstlisting}
\begin{enumerate}
\item[$\Rightarrow$] Read a line:
\end{enumerate}
\begin{lstlisting}
string line;(*\index{Strings!\texttt{string}}*)
getline(in_stream, line);(*\index{Files!reading a line}*)
\end{lstlisting}
\begin{enumerate}
\item[$\Rightarrow$] Ignore input (extract and discard):
\end{enumerate}
\begin{lstlisting}
// ignore up to a newline or 9999 characters(*\index{Files!ignoring input}*)
in_stream.ignore(9999,'\n');
\end{lstlisting}
\begin{enumerate}
\item[$\Rightarrow$] Move the file pointer:
\end{enumerate}
\begin{lstlisting}
// skip 5 characters when reading (seek get)
in_stream.seekg(5);(*\index{Files!moving the file pointer!reading with \texttt{seekg} (seek get)}*)
// skip 8 characters when writing (seek put)(*\index{Files!moving the file pointer!writing with \texttt{seekp} (seek put)}*)
out_stream.seekp(8);
\end{lstlisting}
\begin{enumerate}
\item[$\Rightarrow$] Checking for end of file:
\end{enumerate}
\begin{lstlisting}
// the failing read sets the EOF flag but avoids further processing
while ( in_stream >> next )(*\index{Files!loop for reading all the input}*)
{
    // process next
}

// check the EOF flag
if ( in_stream.eof() )
    cout << "EOF reached!" << endl;(*\index{Input-output streams!\texttt{cout}}*)
\end{lstlisting}
\begin{enumerate}
\item[$\Rightarrow$] When a file object gets out of scope, the file is closed automatically, but explicit
close is also possible (not recommended):\index{Files!closing by going out of scope}
\end{enumerate}
\begin{lstlisting}
// explicitily close files(*\index{Files!closing explicitly}*)
in_stream.close();
out_stream.close()
\end{lstlisting}
%
% Strings
%
\section{Strings}
\small
\begin{enumerate}
\item[$\Rightarrow$] Strings as supported by the C++ standard library:
\end{enumerate}
\begin{lstlisting}
#include <string>(*\index{\texttt{<string>}}*)

// initialization(*\index{Strings!initialization}*)
string s1 = "Hello";(*\index{Strings!\texttt{string}}*)
string s2("World");(*\index{Strings!\texttt{string}}*)
string s3{"World"};(*\index{Strings!\texttt{string}}*)
\end{lstlisting}
\begin{enumerate}
\item[$\Rightarrow$] Fill constructor:
\end{enumerate}
\begin{lstlisting}
string s4{string(5,'*')}; // fill constructor (*"*****"*)(*\index{Strings!\texttt{string}}*)(*\index{Strings!fill constructor}*)
\end{lstlisting}
\begin{enumerate}
\item[$\Rightarrow$] Substring constructor from a certain position and a given optional length:
\end{enumerate}
\begin{lstlisting}
string s5{string(string{"Hello, world!"},0,5)}; // (*"Hello"*)(*\index{Strings!\texttt{string}}*)(*\index{Strings!substring constructor}*)
string s6{string(string{"Hello, world!"},7)}; // (*"world!"*)(*\index{Strings!\texttt{string}}*)(*\index{Strings!substring constructor}*)
\end{lstlisting}
\begin{enumerate}
\item[$\Rightarrow$] Buffer constructor from a character array:
\end{enumerate}
\begin{lstlisting}
string s7{string("Hello, world!",5)}; // (*"Hello"*)(*\index{Strings!\texttt{string}}*)(*\index{Strings!buffer constructor}*)
\end{lstlisting}
\begin{enumerate}
\item[$\Rightarrow$] Concatenation:
\end{enumerate}
\begin{lstlisting}
// concatenation(*\index{Strings!concatenation}*)
string s3 = s1 + ", " + s2;(*\index{Strings!\texttt{string}}*)(*\index{Strings!\texttt{+}}*)
\end{lstlisting}
\begin{enumerate}
\item[$\Rightarrow$] Read a line:
\end{enumerate}
\begin{lstlisting}
// read a line(*\index{Strings!reading a line}*)
string line;(*\index{Strings!\texttt{string}}*)
getline(cin,line);(*\index{Input-output streams!\texttt{cin}}*)
\end{lstlisting}
\begin{enumerate}
\item[$\Rightarrow$] Access to a character:
\end{enumerate}
\begin{lstlisting}
// access to the ith character (no illegal index checking)(*\index{Strings!access to character!no illegal index checking}*)
s1[i];

// access to the ith character (with illegal index checking)(*\index{Strings!access to character!with illegal index checking}*)
s1.at(i);(*\index{Strings!\texttt{at}}*)
\end{lstlisting}
\begin{enumerate}
\item[$\Rightarrow$] Append:
\end{enumerate}
\begin{lstlisting}
// append
s1.append(s2);(*\index{Strings!\texttt{append}}*)
\end{lstlisting}
\begin{enumerate}
\item[$\Rightarrow$] Size and length:
\end{enumerate}
\begin{lstlisting}
// size and length
s1.size();(*\index{Strings!\texttt{size}}*)
s1.length();(*\index{Strings!\texttt{length}}*)
\end{lstlisting}
\begin{enumerate}
\item[$\Rightarrow$] Substring:
\end{enumerate}
\begin{lstlisting}
// substring from position 5 and length 4 characters
string substring;
substring = s4.substr(5,4);(*\index{Strings!\texttt{substr}}*)
\end{lstlisting}
\begin{enumerate}
\item[$\Rightarrow$] Find:
\end{enumerate}
\begin{lstlisting}
// find (returns string::npos if not found)
size_t pos;(*\index{\texttt{size\_t}}*)
pos = s3.find("World");(*\index{Strings!\texttt{find}}*)
if (pos == string::npos)
    cerr << "Error: String not found!\n";(*\index{Input-output streams!\texttt{cerr}}*)

// find starting from position 5
s3.find("l",5);(*\index{Strings!\texttt{find}}*)
\end{lstlisting}
\begin{enumerate}
\item[$\Rightarrow$] C-string:
\end{enumerate}
\begin{lstlisting}
// C-string(*\index{Strings!\texttt{c\_str}}*)
s3.c_str();
\end{lstlisting}
\begin{enumerate}
\item[$\Rightarrow$] Conversions:
\end{enumerate}
\begin{lstlisting}
// from (*string*) to (*int*), (*long*), (*float*)
int    n = stoi("456");(*\index{Strings!from string to!integer}*)(*\index{\texttt{stoi}}*)
long   n = stol("1234567");(*\index{Strings!from string to!long integer}*)(*\index{\texttt{stol}}*)
double n = stod("12.345");(*\index{Strings!from string to!double}*)(*\index{\texttt{stod}}*)

// from numeric type to (*string*)(*\index{Strings!from numeric type to string}*)
string s = to_string(123.456);(*\index{\texttt{to\_string}}*)(*\index{Strings!\texttt{string}}*)
\end{lstlisting}
%
% String streams
%
\section{String streams}
\small
A string is used as a source for an input stream or as a target for an output stream. \\
\begin{enumerate}
\item[$\Rightarrow$] Input string stream: \texttt{istringstream}
\end{enumerate}
\begin{lstlisting}
#include <sstream>(*\index{\texttt{<sstream>}}*)

// input string stream(*\index{String streams!input string stream}*)
istringstream data_stream{"1.234 -5643.32"};(*\index{\texttt{istringstream}}*)

// read numbers from data stream(*\index{String streams!read numbers from data stream}*)
double val;
while ( is >> val )
    cout << val << endl;(*\index{Input-output streams!\texttt{cout}}*)
\end{lstlisting}
\begin{enumerate}
\item[$\Rightarrow$] Output string stream: \texttt{ostringstream}
\end{enumerate}
\begin{lstlisting}
#include <sstream>(*\index{\texttt{<sstream>}}*)

// output string stream(*\index{String streams!output string stream}*)
ostringstream data_stream;(*\index{\texttt{ostringstream}}*)

// the same manipulators of input-output streams
// can be used(*\index{String streams!manipulators|see{Input-output streams!integer format manipulators}}*)(*\index{String streams!manipulators|see{Input-output streams!floating format manipulators}}*)
data_stream << fixed << setprecision(2) << showpos;
data_stream << 6.432 << " " << -313.2134 << "\n";

// the (*str()*) method returns the string in the stream(*\index{String streams!return the string in the stream}*)
cout << data_stream.str();(*\index{Input-output streams!\texttt{cout}}*)
\end{lstlisting}
%
% Vectors
%
\section{Vectors}
\small
\begin{enumerate}
\item[$\Rightarrow$] Vectors\index{Vectors} as supported by the C++ standard library:
\end{enumerate}
\begin{lstlisting}
#include <vector>(*\index{\texttt{<vector>}}*)

// vector with base type (*int*)
vector<int> v = {2, 4, 6, 8};(*\index{Vectors!initialized with initializer list}*)(*\index{Containers!\texttt{vector}}*)

// vector with 10 elements all initialized to 0
vector<int> v(10);(*\index{Vectors!initialized with all elements to 0}*)(*\index{Containers!\texttt{vector}}*)
\end{lstlisting}
\begin{enumerate}
\item[$\Rightarrow$] Access:
\end{enumerate}
\begin{lstlisting}
// unchecked access to the (*i*)th element
cout << v[i];(*\index{Vectors!access to an element!unchecked}*)(*\index{Input-output streams!\texttt{cout}}*)

// checked access to the (*i*)th element
cout << v.at(i);(*\index{Vectors!access to an element!checked}*)(*\index{Input-output streams!\texttt{cout}}*)(*\index{Containers!\texttt{vector}!\texttt{at}}*)
\end{lstlisting}
\begin{enumerate}
\item[$\Rightarrow$] Add:
\end{enumerate}
\begin{lstlisting}
// add an element
v.push_back(10);(*\index{Vectors!\texttt{push\_back}}*)(*\index{Containers!\texttt{vector}!\texttt{push\_back}}*)
\end{lstlisting}
\begin{enumerate}
\item[$\Rightarrow$] Resize:
\end{enumerate}
\begin{lstlisting}
// resize to 20 elements
// new elements are initialized to 0
v.resize(20);(*\index{Vectors!\texttt{resize}}*)(*\index{Containers!\texttt{vector}!\texttt{resize}}*)
\end{lstlisting}
\begin{enumerate}
\item[$\Rightarrow$] Loop over:
\end{enumerate}
\begin{lstlisting}
// range-for-loop(*\index{range-for-loop}*)
for (auto x : v)(*\index{Vectors!loop over elements}*)(*\index{\textbf{auto}}*)
  cout << x << endl;(*\index{Input-output streams!\texttt{cout}}*)

// (*auto*) gives to (*x*) the same type of the element on the right
// hand side of the assignment, in this case a (*vector::iterator*)
for (auto x = v.begin(); x<v.end(); x++)(*\index{Vectors!loop over elements}*)(*\index{\textbf{auto}}*)(*\index{Iterators!\texttt{vector}!\texttt{iterator}}*)
  cout << *x << endl;(*\index{Input-output streams!\texttt{cout}}*)
\end{lstlisting}
\begin{enumerate}
\item[$\Rightarrow$] Size and capacity:
\end{enumerate}
\begin{lstlisting}
// size
cout << v.size();(*\index{Vectors!\texttt{size}}*)(*\index{Input-output streams!\texttt{cout}}*)(*\index{Containers!\texttt{vector}!\texttt{size}}*)

// capacity: number of elements currently allocated
cout << v.capacity();(*\index{Vectors!\texttt{capacity}}*)(*\index{Input-output streams!\texttt{cout}}*)(*\index{Containers!\texttt{vector}!\texttt{capacity}}*)
\end{lstlisting}
\begin{enumerate}
\item[$\Rightarrow$] Reserve more capacity:
\end{enumerate}
\begin{lstlisting}
// reserve (reallocate) more capacity e.g. at least 64 (*int*)s
v.reserve(64);(*\index{Vectors!\texttt{reserve}}*)(*\index{Containers!\texttt{vector}!\texttt{reserve}}*)
\end{lstlisting}
\begin{enumerate}
\item[$\Rightarrow$] Throws an \texttt{out\_of\_range} exception if accessed out of bounds:\index{Exceptions!\texttt{out\_of\_range}}
\end{enumerate}
\begin{lstlisting}
// out of bounds access
vector<int> v = {2, 4, 6, 8};(*\index{Containers!\texttt{vector}}*)

try(*\index{Exceptions!\textbf{try-catch}}*)
{
    cout << v.at(7);(*\index{Input-output streams!\texttt{cout}}*)
} catch (out_of_range e)(*\index{Vectors!out of range exception}*)
{
    // access error!
}
\end{lstlisting}
%
% Enumerations
%
\section{Enumerations}
\small
\begin{enumerate}
\item[$\Rightarrow$] \textbf{enum class} defines symbolic constants in the scope of the class:\index{Enumerations!in \textbf{class} scope}
\end{enumerate}
\begin{lstlisting}
// enum definition(*\index{Enumerations!definition}*)
enum class Weekdays(*\index{\textbf{enum class}}*)
{
    mon=1, tue, wed, thu, fri
};

// usage(*\index{Enumerations!usage}*)
Weekdays day = Weekdays::tue;
\end{lstlisting}
\begin{enumerate}
\item[$\Rightarrow$] \textbf{int}s cannot be assigned to \textbf{enum class} and vice versa:
\end{enumerate}
\begin{lstlisting}
// errors!(*\index{Enumerations!prohibited conversions}*)
Weekdays day = 3;
int d = Weekdays::wed;
\end{lstlisting}
\begin{enumerate}
\item[$\Rightarrow$] A conversion function should be written which uses unchecked conversions:
\end{enumerate}
\begin{lstlisting}
// valid(*\index{Enumerations!conversion function}*)
Weekdays day = Weekdays(2);
int d = int(Weekdays::fri);
\end{lstlisting}
%
% Classes
%
\section{Classes}
\small
\begin{enumerate}
\item[$\Rightarrow$] Class using dynamic arrays:\index{Classes!example of a vector class}
\end{enumerate}
\begin{lstlisting}
#include <algorithm>(*\index{\texttt{<algorithm>}}*)

class MyVector(*\index{\textbf{class}|see{Classes}}*)(*\index{Classes!\textbf{class}!\texttt{MyVector}}*)
{
public:
    // explicit constructor (avoids type conversions)
    explicit MyVector();(*\index{\textbf{explicit}}*)
    // explicit constructor with size parameter
    explicit MyVector(size_t);(*\index{\textbf{explicit}}*)(*\index{\texttt{size\_t}}*)
    // explicit constructor with initializer list
    explicit MyVector(initializer_list<double>);(*\index{\texttt{initializer\_list}}*)(*\index{\textbf{explicit}}*)
    // copy constructor (pass by
    // reference, no copying!)
    MyVector(const MyVector&);(*\index{Constants!\textbf{const}}*)
    // move constructor
    MyVector(MyVector&&);
    // copy assignment
    MyVector& operator=(const MyVector&);(*\index{\texttt{operator=}}*)(*\index{Constants!\textbf{const}}*)(*\index{\texttt{operator=}}*)
    // move assignment
    MyVector& operator=(MyVector&&);(*\index{\texttt{operator=}}*)(*\index{\texttt{operator=}}*)
    // virtual destructor
    virtual ~MyVector() { if (e) delete[] e; }(*\index{Classes!virtual destructor}*)(*\index{\textbf{delete}}*)(*\index{\textbf{virtual}}*)
    // subscript operators(*\index{Classes!subscript operator}*)
    // write
    double& operator[](size_t i) { return e[i]; }(*\index{Classes!subscript operator!write}*)(*\index{\texttt{size\_t}}*)
    // read
    const double& operator[](size_t i) const { return e[i]; };(*\index{Classes!subscript operator!read}*)(*\index{Constants!\textbf{const}}*)(*\index{\texttt{size\_t}}*)
    // size
    size_t size() const { return n; }(*\index{Classes!constant member function}*)(*\index{Constants!\textbf{const}}*)(*\index{\texttt{size\_t}}*)
    // capacity
    size_t capacity() const { return m; }(*\index{Classes!constant member function}*)(*\index{Constants!\textbf{const}}*)(*\index{\texttt{size\_t}}*)
    // reserve
    void reserve(size_t);(*\index{\texttt{size\_t}}*)
    // resize
    void resize(size_t);(*\index{\texttt{size\_t}}*)
    // push back
    void push_back(double);
private:
    size_t n{0}; // size(*\index{\texttt{size\_t}}*)
    size_t m{0}; // capacity(*\index{\texttt{size\_t}}*)
    double *e{nullptr};(*\index{\textbf{nullptr}}*)
};
\end{lstlisting}
\begin{enumerate}
\item[$\Rightarrow$] Constructors definitions\\ \\ By using the \textbf{explicit} qualifier, undesired type conversions are avoided\index{Classes!constructors!type conversions}. If you give no constructor, the compiler will generate a default constructor that does nothing.
If you give at least one constructor, then the compiler will generate no other constructors\index{Classes!constructors}. Notice the use of \texttt{double()} as the default value (0.0)
when initializing the vector:\index{Default value!\texttt{double()}}
\end{enumerate}
\begin{lstlisting}
// constructor with member initialization list(*\index{Classes!constructors!vector size}*)(*\index{Classes!constructors!member initialization list}*)
MyVector::MyVector(size_t s) : n{s}, m{s}, e{new double[n]}(*\index{\textbf{new}}*)(*\index{\texttt{size\_t}}*)
{
     for (int i=0; i<n; i++) e[i] = double();
}

// constructor with initializer list parameter(*\index{Classes!constructors!initializer list parameter}*)
MyVector::MyVector(initializer_list<double> l)(*\index{\texttt{initializer\_list}}*)
{
    n = m = l.size();
    e = new double[n];(*\index{\textbf{new}}*)
    copy(l.begin(),l.end(),e);(*\index{\texttt{copy}|see{Algorithms}}*)(*\index{Algorithms!\texttt{copy}}*)(*\index{Iterators!\texttt{initializer\_list<T>}!\texttt{iterator}}*)(*\index{Iterators!\texttt{initializer\_list<T>}!\texttt{iterator}!\texttt{begin}}*)(*\index{Iterators!\texttt{initializer\_list<T>}!\texttt{iterator}!\texttt{end}}*)
}
\end{lstlisting}
\begin{enumerate}
\item[$\Rightarrow$] Copy constructor\\ \\ The argument is passed by \textbf{const} reference, i.e. no copies and no changes. If not defined, C++ automatically adds the default copy
constructor. This might not be correct if dynamic variables are used, because class members are simply copied:\index{Classes!copy constructor}
\end{enumerate}
\begin{lstlisting}
// copy constructor
MyVector::MyVector(const MyVector& v)(*\index{Constants!\textbf{const}}*)
{
    n = v.n;
    m = v.m;
    e = new double[n];(*\index{\textbf{new}}*)
    copy(v.e,v.e+v.n,e);(*\index{Algorithms!\texttt{copy}}*)
}
\end{lstlisting}
\begin{enumerate}
\item[$\Rightarrow$] Move constructor:\index{Classes!move constructor}
\end{enumerate}
\begin{lstlisting}
// move constructor
MyVector::MyVector(MyVector&& v)
{
    n = v.n;
    m = v.m;
    e = v.e;
    v.n = 0;
    v.m = 0;
    v.e = nullptr;(*\index{\textbf{nullptr}}*)
}
\end{lstlisting}
\begin{enumerate}
\item[$\Rightarrow$] Copy assignment\\ \\ If not defined, C++ automatically adds  the default assignment operator.
It might not be correct if dynamic variables are used, because class members are simply copied:\index{Classes!copy assignment}
\end{enumerate}
\begin{lstlisting}
// copy assignment
MyVector& MyVector::operator=(const MyVector& rv)(*\index{Constants!\textbf{const}}*)(*\index{\texttt{operator=}}*)
{
    // check for self assignment
    if (this == &rv)(*\index{\textbf{this}}*)
        return *this;(*\index{\textbf{this}}*)
    // check if new allocation is needed
    if (rv.n > m)
    {
        if (e) delete[] e;(*\index{\textbf{delete}}*)
        e = new double[rv.n];(*\index{\textbf{new}}*)
        m = rv.n;
    }
    // copy the values
    copy(rv.e,rv.e+rv.n,e);(*\index{Algorithms!\texttt{copy}}*)
    n = rv.n;
    return *this;(*\index{\textbf{this}}*)
}
\end{lstlisting}
\begin{enumerate}
\item[$\Rightarrow$] Move assignment:\index{Classes!move assignment}
\end{enumerate}
\begin{lstlisting}
// move assignment
MyVector& MyVector::operator=(MyVector&& rv)(*\index{\texttt{operator=}}*)
{
    delete[] e;(*\index{\textbf{delete}}*)
    n = rv.n;
    m = rv.m;
    e = rv.e;
    rv.n = 0;
    rv.m = 0;
    rv.e = nullptr;(*\index{\textbf{nullptr}}*)
    return *this;(*\index{\textbf{this}}*)
}
\end{lstlisting}
\begin{enumerate}
\item[$\Rightarrow$] Reserve (reallocation), resize and push back:\index{Classes!reallocation of resources}
\end{enumerate}
\begin{lstlisting}
// reserve
void MyVector::reserve(size_t new_m)(*\index{\texttt{size\_t}}*)
{
    if (new_m <= m)
        return;
    // new allocation
    double* p = new double[new_m];(*\index{\textbf{new}}*)
    if (e)
    {
        copy(e,e+n,p);(*\index{Algorithms!\texttt{copy}}*)
        delete[] e;(*\index{\textbf{delete}}*)
    }
    e = p;
    m = new_m;
}

// resize
void MyVector::resize(size_t new_n)(*\index{\texttt{size\_t}}*)
{
    reserve(new_n);
    for (size_t i = n; i < new_n; i++) e[i] = double();(*\index{\texttt{size\_t}}*)
    n = new_n;
}

// push back
void MyVector::push_back(double d)
{
    if (m == 0)
        reserve(8);
    else if (n == m)
        reserve(2*m);
    e[n] = d;
    ++n;
}
\end{lstlisting}
\begin{enumerate}
\item[$\Rightarrow$] Constructor invocations:\index{Classes!constructor invocations}
\end{enumerate}
\begin{lstlisting}
// constructor with size
MyVector v1(4); 

// constructor with initializer list
MyVector v2{1,2,3,4};(*\index{\texttt{initializer\_list}}*)

// copy constructor
MyVector v3{v2}; 

// pass by value with copy constructor
// (prefer const reference!)
void func(MyVector v4)
{
    // do something
}
\end{lstlisting}
\begin{enumerate}
\item[$\Rightarrow$] Move invocations\\ \\ Avoids copying when moving is sufficient, e.g. when returning an object from a function:\index{Classes!move invocations}
\end{enumerate}
\begin{lstlisting}
// example of a function returning an object
MyVector func()
{
    MyVector v4{11,12,13,14,15};
    for (size_t i=0; i<v4.size(); i++) v4[i] += i;(*\index{\texttt{size\_t}}*)
    return v4;
}

// move constructor
MyVector v5 = func(); 

// move assignment
v4 = func();       
\end{lstlisting}
\begin{enumerate}
\item[$\Rightarrow$] Compiler generated methods \\ \\ If not implemented or deleted, a compiler will generate default implementations for these methods (\emph{rule of five}):\index{Classes!compiler generated methods}\index{Classes!\emph{rule of five}}
\end{enumerate}
\begin{lstlisting}
// basic class with default copy and move semantics
// the compiler generates the default implementation
class Basic(*\index{Classes!\textbf{class}!\texttt{Basic}}*)
{
public:
    // default constructor and destructor
    Basic() = default;
    ~Basic() = default;
    // default copy constructor
    Basic(const Basic& b) = default;
    // default copy assignment
    Basic& operator=(const Basic& b) = default;
     // default move constructor
    Basic(const Basic&& b) = default;
     // default move assignment
    Basic& operator=(const Basic&& b) = default;  
}

// fancy class with deleted copy and move semantics
// the compiler generates no default implementation
class Fancy(*\index{Classes!\textbf{class}!\texttt{Fancy}}*)
{
public:
    // no constructor and destructor
    Basic() = delete;
    ~Basic() = delete;
    // no copy constructor
    Basic(const Basic& b) = delete;
    // no copy assignment
    Basic& operator=(const Basic& b) = delete;
     // no move constructor
    Basic(const Basic&& b) = delete;
     // no move assignment
    Basic& operator=(const Basic&& b) = delete;  
}
\end{lstlisting}

%
% Operator overloading
%
\section{Operator overloading}
\small
The behaviour is different if an operator is overloaded as  a class member or friend function.
\begin{enumerate}
\item[$\Rightarrow$] As class members\index{Operator overloading!as class member}
\end{enumerate}
\begin{lstlisting}
class Euro(*\index{Classes!\textbf{class}!\texttt{Euro}}*)
{
public:
    // constructor for  (*euro*)
    Euro(int euro);
    // constructor for  (*euro*) and (*cents*)
    Euro(int euro, int cents);
    Euro operator+(const Euro& amount);(*\index{\texttt{operator+}}*)(*\index{Constants!\textbf{const}}*)
private:
    int euro;
    int cents;
};
\end{lstlisting}
\begin{enumerate}
\item[$\Rightarrow$] The definition above requires a calling object:
\end{enumerate}
\begin{lstlisting}
// works, equivalent to (*Euro\{5\}.operator+( Euro\{2\} )*)
Euro result = Euro{5} + 2; 

// doesn't work, (*2*) is not a calling object of type (*Euro*) !
Euro result = 2 + Euro{5}; 
\end{lstlisting}
\begin{enumerate}
\item[$\Rightarrow$] As friend members\index{Operator overloading!as \textbf{friend} member}
\end{enumerate}
\begin{lstlisting}
#include <istream>
#include <ostream>

class Euro(*\index{Classes!\textbf{class}!\texttt{Euro}}*)
{
public:
    // constructor for  (*euro*)
    Euro(int euro);
    // constructor for (*euro*) and (*cents*)
    Euro(int euro, int cents);
    friend Euro operator+(const Euro& amount1, const Euro& amount2);(*\index{Constants!\textbf{const}}*)(*\index{\textbf{friend}}*)(*\index{\texttt{operator+}}*)
    // insertion and extraction operators
    friend ostream& operator<<(ostream& outs, const Euro& amount);(*\index{\texttt{operator<<}}*)(*\index{\textbf{friend}}*)(*\index{Constants!\textbf{const}}*)(*\index{\textbf{friend}}*)
    friend istream& operator>>(istream& ins, Euro& amount);(*\index{\texttt{operator>>}}*)(*\index{\textbf{friend}}*)
private:
    int euro;
    int cents;
};
\end{lstlisting}
\begin{enumerate}
\item[$\Rightarrow$] The definition above works for every combination because \textbf{int} arguments are converted by the constructor to \texttt{Euro} objects:
\index{Classes!constructors!type conversions}
\end{enumerate}
\begin{lstlisting}
// works, equivalent to (*Euro\{5\} + Euro\{2\}*)
Euro result = Euro{5} + 2; 

// works, equivalent to (*Euro\{2\} + Euro\{5\}*)
Euro result = 2 + Euro{5}; 
\end{lstlisting}
%
% Inheritance
%
\section{Inheritance}
\small
\begin{enumerate}
\item[$\Rightarrow$] Abstract base class (excerpt):
\index{Inheritance!abstract base class}
\end{enumerate}
\begin{lstlisting}
class Shape : public Widget(*\index{Classes!\textbf{class}!\texttt{Shape}}*)
{
public:
    // no copy constructor allowed
    Shape(const Shape&) = delete;(*\index{Constants!\textbf{const}}*)(*\index{\texttt{= delete}}*)
    // no copy assignment allowed
    Shape& operator=(const Shape&) = delete;(*\index{Constants!\textbf{const}}*)(*\index{\texttt{= delete}}*)(*\index{\texttt{operator=}}*)
    // virtual destructor
    virtual ~Shape() {}(*\index{\textbf{virtual}}*)(*\index{Classes!virtual destructor}*)
    // overrides Fl_Widget::draw()
    void draw();
    // moves a shape relative to the current
    // top-left corner (call of redraw()
    // might be needed)
    void move(int dx, int dy);
    // setter and getter methods for
    // color, style, font, transparency
    // (call of redraw() might be needed)
    void set_color(Color_type c);
    void set_color(int c);
    Color_type get_color() const { return to_color_type(new_color); }(*\index{Constants!\textbf{const}}*)
    void set_style(Style_type s, int w);
    Style_type get_style() const { return to_style_type(line_style); }(*\index{Constants!\textbf{const}}*)
    void set_font(Font_type f, int s);
protected:
    // Shape is an abstract class,
    // no instances of Shape can be created!
    Shape() : Widget() {}
    // protected virtual methods to be overridden
    // by derived classes
    virtual void draw_shape() = 0;(*\index{\textbf{virtual}}*)(*\index{\texttt{= 0}}*)
    virtual void move_shape(int dx, int dy) = 0;(*\index{\textbf{virtual}}*)(*\index{\texttt{= 0}}*)
    // protected setter methods
    virtual void set_color_shape(Color_type c) {(*\index{\textbf{virtual}}*)
        new_color = to_fl_color(c);
    }
    virtual void set_color_shape(int c) {(*\index{\textbf{virtual}}*)
        new_color = to_fl_color(c);
    }
    virtual void set_style_shape(Style_type s, int w);(*\index{\textbf{virtual}}*)
    virtual void set_font_shape(Font_type f, int s);(*\index{\textbf{virtual}}*)
    // helper methods for FLTK style and font
    void set_fl_style();
    void restore_fl_style();
    void set_fl_font();
    void restore_fl_font() { fl_font(old_font,old_fontsize); }
    // test method for checking resize calls
    void draw_outline();
private:
    Fl_Color new_color{Fl_Color()};   // color
    Fl_Color old_color{Fl_Color()};   // old color
    Fl_Font new_font{0};              // font
    Fl_Font old_font{0};              // old font
    Fl_Fontsize new_fontsize{0};      // font size
    Fl_Fontsize old_fontsize{0};      // old font size
    int line_style{0};                // line style
    int line_width{0};                // line width
};
\end{lstlisting}
\begin{enumerate}
\item[$\Rightarrow$] A base class can be a derived class itself:
\index{Inheritance!base class}
\end{enumerate}
\begin{lstlisting}
// (*Shape*) is a base class for (*Line*)
// but (*Shape*) is derived from (*Widget*)
class Line : public Shape(*\index{Classes!\textbf{class}!\texttt{Line}}*)
{
    ...
};
\end{lstlisting}
\begin{enumerate}
\item[$\Rightarrow$] Disabling copy constructors and assignment\\ \\ Notice the \texttt{= delete} syntax for disabling them. If they were allowed, slicing\index{Slicing} might occur when derived objects are copied into base objects. Usually, \textbf{sizeof}\texttt{(Shape) <= }\textbf{sizeof}\texttt{(derived classes from Shape)}. By allowing copying, some attributes are not be copied, which might lead to crashes when member functions of the derived classes are called! Note that slicing is the class object equivalent of integer truncation.
\index{Inheritance!disabling copy constructors and assignment}
\end{enumerate}
\begin{lstlisting}
class Shape : public Widget(*\index{Classes!\textbf{class}!\texttt{Shape}}*)
{
public:
    // no copy constructor allowed
    Shape(const Shape&) = delete;(*\index{Constants!\textbf{const}}*)(*\index{\texttt{= delete}}*)
    // no copy assignment allowed
    Shape& operator=(const Shape&) = delete;(*\index{Constants!\textbf{const}}*)(*\index{\texttt{= delete}}*)(*\index{\texttt{operator=}}*)
    ...
};
\end{lstlisting}
\begin{enumerate}
\item[$\Rightarrow$] Virtual destructor\\ \\ Destructors should be declared \textbf{virtual}. When derived
objects are referenced by base class pointers, the destructor of the derived class is called if it is declared \textbf{virtual}.
\index{Inheritance!virtual destructor}
\end{enumerate}
\begin{lstlisting}
class Shape : public Widget(*\index{Classes!\textbf{class}!\texttt{Shape}}*)
{
public:
    ...
    // virtual destructor
    virtual ~Shape() {}(*\index{\textbf{virtual}}*)(*\index{Classes!virtual destructor}*)
    ...
};
\end{lstlisting}
\begin{enumerate}
\item[$\Rightarrow$] Protected constructor\\ \\ By declaring the constructor as \textbf{protected}, no instances of this class can be created by a user. Since \texttt{Shape} is an abstract class, it should be used only as a base class for derived classes.
\index{Inheritance!protected constructor}
\end{enumerate}
\begin{lstlisting}
class Shape : public Widget(*\index{Classes!\textbf{class}!\texttt{Shape}}*)
{
    ...
protected:
    ...
    // Shape is an abstract class
    // no instances of Shape can be created!
    Shape() : Widget() {}
    ...
};
\end{lstlisting}
\begin{enumerate}
\item[$\Rightarrow$] Protected member functions\\ \\ By declaring member functions as protected, access is restricted only to the class itself or to derived classes, a user cannot call such functions. This is useful for helper functions which are not supposed to be called directly outside the class.
\index{Inheritance!protected member functions}
\end{enumerate}
\begin{lstlisting}
class Shape : public Widget(*\index{Classes!\textbf{class}!\texttt{Shape}}*)
{
    ...
protected:
    ...
    // helper methods for FLTK style and font
    void set_fl_style();
    void restore_fl_style();
    void set_fl_font();
    void restore_fl_font() { fl_font(old_font,old_fontsize); }
    ...
};
\end{lstlisting}
\begin{enumerate}
\item[$\Rightarrow$] Pure virtual functions\\ \\ The protected member functions \texttt{draw\_shape()} and \texttt{move\_shape()} are pure virtual functions, i.e. a derived class must provide an implementation for them. Notice the syntax \index{\texttt{= 0}} which signals that the function is a pure virtual function. When a class has function members that are declared as pure virtual functions, then the class becomes an abstract class.
\index{Inheritance!pure virtual functions}
\end{enumerate}
\begin{lstlisting}
class Shape : Widget(*\index{Classes!\textbf{class}!\texttt{Shape}}*)
{
    ...
protected:
    ...
    // protected virtual methods to be overridden by 
    // derived classes
    virtual void draw_shape() = 0;(*\index{\texttt{= 0}}*)(*\index{\textbf{virtual}}*)
    virtual void move_shape(int dx, int dy) = 0;(*\index{\texttt{= 0}}*)(*\index{\textbf{virtual}}*)
    ...
};
\end{lstlisting}
\begin{enumerate}
\item[$\Rightarrow$] Virtual functions\\ \\ The protected member functions \texttt{set\_color\_shape()} is declared as a virtual function and an implementation is provided. This means that if a derived class does not override the implementation of the base class, the derived class inherits the implementation from the base class.
\index{Inheritance!virtual functions}
\end{enumerate}
\begin{lstlisting}
class Shape : Widget(*\index{Classes!\textbf{class}!\texttt{Shape}}*)
{
    ...
protected:
    ...
    // protected setter methods
    virtual void set_color_shape(Color_type c) {(*\index{\textbf{virtual}}*)
        new_color = to_fl_color(c);
    }
    virtual void set_color_shape(int c) {(*\index{\textbf{virtual}}*)
        new_color = to_fl_color(c);
    }
    ...
};
\end{lstlisting}
\begin{enumerate}
\item[$\Rightarrow$] A derived class from the base class \texttt{Shape}:
\index{Inheritance!derived class}
\end{enumerate}
\begin{lstlisting}
class Line : public Shape(*\index{Classes!\textbf{class}!\texttt{Line}}*)
{
public:
    Line(pair<Point,Point> line) : l{line} {(*\index{\texttt{pair}}*)
        resize_shape(l.first,l.second);
    }
    virtual ~Line() {}(*\index{\textbf{virtual}}*)(*\index{Classes!virtual destructor}*)
    pair<Point,Point> get_line() const { return l; }(*\index{Constants!\textbf{const}}*)(*\index{\texttt{pair}}*)
    void set_line(pair<Point,Point> line) { l = line; }(*\index{\texttt{pair}}*)
protected:
    void draw_shape() {
        fl_line(l.first.x, l.first.y, l.second.x, l.second.y);
    }
    void move_shape(int dx, int dy) {
        l.first.x  += dx;  l.first.y += dy;
        l.second.x += dx; l.second.y += dy;
        resize_shape(l.first,l.second);
    }
private:
    pair<Point,Point> l;(*\index{\texttt{pair}}*)
};
\end{lstlisting}
\begin{enumerate}
\item[$\Rightarrow$] \texttt{Line} is derived from \texttt{Shape}, it models the relationship that a \texttt{Line} is a \texttt{Shape}
\index{Inheritance!derived class}
\end{enumerate}
\begin{lstlisting}
class Line : public Shape(*\index{Classes!\textbf{class}!\texttt{Line}}*)
{
    ...
};
\end{lstlisting}
\begin{enumerate}
\item[$\Rightarrow$] \texttt{Line} has its own getter and setter functions for accessing its own internal private representation:
\index{Classes!getter and setter functions}
\end{enumerate}
\begin{lstlisting}
class Line : public Shape(*\index{Classes!\textbf{class}!\texttt{Line}}*)
{
public:
    ...
    pair<Point,Point> get_line() const { return l; }(*\index{Constants!\textbf{const}}*)(*\index{\texttt{pair}}*)
    void set_line(pair<Point,Point> line) { l = line; }(*\index{\texttt{pair}}*)
    ...
private:
    pair<Point,Point> l;(*\index{\texttt{pair}}*)
};
\end{lstlisting}
\begin{enumerate}
\item[$\Rightarrow$] \texttt{Line} specialises the virtual functions \texttt{draw\_shape()} and \texttt{move\_shape()} according to its representation:
\index{Inheritance!function specialisation}
\end{enumerate}
\begin{lstlisting}
class Line : public Shape(*\index{Classes!\textbf{class}!\texttt{Line}}*)
{
public:
    ...
protected:
    void draw_shape() {
        fl_line(l.first.x, l.first.y, l.second.x, l.second.y);
    }
    void move_shape(int dx, int dy) {
        l.first.x  += dx;  l.first.y += dy;
        l.second.x += dx; l.second.y += dy;
        resize_shape(l.first,l.second);
    }
    ...
};
\end{lstlisting}
\begin{enumerate}
\item[$\Rightarrow$] \texttt{Circle} is also derived from \texttt{Shape}, a \texttt{Circle} is also a \texttt{Shape}.
\index{Inheritance!function specialisation}
\end{enumerate}
\begin{lstlisting}
class Circle : public Shape(*\index{Classes!\textbf{class}!\texttt{Circle}}*)
{
public:
    Circle(Point a, int rr) : c{a}, r{rr} {
        resize_shape(Point{c.x-r,c.y-r},Point{c.x+r,c.y+r});
    }
    virtual ~Circle() {}(*\index{\textbf{virtual}}*)(*\index{Classes!virtual destructor}*)
    Point get_center() const { return c; }(*\index{Constants!\textbf{const}}*)
    void set_center(Point p) {
        c = p;
        resize_shape(Point{c.x-r,c.y-r},Point{c.x+r,c.y+r});
    }
    int get_radius() const { return r; }(*\index{Constants!\textbf{const}}*)
    void set_radius(int rr) {
        r = rr;
        resize_shape(Point{c.x-r,c.y-r},Point{c.x+r,c.y+r});
    }
protected:
    void draw_shape() {
        Point tl = get_tl();
        Point br = get_br();
        fl_arc(tl.x,tl.y,br.x-tl.x,br.y-tl.y,0,360);
    }
    void move_shape(int dx,int dy) {
        c.x += dx; c.y += dy;
        resize_shape(Point{c.x-r,c.y-r},Point{c.x+r,c.y+r});
    }
private:
    Point c{}; // center
    int r{0};  // radius
};
\end{lstlisting}
%
% Polymorphism
%
\section{Polymorphism}
\small
\begin{enumerate}
\item[$\Rightarrow$] From a window perspective, it is possible to attach and draw any type of widget, and the window just needs to call the \texttt{Fl\_Widget::draw()} method:
\index{Polymorphism}
\end{enumerate}
\begin{lstlisting}
void Window::draw(Fl_Widget& w) {
    w.draw();
}
\end{lstlisting}
\begin{enumerate}
\item[$\Rightarrow$] Since \texttt{Fl\_Widget::draw()} is a pure virtual function, it is overridden by \texttt{Shape::draw()}, which in turn calls the pure virtual function \texttt{Shape::draw\_shape()}, which gets specialised in every derived class, e.g. as in \texttt{Line} or \texttt{Circle}:
\end{enumerate}
\begin{lstlisting}
void Shape::draw() {
    set_fl_style();
    if ( is_visible() ) draw_shape();
    restore_fl_style();
}

void Circle:: draw_shape() {
    Point tl = get_tl();
    Point br = get_br();
    fl_arc(tl.x,tl.y,br.x-tl.x,br.y-tl.y,0,360);
}

void Line::draw_shape() {
    fl_line(l.first.x, l.first.y, l.second.x, l.second.y);
}
\end{lstlisting}
\begin{enumerate}
\item[$\Rightarrow$] Polymorphism is allowed by the \textbf{virtual} keyword which guarantees late binding: the call \texttt{w.draw()} inside \texttt{Windows::draw()} binds to the \texttt{draw\_shape()} function of the actual object referenced, either to a \texttt{Line} or \texttt{Circle} instance.
\index{Polymorphism!late binding}
\end{enumerate}
\begin{lstlisting}
Window win;
Line diagonal { {Point{200,200},Point{250,250}} };
Circle c1{Point{100,200},50};

win.draw(diagonal); // calls (*Line::draw\_shape()*)
win.draw(c1); // calls (*Circle::draw\_shape()*)
\end{lstlisting}
%
% Exceptions
%
\section{Exceptions}
\small
\begin{enumerate}
\item[$\Rightarrow$] The value thrown by
\textbf{throw} can be of any type.
\end{enumerate}
\begin{lstlisting}
// exception class
class My_exception(*\index{Classes!\textbf{class}!\texttt{My\_exception}}*)
{
public:
    My_exception(string s);(*\index{Strings!\texttt{string}}*)
    virtual ~My_exception();(*\index{\textbf{virtual}}*)(*\index{Classes!virtual destructor}*)
    friend ostream& operator<<(ostream& os, const My_exception& e);(*\index{Constants!\textbf{const}}*)(*\index{\textbf{friend}}*)(*\index{\texttt{operator<<}}*)
protected:
    string msg;(*\index{Strings!\texttt{string}}*)
};

try
{
    throw My_exception("error");(*\index{\textbf{throw}}*)
}
catch (My_exception& e)
{
    // error stream
    cerr << e;(*\index{Input-output streams!\texttt{cerr}}*)
}
// everything else
catch (...)
{
    exit(1);
}
\end{lstlisting}
\begin{enumerate}
\item[$\Rightarrow$] The standard library defines a hierarchy of exceptions.  For example \texttt{runtime\_error} can be 
thrown when runtime errors occur:
\end{enumerate}
\begin{lstlisting}
try(*\index{Exceptions!\textbf{try-catch}}*)
{
    throw runtime_error("unexpected result!");(*\index{\textbf{throw}}*)(*\index{Exceptions!\texttt{runtime\_error}}*)
}
catch (runtime_error& e)
{
    // error stream
    cerr << "runtime error: " << e.what() << "\n";(*\index{Input-output streams!\texttt{cerr}}*)(*\index{Exceptions!\texttt{runtime\_error}!\texttt{what}}*)
    return 1;
}
\end{lstlisting}
\begin{enumerate}
\item[$\Rightarrow$] Functions throwing exceptions should list the exceptions thrown in
the exception specification list. These exceptions are not 
caught by the function itself!
\end{enumerate}
\begin{lstlisting}
// exceptions of type (*DivideByZero*) or (*OtherException*) are(*\index{Exceptions!\texttt{DivideByZero}}\index{Exceptions!\texttt{OtherException}}*)
// to be caught outside the function. All other exceptions 
// end the program if not caught inside the function.
void my_function( ) throw (DivideByZero, OtherException);(*\index{\textbf{throw}}*)

// empty exception list, i.e. all exceptions end the
// program if thrown but not caught inside the function.
void my_function( ) throw ( );(*\index{\textbf{throw}}*)

// all exceptions of all types treated normally.
void my_function( );
\end{lstlisting}
\begin{enumerate}
\item[$\Rightarrow$] \emph{Basic guarantee}\index{Exceptions!Basic guarantee}: Any part of your code should either succeed or throw an exception without leaking any resource.
\end{enumerate}
\begin{lstlisting}
// Does local cleanup avoiding leaking of resources
// if exception occurs
void my_function(void)
{
    void *p;(*\index{Pointers!\texttt{void $\ast$}}*) 
    socket *s;
    
    try(*\index{Exceptions!\textbf{try-catch}}*)
    {
        /* code that acquires some resource (memory, socket, etc.) */
        /* and might throw an exception */
    }
    catch (...)
    { 
        /* local cleanup here */
        delete p;      /* free memory */(*\index{\textbf{delete}}*)
        s.release();  /* release socket */  
        /* re-throw because function didn't succeed */
        throw()(*\index{\textbf{throw}}*)
    }    
}
\end{lstlisting}
%
% Templates
%
\section{Templates}
\small
Types are used as parameters for a function or a class.  C++ does not need the template declaration. Always put the 
template definition in the header file directly!
\begin{enumerate}
\item[$\Rightarrow$] Function template:\index{Templates!function}
\end{enumerate}
\begin{lstlisting}
// generic swap function
template<class T>(*\index{\textbf{template}|see{Templates}}*)(*\index{Templates!\textbf{template}}*)
void swap(T& a, T& b)
{
    T temp = a;
    
    a = b;
    b = temp;
}

int a, b;
char c, d;

// swaps two (*int*)s
swap(a, b);

// swaps two (*char*)s
swap(c, d);
\end{lstlisting}
\begin{enumerate}
\item[$\Rightarrow$] Class templates: extending \texttt{MyVector} with templates. Class templates are also called \emph{type generators}.
\index{Templates!class}\index{Templates!type generator}
\end{enumerate}
\begin{lstlisting}
template<class T>(*\index{Templates!\textbf{template}}*)
class MyVector(*\index{Classes!\textbf{class}!\texttt{MyVector}}*)
{
public:
    // constructor
    explicit MyVector();(*\index{\textbf{explicit}}*)
    // constructor with size
    explicit MyVector(size_t);(*\index{\textbf{explicit}}*)(*\index{\texttt{size\_t}}*)
    // constructor with initializer list
    explicit MyVector(initializer_list<T>);(*\index{\textbf{explicit}}*)(*\index{\texttt{initializer\_list}}*)
    // copy constructor (pass by
    // reference, no copying!)
    MyVector(const MyVector&);(*\index{Constants!\textbf{const}}*)
    // move constructor
    MyVector(MyVector&&);
    // copy assignment
    MyVector& operator=(const MyVector&);(*\index{Constants!\textbf{const}}*)(*\index{\texttt{operator=}}*)
    // move assignment
    MyVector& operator=(MyVector&&);(*\index{\texttt{operator=}}*)
    // virtual destructor
    virtual ~MyVector() { if (e) delete[] e; }(*\index{\textbf{delete}}*)(*\index{\textbf{virtual}}*)(*\index{Classes!virtual destructor}*)
    // subscript operators
    // write
    T& operator[](size_t i) { return e[i]; }(*\index{\texttt{size\_t}}*)
    // read
    const T& operator[](size_t i) const { return e[i]; };(*\index{Constants!\textbf{const}}*)(*\index{\texttt{size\_t}}*)
    // size
    size_t size() const { return n; }(*\index{Constants!\textbf{const}}*)(*\index{\texttt{size\_t}}*)
    // capacity
    size_t capacity() const { return m; }(*\index{Constants!\textbf{const}}*)(*\index{\texttt{size\_t}}*)
    // reserve
    void reserve(size_t);(*\index{\texttt{size\_t}}*)
    // resize
    void resize(size_t);(*\index{\texttt{size\_t}}*)
    // push back
    void push_back(T);
private:
    size_t n{0}; // size(*\index{\texttt{size\_t}}*)
    size_t m{0}; // capacity(*\index{\texttt{size\_t}}*)
    T *e{nullptr};(*\index{\textbf{nullptr}}*)
};
\end{lstlisting}
\begin{enumerate}
\item[$\Rightarrow$] Method definition with templates:\index{Templates!method definition}
\end{enumerate}
\begin{lstlisting}
// copy assignment
template<class T>(*\index{Templates!\textbf{template}}*)
MyVector<T>& MyVector<T>::operator=(const MyVector<T>& rv)(*\index{Classes!\textbf{class}!\texttt{MyVector<T>}}*)(*\index{Constants!\textbf{const}}*)(*\index{\texttt{operator=}}*)
{
    // check for self assignment
    if (this == &rv)(*\index{\textbf{this}}*)
        return *this;(*\index{\textbf{this}}*)
    // check if new allocation is needed
    if (rv.n > m)
    {
        if (e) delete[] e;(*\index{\textbf{delete}}*)
        e = new T[rv.n];(*\index{\textbf{new}}*)
        m = rv.n;
    }
    // copy the values
    copy(rv.e,rv.e+rv.n,e);(*\index{Algorithms!\texttt{copy}}*)
    n = rv.n;
    return *this;(*\index{\textbf{this}}*)
}
\end{lstlisting}
\begin{enumerate}
\item[$\Rightarrow$] \emph{Specialisation} or \emph{template instantiation}:\index{Templates!specialisation}\index{Templates!instantiation}
\end{enumerate}
\begin{lstlisting}
// (*MyVector*) of (*double*)
MyVector<double> v4{11,12,13,14,15};

// function returning a (*MyVector*) of (*double*)
MyVector<double> func()
{
    MyVector<double> v4{11,12,13,14,15};
    for (size_t i=0; i<v4.size(); i++) v4[i] += i;(*\index{\texttt{size\_t}}*)
    return v4;
}
\end{lstlisting}
\begin{enumerate}
\item[$\Rightarrow$] Integer template parameters\index{Templates!integer parameters}
\end{enumerate}
\begin{lstlisting}
// Wrapper class for an array 
template<class T,size_t N>(*\index{Templates!\textbf{template}}*)
class Wrapper(*\index{Classes!\textbf{class}!\texttt{Wrapper}}*)
{
public:
    Wrapper() { for(T& e : v)  e=T(); }
    ~Wrapper() {}
    T& operator[](int n) { return v[n]; };
    const T& operator[](int n) const { return v[n]; };(*\index{Constants!\textbf{const}}*)
    size_t size() const { return N; }(*\index{Constants!\textbf{const}}*)(*\index{\texttt{size\_t}}*)
private:
    T v[N];
};

// usage
Wrapper<double,5> array;
Wrapper<char,3> array;
\end{lstlisting}
\begin{enumerate}
\item[$\Rightarrow$] Class template parameter\index{Templates!class parameter}
\end{enumerate}
\begin{lstlisting}
// Usage of an allocator as a class template parameter
// Generalises (*\texttt{MyVector}*) for data types without a default constructor
// and with customised memory management
template<class T, class A=allocator<T>>(*\index{\texttt{allocator}}*)(*\index{Templates!\textbf{template}}*)
class MyVector(*\index{Classes!\textbf{class}!\texttt{MyVector<T,A>}}*)
{
public:
    // constructor
    explicit MyVector();(*\index{\textbf{explicit}}*)
    // constructor with size and default value
    explicit MyVector(size_t,T def = T());(*\index{\textbf{explicit}}*)(*\index{\texttt{size\_t}}*)
    // constructor with initializer list
    explicit MyVector(initializer_list<T>);(*\index{\textbf{explicit}}*)(*\index{\texttt{initializer\_list}}*)
    // copy constructor (pass by
    // reference, no copying!)
    MyVector(const MyVector&);(*\index{Constants!\textbf{const}}*)
    // move constructor
    MyVector(MyVector&&);
    // copy assignment
    MyVector& operator=(const MyVector&);(*\index{Constants!\textbf{const}}*)(*\index{\texttt{operator=}}*)
    // move assignment
    MyVector& operator=(MyVector&&);(*\index{\texttt{operator=}}*)
    // virtual destructor
    virtual ~MyVector();(*\index{\textbf{virtual}}*)(*\index{Classes!virtual destructor}*)
    // subscript operators
    // write
    T& operator[](size_t i) { return e[i]; }
    // read
    const T& operator[](size_t i) const { return e[i]; };(*\index{Constants!\textbf{const}}*)
    // size
    size_t size() const { return n; }(*\index{Constants!\textbf{const}}*)(*\index{\texttt{size\_t}}*)
    // capacity
    size_t capacity() const { return m; }(*\index{Constants!\textbf{const}}*)(*\index{\texttt{size\_t}}*)
    // reserve
    void reserve(size_t);(*\index{\texttt{size\_t}}*)
    // resize
    void resize(size_t,T def = T());(*\index{\texttt{size\_t}}*)
    // push back
    void push_back(T);
private:
    A alloc;
    size_t n{0}; // size(*\index{\texttt{size\_t}}*)
    size_t m{0}; // capacity(*\index{\texttt{size\_t}}*)
    T *e{nullptr};(*\index{\textbf{nullptr}}*)
};

// reserve
template<class T,class A>(*\index{Templates!\textbf{template}}*)
void MyVector<T,A>::reserve(size_t new_m)(*\index{\texttt{size\_t}}*)
{
    if (new_m <= m)
        return;
    // new allocation
    T* p = alloc.allocate(new_m);
    if (e)
    {
        // copy
        for (size_t i=0; i<n; ++i) alloc.construct(&p[i],e[i]);(*\index{\texttt{size\_t}}*)
        // destroy
        for (size_t i=0; i<n; ++i) alloc.destroy(&e[i]);(*\index{\texttt{size\_t}}*)
        // deallocate
        alloc.deallocate(e,m);
    }
    e = p;
    m = new_m;
}
\end{lstlisting}
\begin{enumerate}
\item[$\Rightarrow$] Template friend operator:\index{Templates!friend operator}
\end{enumerate}
\begin{lstlisting}
// Note the declaration of the template friend operator.
template<class T>(*\index{Templates!\textbf{template}}*)
class A_list(*\index{Classes!\textbf{class}!\texttt{A\_list}}*)
{
    // constructor with size of the list
    A_list(int size);
    // destructor
    ~A_list();
    // copy constructor
    A_list(A_list<T>& b);
    // assignment operator
    A_list<T>& operator=(const A_list<T>& b);(*\index{Constants!\textbf{const}}*)(*\index{\texttt{operator=}}*)
    // friend insertion operator
    template <class TT>(*\index{Templates!\textbf{template}}*)
    friend ostream& operator<<(ostream& outs, const A_list<TT>& rhs);(*\index{Constants!\textbf{const}}*)(*\index{\textbf{friend}}*)(*\index{\texttt{operator<<}}*)
private:
    T *p;
    int size;
}
\end{lstlisting}
%
% Iterators
%
\section{Iterators}
\small
\begin{enumerate}
\item[$\Rightarrow$] An iterator is a generalisation of a pointer. It is an object that identifies an element of a sequence. Different containers have
different iterators.\index{Iterators}
\end{enumerate}
\begin{lstlisting}
#include <vector>(*\index{\texttt{<vector>}}*)

vector<int> v = {1,2,3,4,5};(*\index{Containers!\texttt{vector}}*)
// mutable iterator
vector<int>::iterator e;(*\index{\texttt{iterator}|see{Iterators}}*)(*\index{Iterators!\texttt{vector}!\texttt{iterator}}*)

// bidirectional access
e = v.begin();(*\index{Iterators!bidirectional access}*)(*\index{Iterators!\texttt{vector}!\texttt{iterator}}*)
++e;
// print (*v[1]*)
cout << *e << endl;(*\index{Input-output streams!\texttt{cout}}*)
--e;
// print (*v[0]*)
cout << *e << endl;(*\index{Input-output streams!\texttt{cout}}*)

// random access(*\index{Iterators!random access}*)
e = v.begin();(*\index{Iterators!\texttt{vector}!\texttt{iterator}}*)
// print (*v[3]*)
cout << e[3] << endl;(*\index{Input-output streams!\texttt{cout}}*)

// change an element
e[3] = 9;
\end{lstlisting}
\begin{enumerate}
\item[$\Rightarrow$] Constant iterator
\end{enumerate}
\begin{lstlisting}
// constant iterator (only read)
vector<int>::const_iterator c;(*\index{\texttt{const\_iterator}|see{Iterators}}*)(*\index{Iterators!\texttt{vector}!\texttt{const\_iterator}}*)

// print out the vector content (read only)
// (*\texttt{end()}*) points one element beyond the last one!
for (c = v.begin(); c != v.end(); c++)
    cout << *c << endl;(*\index{range-for-loop}*)(*\index{Input-output streams!\texttt{cout}}*)

// not allowed
// (*c[2] = 2;*)
\end{lstlisting}
\begin{enumerate}
\item[$\Rightarrow$] Reverse iterator
\end{enumerate}
\begin{lstlisting}
// reverse iterator
vector<int>::reverse_iterator r;(*\index{\texttt{reverse\_iterator}|see{Iterators}}*)(*\index{Iterators!\texttt{vector}!\texttt{reverse\_iterator}}*)

// print out the vector content in reverse order
for (r = v.rbegin(); r != v.rend(); r++)
    cout << *r << endl;(*\index{range-for-loop}*)(*\index{Input-output streams!\texttt{cout}}*)
\end{lstlisting}
\begin{enumerate}
\item[$\Rightarrow$] Example implementation of an iterator for a custom linked list class\index{Iterators!custom implementation}
\end{enumerate}
\begin{lstlisting}
#include <iostream>(*\index{\texttt{<iostream>}}*)
#include <ostream>(*\index{\texttt{<ostream>}}*)
#include <algorithm>(*\index{\texttt{<algorithm>}}*)

// node of the linked list
template <class T>(*\index{Templates!\textbf{template}}*)
class LListNode(*\index{Classes!\textbf{class}!\texttt{LListNode<T>}}*)
{
public:
    // constructor for a new node
    LListNode(T new_data = T(), LListNode<T>* new_next = nullptr) :(*\index{\textbf{nullptr}}*)
    data(new_data), next(new_next) {};
    // friends
    friend class LList<T>;(*\index{\textbf{friend}}*)
    template <class TT>(*\index{Templates!\textbf{template}}*)
    friend ostream& operator<<(ostream& outs, const LList<TT>& rhs);(*\index{Constants!\textbf{const}}*)(*\index{\textbf{friend}}*)(*\index{\texttt{operator<<}}*)
private:
    // data element
    T data{T()};
    // next pointer
    LListNode<T>* next{nullptr};(*\index{\textbf{nullptr}}*)
};

// linked list declaration
template <class T>(*\index{Templates!\textbf{template}}*)
class LList(*\index{Classes!\textbf{class}!\texttt{LList<T>}}*)
{
public:
    // default constructor
    LList() : head(nullptr) {};(*\index{\textbf{nullptr}}*)
    // copy constructor
    LList(const LList<T>& rhs) { *this = rhs; };(*\index{\textbf{this}}*)
    // assignment operator
    LList<T>& operator=(const LList<T>& rhs);(*\index{Constants!\textbf{const}}*)(*\index{\texttt{operator=}}*)
    // virtual destructor
    virtual ~LList() { clear(); };(*\index{\textbf{virtual}}*)(*\index{Classes!virtual destructor}*)
    // clear (free) the list
    void clear();
    // get head
    LListNode<T>* get_head() const { return head; };(*\index{Constants!\textbf{const}}*)
    // get node
    LListNode<T>* get_node(int n=0) const;
    // insert a new data element at the head of the list
    void insert_at_head(T new_data);
    // insert a new data element at the end of the list
    void insert_at_end(T new_data);
    // insert a new element at a given pointed node
    void insert_at_point(LListNode<T>* ptr, T new_data);
    // remove the data element at the head of the list
    T remove_head();
    // test for empty list
    bool is_empty() const { return head == nullptr; };(*\index{Constants!\textbf{const}}*)(*\index{\textbf{nullptr}}*)
    // count of the elements stored in the list
    int size() const;
    // insertion operator
    template <class TT>(*\index{Templates!\textbf{template}}*)
    friend ostream& operator<<(ostream& outs, const LList<TT>& rhs);(*\index{Constants!\textbf{const}}*)(*\index{\textbf{friend}}*)(*\index{\texttt{operator<<}}*)
    // iterator type
    class iterator;(*\index{Iterators!\texttt{LList<T>}!\texttt{iterator}}*)
    // iterator to first element
    iterator begin() { return iterator(head); }(*\index{Iterators!\texttt{LList<T>}!\texttt{iterator}!\texttt{begin}}*)
    // iterator to one beyond last element
    iterator end() { return iterator(nullptr); }(*\index{\textbf{nullptr}}*)(*\index{Iterators!\texttt{LList<T>}!\texttt{iterator}!\texttt{end}}*)
private:
    // head pointer
    LListNode<T>* head{nullptr};(*\index{\textbf{nullptr}}*)
    // recursive copy list function
    LListNode<T>* recursive_copy(LListNode<T>* rhs);
};
\end{lstlisting}
\begin{enumerate}
\item[$\Rightarrow$] Example iterator class for the custom linked list class above\index{Iterators!custom implementation}
\end{enumerate}
\begin{lstlisting}
// iterator class for the linked list
template <class T>(*\index{Templates!\textbf{template}}*)
class LList<T>::iterator(*\index{Iterators!\texttt{LList<T>}!\texttt{iterator}}*)
{
public:
    iterator(LListNode<T>* p) : curr{p} {}
    // prefix increment, returns a reference!(*\index{Operator overloading!prefix increment ++}*)
    iterator& operator++() { curr = curr->next; return *this; }(*\index{\textbf{this}}*)(*\index{\texttt{operator++}}*)
    T& operator*() const { return curr->data; }(*\index{Constants!\textbf{const}}*)
    bool operator==(const iterator& b) const { return curr == b.curr; }(*\index{Constants!\textbf{const}}*)(*\index{\texttt{operator=}}*)
    bool operator!=(const iterator& b) const { return curr != b.curr; }(*\index{Constants!\textbf{const}}*)(*\index{\texttt{operator"!=}}*)
private:
    LListNode<T>* curr{nullptr};(*\index{\textbf{nullptr}}*)
};

// example usage
LList<int> data_list;

// inserts element into the list
data_list.insert_at_head(45);
data_list.insert_at_head(-21);
data_list.insert_at_end(127);

// prints (*data\_list = (-21) -> (45) -> (127)*)
cout << "data_list = " << data_list << endl;(*\index{Input-output streams!\texttt{cout}}*)
// prints  (*data\_list.size() = 3*)
cout << "data_list.size() = " << data_list.size() << "\n\n";(*\index{Input-output streams!\texttt{cout}}*)

// applies standard algorithms on the custom linked list
LList<int>::iterator p = find(data_list.begin(),data_list.end(),45);(*\index{\texttt{find}|see{Algorithms}}*)(*\index{Algorithms!\texttt{find}}*)(*\index{Iterators!\texttt{LList<T>}!\texttt{iterator}}*)(*\index{Iterators!\texttt{LList<T>}!\texttt{iterator}!\texttt{begin}}*)(*\index{Iterators!\texttt{LList<T>}!\texttt{iterator}!\texttt{end}}*)

// checks if the element has been found(*\index{Iterators!end of sequence convention}*)
// standard algorithms return the end of a sequence, 
// i.e. the (*end*) iterator, to indicate failure
if ( p != second_list.end() )
        cout << "found element " << *p << "\n\n";(*\index{Input-output streams!\texttt{cout}}*)
else
        cout << "cannot find element " << 45 << "\n\n";(*\index{Input-output streams!\texttt{cout}}*)

// write access
*p = 180;

// prints (*data\_list = (-21) -> (180) -> (127)*)
cout << "data_list = " << data_list << endl;(*\index{Input-output streams!\texttt{cout}}*)
\end{lstlisting}
%
% Containers
%
\section{Containers}
\small
\begin{enumerate}
\item[$\Rightarrow$] Sequential containers: \texttt{list}\index{Containers!sequential|see{\texttt{list}}}\index{Containers!sequential|see{\texttt{vector}}}
\end{enumerate}
\begin{lstlisting}
#include <list>(*\index{\texttt{<list>}}*)

list<double> data = {1.32,-2.45,5.65,-8.93,2.76};(*\index{\texttt{list}|see{Containers}}*)(*\index{Containers!\texttt{list}}*)

// adds elements
data.push_back(9.23);(*\index{Containers!\texttt{list}!\texttt{push\_back}}*)
data.push_front(-3.94);(*\index{Containers!\texttt{list}!\texttt{push\_front}}*)

// bidirectional iterator, no random access    
list<double>::iterator e;(*\index{Containers!\texttt{list}}*)(*\index{Iterators!\texttt{list}!\texttt{iterator}}*)

// (*advance*)
e = data.begin();
advance(e,2);(*\index{Algorithms!\texttt{advance}}*)

// erases element (*5.65*)
data.erase(e);(*\index{Containers!\texttt{list}!\texttt{erase}}*)

// print out the content    
for (e = data.begin(); e != data.end(); e++)
    cout << *e << endl;(*\index{Input-output streams!\texttt{cout}}*)
    
// range-for-loop
for (auto x : data)(*\index{range-for-loop}*)
    cout << x << endl;(*\index{\textbf{auto}}*)(*\index{Input-output streams!\texttt{cout}}*)
\end{lstlisting}
\begin{enumerate}
\item[$\Rightarrow$] Adapter containers: \texttt{stack}\index{Containers!adapter|see{\texttt{stack}}}
\end{enumerate}
\begin{lstlisting}
#include <stack>(*\index{\texttt{<stack>}}*)

stack<double> numbers;(*\index{\texttt{stack}|see{Containers}}*)(*\index{Containers!\texttt{stack}}*)

// push on the stack
numbers.push(5.65);(*\index{Containers!\texttt{stack}!\texttt{push}}*)
numbers.push(-3.95);
numbers.push(6.95);

// size
cout << numbers.size()(*\index{Containers!\texttt{stack}!\texttt{size}}*)(*\index{Input-output streams!\texttt{cout}}*)

// read top data element
double d = numbers.top();(*\index{Containers!\texttt{stack}!\texttt{top}}*)

// pop top element
numbers.pop();(*\index{Containers!\texttt{stack}!\texttt{pop}}*)
\end{lstlisting}
\begin{enumerate}
\item[$\Rightarrow$] Associative containers: \texttt{set}, ordered according to its keys\index{Containers!associative|see{\texttt{set}}}
\end{enumerate}
\begin{lstlisting}
#include <set>(*\index{\texttt{<set>}}*)

set<char> letters;(*\index{\texttt{set}|see{Containers}}*)(*\index{Containers!\texttt{set}}*)

// inserting elements    
letters.insert('a');(*\index{Containers!\texttt{set}!\texttt{insert}}*)
letters.insert('d');
// no duplicates!
letters.insert('d');
letters.insert('g');

// erase    
letters.erase('a');(*\index{Containers!\texttt{set}!\texttt{erase}}*)

// const iterator 
set<char>::const_iterator c;(*\index{Constants!\textbf{const}}*)(*\index{Iterators!\texttt{set}!\texttt{const\_iterator}}*)
for (c = letters.begin(); c != letters.end(); c++)
    cout << *c << endl;(*\index{Input-output streams!\texttt{cout}}*)
\end{lstlisting}
\begin{enumerate}
\item[$\Rightarrow$] Associative containers: \texttt{map}, ordered according to its key in pairs (key,value)\index{Containers!associative|see{\texttt{map}}}
\end{enumerate}
\begin{lstlisting}
#include <map>(*\index{\texttt{<map>}}*)
#include <utility>(*\index{\texttt{<utility>}}*)
#include <string>(*\index{\texttt{<string>}}*)

// initialization
map<string,int> dict = { {"one",1}, {"two",2} };(*\index{\texttt{map}|see{Containers}}*)(*\index{Containers!\texttt{map}}*)(*\index{Strings!\texttt{string}}*)
pair<string,int> three("three",3);(*\index{\texttt{pair}}*)(*\index{Strings!\texttt{string}}*)

// insertion    
dict.insert(three);(*\index{Containers!\texttt{map}!\texttt{insert}}*)
dict["four"] = 4;
dict["five"] = 5;

// (*make\_pair*)
pair<string,int> six;
six = make_pair<string,int>("six",6);(*\index{\texttt{make\_pair}}*)
dict.insert(six);

// iterator    
map<string,int>::iterator two;(*\index{Strings!\texttt{string}}*)(*\index{Iterators!\texttt{map}!\texttt{iterator}}*)

// find    
two = dict.find("two");(*\index{Containers!\texttt{map}!\texttt{find}}*)

// erase    
dict.erase(two);(*\index{Containers!\texttt{map}!\texttt{erase}}*)

// range-for-loop(*\index{range-for-loop}*)
// inside the loop (*n*) is a (*pair*)
// the key is (*n.first*) and the value is (*n.second*)
for (auto n : dict)(*\index{\textbf{auto}}*)
    cout << "(" <<  n.first << "," <<  n.second << ")" << endl;(*\index{Input-output streams!\texttt{cout}}*)
\end{lstlisting}
\begin{enumerate}
\item[$\Rightarrow$] Associative containers: \texttt{multimap}, ordered according to its key in pairs (key,value), keys can be repeated\index{Containers!associative|see{\texttt{multimap}}}
\end{enumerate}
\begin{lstlisting}
#include <map>(*\index{\texttt{<map>}}*)
#include <utility>(*\index{\texttt{<utility>}}*)
#include <string>(*\index{\texttt{<string>}}*)

multimap<string,int> mm;(*\index{\texttt{multimap}|see{Containers}}*)(*\index{Containers!\texttt{multimap}}*)
 
// insert
mm.insert(make_pair<string,int>("Mary",1));(*\index{Containers!\texttt{multimap}!\texttt{insert}}*)
mm.insert(make_pair<string,int>("Dick",6));
mm.insert(make_pair<string,int>("Mary",7));
mm.insert(make_pair<string,int>("John",1));
mm.insert(make_pair<string,int>("Mary",4));

// search for (*Mary*), returns a pair of iterators
auto pp = mm.equal_range("Mary");(*\index{Containers!\texttt{multimap}!\texttt{equal\_range}}*)

// prints out (*Mary : 1, Mary : 7, Mary : 4*)
for (auto p = pp.first; p !=pp.second; ++p)
    cout << p->first << " : " << p->second << ", ";
\end{lstlisting}
%
% Algorithms
%
\section{Algorithms}
\small
Provided by the C++ standard template library (STL)\index{STL}.
\begin{enumerate}
\item[$\Rightarrow$]  \texttt{find}
\end{enumerate}
\begin{lstlisting}
#include <algorithm>(*\index{\texttt{<algorithm>}}*)
#include <vector>(*\index{\texttt{<vector>}}*)

vector<int> v = {6,2,7,13,4,3,1};(*\index{\texttt{vector}|see{Containers}}*)(*\index{Containers!\texttt{vector}}*)
vector<int>::iterator p;(*\index{Iterators!\texttt{vector}!\texttt{iterator}}*)

// (*find*)(*\index{Algorithms!\texttt{find}}*)
// points to the first occurrence of 13 in (*v*)
p = find(v.begin(),v.end(),13);(*\index{Algorithms!\texttt{find}}*)
\end{lstlisting}
\begin{enumerate}
\item[$\Rightarrow$]  \texttt{find\_if}
\end{enumerate}
\begin{lstlisting}
bool test_greater_than_5(int x) { return x>5; }
// (*find\_if*)(*\index{Algorithms!\texttt{find\_if}}*)(*\index{\texttt{find\_if}|see{Algorithms}}*)
// general search, stops as soon as the predicate is satisfied
// points to the first occurrence of an element greater than 5 in (*v*)
p = find_if(v.begin(),v.end(),test_greater_than_5);(*\index{Algorithms!\texttt{find\_if}}*)
\end{lstlisting}
\begin{enumerate}
\item[$\Rightarrow$] Function object
\end{enumerate}
\begin{lstlisting}
// function object(*\index{function object}*)
class Test_greater(*\index{Classes!\textbf{class}!\texttt{Test\_greater}}*)
{
public:
    Test_greater(int x) : n{x} {}
    bool operator() (const int x) const { return x>n; }(*\index{Constants!\textbf{const}}*)
private:
    int n{0};
};

// (*find\_if*)(*\index{Algorithms!\texttt{find\_if}}*)
// general search, stops as soon as the predicate is satisfied
// points to the first occurrence of an element greater than 7 in (*v*)
p = find_if(v.begin(),v.end(),Test_greater(7));(*\index{Algorithms!\texttt{find\_if}}*)
\end{lstlisting}
\begin{enumerate}
\item[$\Rightarrow$] \texttt{accumulate}
\end{enumerate}
\begin{lstlisting}
#include <algorithm>(*\index{\texttt{<algorithm>}}*)
#include <vector>(*\index{\texttt{<vector>}}*)
#include <list>(*\index{\texttt{<list>}}*)
#include <set>(*\index{\texttt{<set>}}*)
#include <numeric>(*\index{\texttt{<numeric>}}*)
#include <functional>(*\index{\texttt{<functional>}}*)

// (*accumulate*)
// adds the values from a sequence
// the last parameter is the initial value
// the return type is the type of the initial value!
long res = accumulate(v.begin(),v.end(),long(0));(*\index{Algorithms!\texttt{accumulate}}*)(*\index{\texttt{accumulate}|see{accumulate}}*)

// generic accumulate performing multiplication instead of sum(*\index{Algorithms!generic \texttt{accumulate}}*)
// passes a function object (*multiplies<int>()*) defined in (*<functional>*)(*\index{\texttt{multiplies<int>()}}*)
double res = accumulate(v.begin(),v.end(),1.0,multiplies<int>());
\end{lstlisting}
\begin{enumerate}
\item[$\Rightarrow$]  Inner product
\end{enumerate}
\begin{lstlisting}
// (*inner product*)
vector<int> v1 = {-2,2,4,11,-4,3,1};(*\index{Containers!\texttt{vector}}*)
vector<int> v2 = {4,2,4,18,5,3,1,7,9,};(*\index{Containers!\texttt{vector}}*)

long res = inner_product(v1.begin(), v1.end(), v2.begin(),long(0));(*\index{Algorithms!\texttt{inner\_product}}*)(*\index{\texttt{inner\_product}|see{Algorithms}}*)
\end{lstlisting}
\begin{enumerate}
\item[$\Rightarrow$]  \texttt{copy} between different containers
\end{enumerate}
\begin{lstlisting}
list<double> data = {1.32,-2.45,5.65};
vector<int> int_data(5);

// copy data from the (*list*) of doubles to the (*vector*) of integers
copy(data.begin(),data.end(),int_data.begin());(*\index{Algorithms!\texttt{copy}!between different containers}*)
\end{lstlisting}
\begin{enumerate}
\item[$\Rightarrow$]  \texttt{copy} between a container and an output stream. A container can be initialized by the elements retrieved via a pair of iterators.
\end{enumerate}
\begin{lstlisting}
// from the input character stream (*cin*) elements are 
// read as strings and used to populate a set of (*words*)

// a (*set*) doesn't allow any duplicates and keeps
// elements ordered
set<string> words{ istream_iterator<string>{cin},
                             istream_iterator<string>{} };(*\index{Iterators!initialization with a pair of iterators}*)(*\index{\texttt{istream\_iterator}}*)
                             
// copy the words from the set to the output stream (*cout*)
// and add a new line after each word
copy(words.begin(),words.end(),ostream_iterator<string>{cout,"\n"});(*\index{Algorithms!\texttt{copy}!between container and stream}*)(*\index{\texttt{ostream\_iterator}}*)
\end{lstlisting}
\begin{enumerate}
\item[$\Rightarrow$] Merge sort
\end{enumerate}
\begin{lstlisting}
// merge sort
sort(v.begin(),v.end());(*\index{Algorithms!\texttt{sort}}*)(*\index{\texttt{sort}|see{Algorithms}}*)
\end{lstlisting}
\begin{enumerate}
\item[$\Rightarrow$] Binary search
\end{enumerate}
\begin{lstlisting}
// binary search
bool found;
found = binary_search(v.begin(), v.end(), 3);(*\index{Algorithms!\texttt{binary\_search}}*)(*\index{\texttt{binary\_search}|see{Algorithms}}*)
\end{lstlisting}
\begin{enumerate}
\item[$\Rightarrow$] \texttt{reverse}
\end{enumerate}
\begin{lstlisting}
// reverse
reverse(v.begin(),v.end());(*\index{Algorithms!\texttt{reverse}}*)(*\index{\texttt{reverse}|see{Algorithms}}*)
\end{lstlisting}
%
% Chrono
%
\section{Chrono}
\small
\begin{enumerate}
\item[$\Rightarrow$] Run-time measurement in milliseconds\index{Run-time measurement}
\end{enumerate}
\begin{lstlisting}
#include <chrono>(*\index{\texttt{<chrono>}}*)

using namespace std::chrono;

auto t1 = system_clock::now();(*\index{\texttt{system\_clock::now()}}*)

// ...

auto t2 = system_clock::now();(*\index{\texttt{system\_clock::now()}}*)

// run-time in milliseconds
cout << duration_cast<milliseconds>(t2-t1).count();(*\index{\texttt{duration\_cast<milliseconds>}}*)
\end{lstlisting}
%
% Bibliography
%
\small
\begin{thebibliography}{99}
\bibitem{savitch} Walter Savitch. \textsl{Problem Solving with C++}, 10th edition. Pearson Education, 2018
\bibitem{stroustrup} Bjarne Stroustrup. \textsl{Programming: Principles and Practice Using C++}, 2nd edition. Addison Wesley, 2015
\bibitem{lospinoso} Josh Lospinoso. \textsl{C++ Crash Course: A Fast-Paced Introduction}, 1st edition. No Starch Press, 2019
\end{thebibliography}
%
% Index
%
\printindex
\end{document}
