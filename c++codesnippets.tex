%
%    C++ Code Snippets
%    Copyright (C) 2021 Michele Iarossi - michele@mathsophy.com
%
%    This program is free software: you can redistribute it and/or modify
%    it under the terms of the GNU General Public License as published by
%    the Free Software Foundation version 3 of the License.
%
%    This program is distributed in the hope that it will be useful,
%    but WITHOUT ANY WARRANTY; without even the implied warranty of
%    MERCHANTABILITY or FITNESS FOR A PARTICULAR PURPOSE.  See the
%    GNU General Public License for more details.
%
%    You should have received a copy of the GNU General Public License
%    along with this program.  If not, see <https://www.gnu.org/licenses/>.
%

% Preamble starts here: includes further global commands and specifications
% In the following, the option 12pt is a global option passed to all
% the packages
\documentclass[10pt]{book}

% Using Courier font
\renewcommand{\ttdefault}{pcr}

% A4 paper size
\usepackage[a4paper, total={6in,9in}]{geometry}

% Change the color of the text
\usepackage{xcolor}
\definecolor{codegray}{gray}{0.95}

% Add filling dots for sections in the toc
\usepackage{tocloft}
\renewcommand{\cftsecleader}{\cftdotfill{\cftdotsep}}

% C++ code listings embedded in the document
\usepackage{listings}
\lstset{
    language = [11]C++,
    basicstyle = \ttfamily,
    keywordstyle=\bfseries,
    escapeinside={(*}{*)},          % if you want to add LaTeX within your code
    backgroundcolor = \color{codegray},
    xleftmargin = 1.2cm,
    framexleftmargin = 1em
}

% This encoding corresponds to the encoding used by TexShop for saving this file
% The option utf8 is a local option, valid only for the package inputenc
\usepackage[utf8]{inputenc}

\usepackage[T1]{fontenc}

% Layout package shows the layout of the page
% You can issue the command \layout for printing the applied layout of the pages
\usepackage{layout}

% Some length parameters are rubber lengths: they stretch or reduce as needed
% 1ex means one time the height of the letter x
% Customize page layout for more space
\setlength{\parskip}{1ex plus0.5ex minus0.2ex}

% No header, only footer
\usepackage{fancyhdr}
\fancypagestyle{plain}{
    \fancyfoot[L]{\small www.mathsophy.com}
    \fancyfoot[C]{\small \thepage}
    \fancyfoot[R]{\small GNU GPL v3.0}}
\pagestyle{fancy}
\renewcommand{\headrulewidth}{0pt}
\renewcommand{\footrulewidth}{0pt}
\fancyhead{}
\fancyfoot[L]{\small www.mathsophy.com}
\fancyfoot[C]{\small \thepage}
\fancyfoot[R]{\small GNU GPL v3.0}

%\setcounter{secnumdepth}{0} % sections are level 1

% for british hyphenation
\usepackage[british]{babel}

% for making the index
\usepackage{imakeidx}

\makeindex
\indexsetup{othercode=\fontsize{9}{10}\selectfont}

% This is the body: text mixed with commands

% Commands have the following general syntax:
% \name[optional]{mandatory}
\begin{document}
% Add 2 empty pages at the beginning
\mbox{} \thispagestyle{empty} \newpage
\mbox{} \thispagestyle{empty} \newpage
\pagenumbering{roman}

% Set title and author
\title{\emph{\textbf{C++ Code Snippets}}}
\author{Michele Iarossi\thanks{\texttt{michele@mathsophy.com}}}
\date{\small \today~-~Version 2.14~-~GNU GPL v3.0}

%\layout

\maketitle

\newpage \mbox{}

\noindent
{\scriptsize The cover image shows an amazing landscape of Mongolia\\ Copyright \copyright~2009 Michele Iarossi}

\vspace*{\fill}
\noindent
\emph{Composed in \LaTeX~by the author}
\vspace*{\fill}
{\footnotesize
\begin{verbatim}
C++ Code Snippets

Copyright (C) 2021 Michele Iarossi - michele@mathsophy.com

This book is free software: you can redistribute it and/or modify
it under the terms of the GNU General Public License as published by
the Free Software Foundation, either version 3 of the License, or
(at your option) any later version.

This book is distributed in the hope that it will be useful,
but WITHOUT ANY WARRANTY; without even the implied warranty of
MERCHANTABILITY or FITNESS FOR A PARTICULAR PURPOSE.  See the
GNU General Public License (a copy can be found in Appendix A)
for more details (in particular sections 15 - Disclaimer of Warranty
and 16 - Limitation of Liability).

You should have received a copy of the GNU General Public License
along with this book.  If not, see <https://www.gnu.org/licenses/>.
\end{verbatim}
}

\newpage

\tableofcontents

\newpage
%
% Preface
%
\chapter*{Preface}
\noindent
This book provides a collection of code snippets showing the usage of common C++ language features.
It is meant to be used as a quick reference or language refresher.
In addition, I wanted to collect and document some  particular features or constructs, like the apostrophe as a digit separator.
Chances are that you are not aware of them. Explanations are kept to a minimum and are provided in the comments where necessary. 
The \LaTeX{} source files and artifacts are publicly available at the following URL:
\begin{center}
\texttt{https://github.com/MicheleIarossi/Cpp\_code\_snippets}
\end{center}
\addcontentsline{toc}{chapter}{Preface}
%
% Basics
%
\chapter{Basics}
\pagenumbering{arabic}
\noindent
In the following code snippets, the standard I/O library and namespace are always used:
\begin{lstlisting}
#include <iostream>(*\index{\texttt{<iostream>}}*)
using namespace std;(*\index{\texttt{std}}*)(*\index{Namespaces!\textbf{using namespace} directives}*)(*\index{\textbf{namespace}|see{Namespaces}}*)
\end{lstlisting}
%
% Assert
%
\section{Assertions}
\begin{enumerate}
\item[$\Rightarrow$] The first argument of a \textbf{static\_assert} is a constant expression that must be true:
\end{enumerate}
\begin{lstlisting}
static_assert(8<=sizeof(long),"longs are too small");(*\index{Assertions!\textbf{static\_assert}}*)(*\index{\textbf{static\_assert}}*)
\end{lstlisting}
%
% Constants
%
\section{Constants}
There are two options:
\begin{enumerate}
\item[$\Rightarrow$] \textbf{constexpr} must be known at compile time:
\end{enumerate}
\begin{lstlisting}
constexpr int max = 200;(*\index{\textbf{constexpr}|see{Constants}} *)(*\index{Constants!\textbf{constexpr}}*)
constexpr int c = max + 2;(*\index{Constants!\textbf{constexpr}}*)
\end{lstlisting}
\begin{enumerate}
\item[$\Rightarrow$] \textbf{constexpr} applied to functions instructs the compiler to try to evaluate the function at compile time:\index{Constants!constant functions}
\end{enumerate}
\begin{lstlisting}
constexpr int func(int n) { return n*2+5; }(*\index{Constants!\textbf{constexpr}}*)
constexpr int c = func(122);  // (*149*)(*\index{Constants!\textbf{constexpr}}*)
\end{lstlisting}
\begin{enumerate}
\item[$\Rightarrow$] Integer literal with single quotes for readability:\index{Integer literal!single quotes}
\end{enumerate}
\begin{lstlisting}
// (*1000000*)
constexpr int k = 1'000'000;(*\index{Constants!\textbf{constexpr}}*)
\end{lstlisting}
\begin{enumerate}
\item[$\Rightarrow$] \textbf{const} variables don't change at runtime. They cannot be declared as
\textbf{constexpr} because their value is not known at compile time:
\end{enumerate}
\begin{lstlisting}
// the value of (*n*) is not known at compile time
const int m = n + 1;(*\index{\textbf{const}|see{Constants}}*)(*\index{Constants!\textbf{const}}*)
\end{lstlisting}
\begin{enumerate}
\item[$\Rightarrow$] Immutable class\index{Immutable class}\\ \\ A \textbf{const} attribute is set during construction and cannot be changed afterwards!
\end{enumerate}
\begin{lstlisting}
class ConstInt(*\index{Classes!\textbf{class}!\texttt{ConstInt}}*)(*\index{Classes!immutable}*)
{
public:
    ConstInt() : myint{0} {}
    ConstInt(int n) : myint{n} {}
    int get() const { return myint; }
private:
    const int myint;
};
\end{lstlisting}
%
% Type safety
%
\section{Type Safety}
\begin{enumerate}
\item[$\Rightarrow$] Universal and uniform initialization, also known as \emph{braced initialization},\index{Uniform initialization}\index{Braced initialization} prevents narrowing conversions from happening:
\end{enumerate}
\begin{lstlisting}
// safe conversions (*\index{Conversions!safe}*)
double x {54.21};
int a {2342};

// unsafe conversions (compile error!) (*\index{Conversions!unsafe}*)
int y {x};
char b {a};
\end{lstlisting}
%
% Type casting
%
\section{Type Casting}
The following casts are called \emph{named conversions}:\index{Type casting!named conversions}\index{Named conversions}
\begin{enumerate}
\item[$\Rightarrow$] Use \textbf{static\_cast} for normal casting\index{Type casting!\textbf{static\_cast}}, i.e. types that can be converted into each other:
\end{enumerate}
\begin{lstlisting}
// (*int 15*) to (*double 15.0*)
double num;
num = static_cast<double>(15);(*\index{\textbf{static\_cast}|see{Type casting}}*)
\end{lstlisting}
\begin{enumerate}
\item[$\Rightarrow$] Use \textbf{static\_cast} for casting a void pointer to the desired pointer type:
\end{enumerate}
\begin{lstlisting}
// (*void **) pointer can point to anything
double num;
void *p = &num;(*\index{\texttt{void $\ast$}|see{Pointers}}*)(*\index{Pointers!\texttt{void $\ast$}}*) 

// back to (*double*) type
double *pd = static_cast<double*>(p);(*\index{\textbf{static\_cast}|see{Type casting}}*)
\end{lstlisting}
\begin{enumerate}
\item[$\Rightarrow$] Use \textbf{reinterpret\_cast} for casting between unrelated pointer types\index{Type casting!\textbf{reinterpret\_cast}}:
\end{enumerate}
\begin{lstlisting}
// reinterprets a (*long*) value as a (*double*) one
long n = 53;
double *pd = reinterpret_cast<double *>(&n);(*\index{\textbf{reinterpret\_cast}|see{Type casting}}*)

// prints out (*2.61855e-322*)
cout << *pd << endl;
\end{lstlisting}
\begin{enumerate}
\item[$\Rightarrow$] Use \textbf{const\_cast} for removing the \textbf{const} attribute from a reference variable pointing to a non-const variable!\index{Type casting!\textbf{const\_cast}}
\end{enumerate}
\begin{lstlisting}
// a non-const variable
int a_variable = 23;

// a const reference
const int& ref_constant = a_variable;

// remove the const attribute
int& not_constant = const_cast<int&>(ref_constant);(*\index{\textbf{const\_cast}|see{Type casting}}*)

// change the non-constant variable
not_constant++;

// outputs (*24*) for both
cout << a_variable << endl;
cout << ref_constant << endl;
\end{lstlisting}
\begin{enumerate}
\item[$\Rightarrow$] Use user-defined type conversions\\ \\ Conversions can be implicit or require an explicit cast:\index{Type casting!user defined type conversions}
\end{enumerate}
\begin{lstlisting}
// User defined type
class MyType(*\index{Classes!\textbf{class}!\texttt{MyType}}*)
{
public:
    MyType(int y=1) : x{y} {}
    // implicit conversions
    operator int() const { return x; }(*\index{Type casting!user defined type conversions!\texttt{operator int()}}*)(*\index{\texttt{operator int()}}*)
    // requires an explicit static cast
    explicit operator double() const { return double{x}; }(*\index{Type casting!user defined type conversions!\texttt{explicit operator double()}}*)(*\index{\texttt{operator double()}}*)
private:
    int x{0};
};

MyType a{5};
MyType b{7};

// (*a*) and (*b*) are converted
// implicitely to (*int*) by (*operator int()*)
// (*c = 12*)
int c = a + b;(*\index{Type casting!user defined type conversions!implicit}*)

// (*b*) is converted to (*double*) by (*operator double()*)
// but requires explicit static cast
double =   static_cast<double>(b);(*\index{\textbf{static\_cast}|see{Type casting}}*)(*\index{Type casting!user defined type conversions!explicit}*)
\end{lstlisting}
%
% Type traits
%
\section{Type Traits}
\begin{enumerate}
\item[$\Rightarrow$] Utilities for inspecting type properties \\ \\A type trait\index{Type traits} is a template class having a single parameter. The boolean \texttt{value} member is \texttt{true} if the type verifies the property, else it is \texttt{false}:
\end{enumerate}
\begin{lstlisting}
#include <type_traits>(*\index{\texttt{<type\_traits>}}*)

// example usage
cout << is_integral<int>::value; // (*true*)(*\index{\texttt{is\_integral<T>::value}}*)
cout << is_integral<float>::value; // (*false*)
cout << is_floating_point<double>::value; // (*true*)(*\index{\texttt{is\_floating<T>::value}}*)
\end{lstlisting}
%
% Storage classes
%
\section{Storage Classes}
The storage class\index{Storage class} defines the memory type where an object is stored. The lifetime\index{Lifetime} of an object is from the time it is first initialized until it is destroyed.
\begin{enumerate}
\item[$\Rightarrow$] A simple class for objects:
\end{enumerate}
\begin{lstlisting}
class Object(*\index{Classes!\textbf{class}!\texttt{Object}}*)
{
public:
    Object(string obj_name) : name{obj_name} {
        cout << "Created object: " << obj_name << endl;
    }
    ~Object() { 
        cout << "Destroyed object: " << name << endl;
    }
private:
    string name{};
};
\end{lstlisting}
\begin{enumerate}
\item[$\Rightarrow$] Static storage with global scope, external or internal linkage\\ \\ Storage is allocated before the program starts and deallocated when the program ends:
\end{enumerate}
\begin{lstlisting}
// static storage, external linkage
Object a{"a"};(*\index{Storage class!static!external linkage}*)

// static storage, internal linkage
static Object b{"b"};(*\index{Storage class!static!internal linkage}*)
\end{lstlisting}
\begin{enumerate}
\item[$\Rightarrow$] Static storage with local scope\\ \\ Storage is allocated the first time the function is called and deallocated when the program ends:
\end{enumerate}
\begin{lstlisting}
void func(void)
{
    // static storage, local variable
    static Object c{"c"};(*\index{Storage class!static!local variable}*)
}

int main()
{
    // Object (*c*) is allocated
    func();
}
\end{lstlisting}
\begin{enumerate}
\item[$\Rightarrow$] Automatic storage with local scope\\ \\ Storage is allocated on the stack when the local scope is entered and deallocated after execution leaves the scope:
\end{enumerate}
\begin{lstlisting}
int main()
{
     // automatic storage object
    {
        Object d{"d"};(*\index{Storage class!automatic}*)
    }   
}
\end{lstlisting}
\begin{enumerate}
\item[$\Rightarrow$] Thread-local storage\\ \\ Each thread is given a separate copy of the variable which is not shared with other threads:
\end{enumerate}
\begin{lstlisting}
// if not specified implies also (*static*)
// any thread accessing (*counter*) gets its own copy
// of the variable
thread_local int counter;(*\index{Storage class!thread-local}*)(*\index{\texttt{thread\_local}}*)
\end{lstlisting}
\begin{enumerate}
\item[$\Rightarrow$] Dynamic storage\\ \\ Storage is allocated dynamically on the heap with \textbf{new} and deallocated explicitly with \textbf{delete}:
\end{enumerate}
\begin{lstlisting}
int main()
{
    // dynamic storage object
    Object* e = new Object{"e"};(*\index{Storage class!dynamic}*)(*\index{\textbf{new}}*)
    delete(e);(*\index{\textbf{delete}}*)
}
\end{lstlisting}
\begin{enumerate}
\item[$\Rightarrow$] Example of object declarations:
\end{enumerate}
\begin{lstlisting}
// static storage object, external linkage
Object a{"a"};

// static storage object, internal linkage
static Object b{"b"};

void func(void)
{
    cout<< "Start of func()" << endl;
    
    // static storage, local variable
    static Object c{"c"};
    
    cout<< "End of func()" << endl;
}

int main()
{
    cout<< "Start of main()" << endl;
    
    func();
    
    // local scope
    {
        cout<< "Start of local scope" << endl;
        
        // automatic storage object
        Object d{"d"};
        
        cout<< "End of local scope" << endl;
    }
    
    // dynamic storage object
    Object* e = new Object{"e"};
    delete(e);
    
    cout<< "End of main()" << endl;
    return 0;
}
\end{lstlisting}
\begin{enumerate}
\item[$\Rightarrow$] Example result of the order of allocation and deallocation:
\end{enumerate}
\begin{lstlisting}
// Output printed
Created object: a
Created object: b
Start of main()
Start of func()
Created object: c
End of func()
Start of local scope
Created object: d
End of local scope
Destroyed object: d
Created object: e
Destroyed object: e
End of main()
Destroyed object: c
Destroyed object: b
Destroyed object: a
Program ended with exit code: 0
\end{lstlisting}
%
% Integer types
%
\section{Integer Types}
\begin{enumerate}
\item[$\Rightarrow$] Integer types having specified widths:
\end{enumerate}
\begin{lstlisting}
#include <cstdint>(*\index{\texttt{<cstdint>}}*)

// signed integers(*\index{Integer types!signed}*)
int8_t a;(*\index{\texttt{int8\_t}}*)
int16_t b;(*\index{\texttt{int16\_t}}*)
int32_t c;(*\index{\texttt{int32\_t}}*)
int64_t d;(*\index{\texttt{int64\_t}}*)

// unsigned integers(*\index{Integer types!unsigned}*)
uint8_t a;(*\index{\texttt{uint8\_t}}*)
uint16_t b;(*\index{\texttt{uint16\_t}}*)
uint32_t c;(*\index{\texttt{uint32\_t}}*)
uint64_t d;(*\index{\texttt{uint64\_t}}*)
\end{lstlisting}
\begin{enumerate}
\item[$\Rightarrow$] Secure unsigned addition:\index{Integer types!secure unsigned addition}
\end{enumerate}
\begin{lstlisting}
uint32_t a, b, c;

if ( (UINT32_MAX - a) < b )(*\index{\texttt{UINT32\_MAX}}*)
    // Error! Unsigned integer wrapping
    cout << "Error! a+b causes wrapping!\n";
else c = a + b;
\end{lstlisting}
\begin{enumerate}
\item[$\Rightarrow$] Secure signed addition:\index{Integer types!secure signed addition}
\end{enumerate}
\begin{lstlisting}
int16_t a, b, c;

if ( ( a>0 && a>(INT16_MAX-b) ) ||
     ( a<0 && a<(INT16_MIN-b) ) )(*\index{\texttt{INT16\_MAX}}*)(*\index{\texttt{INT16\_MIN}}*)
    // Error! Signed integer overflow
    cout << "Error! a+b causes overflow!\n";
else c = a + b;
\end{lstlisting}
\begin{enumerate}
\item[$\Rightarrow$] Secure unsigned subtraction:\index{Integer types!secure unsigned subtraction}
\end{enumerate}
\begin{lstlisting}
uint32_t a, b, c;

if ( a < b )
    // Error! Unsigned integer wrapping
    cout << "Error! a-b causes wrapping!\n";
else c = a - b;
\end{lstlisting}
\begin{enumerate}
\item[$\Rightarrow$] Secure signed subtraction:\index{Integer types!secure signed subtraction}
\end{enumerate}
\begin{lstlisting}
int8_t a, b, c;

if ( ( a>0 && a>(INT8_MAX+b) ) ||
     ( a<0 && a<(INT8_MIN+b) ) )(*\index{\texttt{INT8\_MAX}}*)(*\index{\texttt{INT8\_MIN}}*)
    // Error! Signed integer overflow
    cout << "Error! a-b causes overflow!\n";
else c = a - b;
\end{lstlisting}
\begin{enumerate}
\item[$\Rightarrow$] Secure unsigned multiplication:\index{Integer types!secure unsigned multiplication}
\end{enumerate}
\begin{lstlisting}
uint32_t a, b, c;

if ( a >  UINT32_MAX/b )
    // Error! Unsigned integer wrapping
    cout << "Error! a*b causes wrapping!\n";
else c = a*b;
\end{lstlisting}
\begin{enumerate}
\item[$\Rightarrow$] Secure signed multiplication:\index{Integer types!secure signed multiplication}
\end{enumerate}
\begin{lstlisting}
int16_t a, b;

int16_t c = 0;

if ( a>0 && b>0) {
    if ( a > INT16_MAX/b )
        // Error! Signed integer overflow
        cout << "Error! a*b causes overflow!\n";
    else c = a * b;
} else if (a>0 && b<0) {
     if ( b < INT16_MIN/a )
        // Error! Signed integer overflow
        cout << "Error! a*b causes overflow!\n";
    else c = a * b;
} else if (a<0 && b>0) {
     if ( a < INT16_MIN/b )
        // Error! Signed integer overflow
        cout << "Error! a*b causes overflow!\n";
    else c = a * b;
} else if (a<0 && b<0) {
     if ( b < INT16_MAX/a )
        // Error! Signed integer overflow
        cout << "Error! a*b causes overflow!\n";
    else c = a * b;
}
\end{lstlisting}
\begin{enumerate}
\item[$\Rightarrow$] Secure unsigned division:\index{Integer types!secure unsigned division}
\end{enumerate}
\begin{lstlisting}
uint64_t a, b, c;

if ( b==0  )
    // Error! Division by zero
    cout << "Error! a/b causes division by zero!\n";
else c = a/b;
\end{lstlisting}
\begin{enumerate}
\item[$\Rightarrow$] Secure signed division:\index{Integer types!secure signed division}
\end{enumerate}
\begin{lstlisting}
uint64_t a, b, c;

if ( b==0  || 
   ( (a == INT64_MIN) && (b == -1) ) )(*\index{\texttt{INT64\_MIN}}*)
    // Error! Division by zero or signed integer overflow
    cout << "Error! a/b causes division by zero or overflow!\n";
else c = a/b;
\end{lstlisting}
%
% Limits
%
\section{Limits}
\small
\begin{enumerate}
\item[$\Rightarrow$] Use \texttt{numeric\_limits<T>}\index{\texttt{numeric\_limits<T>}} for checking against built-in type limits:
\end{enumerate}
\begin{lstlisting}
#include <limits>(*\index{\texttt{<limits>}}*)

// (*int*) type
cout << numeric_limits<int>::min(); // (*-2147483648*)
cout << numeric_limits<int>::max(); // (*2147483647*)
    
// (*double*) type
cout << numeric_limits<double>::min(); // (*2.22507e-308*)
cout << numeric_limits<double>::max(); // (*1.79769e+308*)
cout << numeric_limits<double>::lowest(); // (*-1.79769e+308*)
cout << numeric_limits<double>::epsilon(); // (*2.22045e-16*)
cout << numeric_limits<double>::round_error(); // (*0.5*)
\end{lstlisting}
%
% Operators
%
\section{Operators}
\begin{enumerate}
\item[$\Rightarrow$] Unary arithmetic operators promote their operands to \textbf{int}:\index{Operators!unary arithmetic}
\end{enumerate}
\begin{lstlisting}
// a variable
short x = 10;

// expression promotes to (*int*)!
+x;

// expression promotes to (*int*)!
-x;
\end{lstlisting}
\begin{enumerate}
\item[$\Rightarrow$] Increment and decrement operators\\ \\ The value of the resulting expression depends whether prefix or postfix is used\index{Operators!increment}:\index{Operators!decrement}
\end{enumerate}
\begin{lstlisting}
int x = 5;

// prefix increment:
// expression evaluates to 6
// (*x*) evaluates to 6
++x; (*\index{Operators!increment!prefix}*)

// postfix increment:
// expression evaluates to 6
// (*x*) evaluates to 7
x++; (*\index{Operators!increment!postfix}*)

// postfix decrement:
// expression evaluates to 7
// (*x*) evaluates to 6
x--; (*\index{Operators!decrement!postfix}*)

// prefix decrement:
// expression evaluates to 5
// (*x*) evaluates to 5
--x; (*\index{Operators!decrement!prefix}*)
\end{lstlisting}
%
% Namespaces
%
\section{Namespaces and Aliases}
\begin{enumerate}
\item[$\Rightarrow$] \textbf{using} declarations for avoiding fully qualified names:
\index{Namespaces!\textbf{using} declarations}
\end{enumerate}
\begin{lstlisting}
// use (*string*) instead of (*std::string*)
using std::string;(*\index{\texttt{std}}*)(*\index{Strings!\texttt{string}}*)(*\index{\texttt{string}|see{Strings}}*)(*\index{\textbf{using}|see{Namespaces}}*)

// use (*cin*), (*cout*) instead of (*std::cin*), (*std::cout*)
using std::cin;(*\index{Input-output streams!\texttt{cin}}*)(*\index{\texttt{std}}*)(*\index{\texttt{cin}|see{Input-output streams}}*)
using std::cout;(*\index{Input-output streams!\texttt{cout}}*)(*\index{\texttt{std}}*)(*\index{\texttt{cout}|see{Input-output streams}}*)
\end{lstlisting}
\begin{enumerate}
\item[$\Rightarrow$] \textbf{using namespace} directives for including the whole namespace:
\end{enumerate}
\begin{lstlisting}
using namespace std;(*\index{Namespaces!\textbf{using namespace} directives}*)(*\index{\texttt{std}}*)(*\index{\textbf{using}|see{Namespaces}}*)
\end{lstlisting}
\begin{enumerate}
\item[$\Rightarrow$] An \emph{alias} is a symbolic name that means exactly the same as what it refers to:
\end{enumerate}
\begin{lstlisting}
using value_type = int; // (*value\_type*) means (*int*)(*\index{Alias!{\textbf{using}}}*)(*\index{\textbf{using}|see{Alias}}*)
using pchar = char*; // (*pchar*) means (*char$\ast$*)(*\index{Alias!{\textbf{using}}}*)
\end{lstlisting}
\begin{enumerate}
\item[$\Rightarrow$] \emph{Partial application}\\ \\ Sets some number of arguments to a template:
\end{enumerate}
\begin{lstlisting}
// template with (*2*) template parameters
template<class T, class U>
class TwoObjects
{
public:
    TwoObjects() : a{}, b{} {}
    TwoObjects(T x, U y) : a{x}, b{y} {}
    T get_a() const { return a; }
    U get_b() const { return b; }
private:
    T a;
    U b;
};

// partial application which sets the first
// template parameter to (*char*)
template <class T>
using OneObject = TwoObjects<char,T>;(*\index{Alias!{\emph{partial application}}}*)

// usage
OneObject<float> one('b',6.7);
\end{lstlisting}
%
% Enumerations
%
\section{Enumerations}
\begin{enumerate}
\item[$\Rightarrow$] \textbf{enum class} defines symbolic constants in the scope of the class:\index{Enumerations!in \textbf{class} scope}
\end{enumerate}
\begin{lstlisting}
// enum definition(*\index{Enumerations!definition}*)
enum class Weekdays(*\index{\textbf{enum class}}*)
{
    mon=1, tue, wed, thu, fri
};

// usage(*\index{Enumerations!usage}*)
Weekdays day = Weekdays::tue;
\end{lstlisting}
\begin{enumerate}
\item[$\Rightarrow$] \textbf{int}s cannot be assigned to \textbf{enum class} and vice versa:
\end{enumerate}
\begin{lstlisting}
// errors!(*\index{Enumerations!prohibited conversions}*)
Weekdays day = 3;
int d = Weekdays::wed;
\end{lstlisting}
\begin{enumerate}
\item[$\Rightarrow$] A conversion function should be written which uses unchecked conversions:
\end{enumerate}
\begin{lstlisting}
// valid(*\index{Enumerations!conversion function}*)
Weekdays day = Weekdays(2);
int d = int(Weekdays::fri);
\end{lstlisting}
%
% Volatile
%
\section{Volatile}
\begin{enumerate}
\item[$\Rightarrow$] \textbf{volatile} imposes restrictions on access and caching:
\end{enumerate}
\begin{lstlisting}
#include <csignal>(*\index{\texttt{<csignal>}}*)

// requires (*volatile*) here
volatile int interrupted;(*\index{\textbf{volatile}}*)

void sigint_handler(int signum) 
{
    interrupted = 1;
}

int main(void)
{
    // install signal handler for interruption signal
    signal( SIGINT, sigint_handler);(*\index{\texttt{signal}}*)
    
    cout << "Waiting for interruption...\n";
    
    // without (*volatile*) the compiler would
    // optimize the while loop as always true
    while(!interrupted)
    {
        // wait
    }
    cout <<  "Interrupted!\n";
}
\end{lstlisting}
\begin{enumerate}
\item[] Even though reordering of accesses to the volatile variable is forbidden, the compiler might still reorder it with reference to other accesses:
\end{enumerate}
\begin{lstlisting}
volatile int done;(*\index{\textbf{volatile}}*)
char arrray[100];

void init_array()
{
    for (size_t i=0; i<100; i++)
        array[i] = 1;
        
    // the compiler might move this
    // assignment before the initialization
    // loop!
    done = 1;
}
\end{lstlisting}
%
% Functions and Lambdas
%
\chapter{Functions and Lambdas}
%
% Functions
%
\section{Functions}
\begin{enumerate}
\item[$\Rightarrow$] With default trailing arguments only in the function declaration:
\index{Functions!arguments!default}
\end{enumerate}
\begin{lstlisting}
// if (*year*) is omitted, then (*year = 2000*)
void set_birthday(int day, int month, int year=2000);
\end{lstlisting}
\begin{enumerate}
\item[$\Rightarrow$] Omitting the name of an argument if not used anymore in the function definition:
\index{Functions!arguments!omitted}
\end{enumerate}
\begin{lstlisting}
// argument (*year*) is not used anymore in the function
// definition (doesn't break legacy code!)
void set_birthday(int day, int month, int) { ...}
\end{lstlisting}
\begin{enumerate}
\item[$\Rightarrow$] With read-only, read-write and copy-by-value parameters:
\index{Functions!arguments!read-only}
\index{Functions!arguments!read-write}
\index{Functions!arguments!copy-by-value}
\index{Functions!arguments!copy-by-reference}
\end{enumerate}
\begin{lstlisting}
// (*day*) input parameter passed by const
// reference (read-only)
// (*month*) output parameter to be changed by the
// function (read-write)
// (*year*) input parameter copied-by-value
void set_birthday(const int& day, int& month, int year);(*\index{Constants!\textbf{const}}*)
\end{lstlisting}
\begin{enumerate}
\item[$\Rightarrow$] Use a function for initializing an object with a complicated initializer (we might not
know exactly when the object gets initialized):\index{Functions!object initialization}
\end{enumerate}
\begin{lstlisting}
const Object& default_value()(*\index{Constants!\textbf{const}}*)
{
  static const Object default{1,2,3};(*\index{Constants!\textbf{const}}*)
  return default;
}
\end{lstlisting}
\begin{enumerate}
\item[$\Rightarrow$] Rule of thumb for passing arguments to functions:\index{Functions!arguments!rule of thumb}
\begin{itemize}
\item Pass-by-value for small objects
\item Pointer parameter type if \textbf{nullptr} means no object given\index{\textbf{nullptr}}
\item Pass-by-const-reference for large objects that are not changed
\item Pass-by-reference for large objects that are changed (output parameters)
\item Return error conditions of the function as return values
\end{itemize}
\end{enumerate}
\begin{enumerate}
\item[$\Rightarrow$] Function pointer type definition:\index{Functions!pointer to function}
\end{enumerate}
\begin{lstlisting}
// pointer to a function returning a (*void*) and
// having parameters a pointer to a (*Fl\_Widget*) and
// a pointer to a (*void*)
typedef void ( *Callback_type )( Fl_Widget*, void* );(*\index{\textbf{typedef}}*)

// (*cb*) is a callback defined as above
Callback_type cb;
\end{lstlisting}
\begin{enumerate}
\item[$\Rightarrow$] C-style linkage:\index{Functions!C-style linkage}
\end{enumerate}
\begin{lstlisting}
// to be put in the header file to be
// shared between C and C++
#ifdef __cplusplus
extern "C" {
#endif

// legacy C-function to be shared
void legacy_function(int p);

#ifdef __cplusplus
}
#endif
\end{lstlisting}
\begin{enumerate}
\item[$\Rightarrow$] Uniform container for callable objects\index{Functions!uniform container!empty function}\\ \\ Empty function and exception:
\end{enumerate}
\begin{lstlisting}
#include <function>(*\index{\texttt{<function>}}*)

// empty function
function<void()> func;

// this causes an exception because (*func*) 
// doesn't point to anything
try {(*\index{Exceptions!\textbf{try-catch}}*)
    func();
} catch (const bad_function_call& e) {(*\index{Exceptions!\texttt{bad\_function\_call}}*)
    cout << "Exception: " << e.what() << endl;(*\index{Exceptions!\texttt{bad\_function\_call}!\texttt{what}}*)
}
\end{lstlisting}
\begin{enumerate}
\item[] Assignment and initialization:\index{Functions!uniform container!assignment and initialization}
\end{enumerate}
\begin{lstlisting}
// example function
void print_sthg() { cout << "Hello" << endl; }

// assignment and call
func = print_sthg;

// call
func();

// works also with lambdas
function<void()> func2 { 
    []() { cout << "Hello" << endl; } 
};

// call
func2();
\end{lstlisting}
\begin{enumerate}
\item[$\Rightarrow$] Structured binding:\index{Functions!structured binding}
\end{enumerate}
\begin{lstlisting}
struct Data {(*\index{\textbf{struct}}*)
    float a;
    int b;
};

// this function returns a (*struct*)
Data func(void) {
    return Data{5.4,2};
}

// assigns (*85.4*) to (*a*) and (*2*) to (*b*)
auto [a,b] = func();
\end{lstlisting}
%
% Modifiers
%
\section{Modifiers}
Modifiers declare specific aspects of functions. This information can be used by compilers to optimize the code. There are prefix and suffix modifiers.
\begin{enumerate}
\item[$\Rightarrow$]  Prefix: \textbf{static}, \textbf{inline}, \textbf{constexpr}\index{Modifiers!prefix}
\end{enumerate}
\begin{lstlisting}
// internal linkage
static void sum(void);(*\index{\textbf{static}|see{Modifiers}}*)(*\index{Modifiers!prefix!\textbf{static}}*)(*\index{\textbf{static}|see{Storage classes}}*)

// inserting the content of the function
// in the execution path directly
inline int sum(int a, int b) { return a+b; }(*\index{Modifiers!prefix!\textbf{inline}}*)

// the value of the function shall be evaluated
// at compile time if possible
constexpr int sum(int a, int b) { return a+b; }(*\index{\textbf{constexpr}|see{Modifiers}} *)(*\index{Modifiers!prefix!\textbf{constexpr}}*)
\end{lstlisting}
\begin{enumerate}
\item[$\Rightarrow$]  Suffix: \texttt{final}, \texttt{override}, \textbf{noexcept}\index{Modifiers!suffix}
\end{enumerate}
\begin{lstlisting}

class Fruit
{
public:
    // a (*final*) method cannot be overridden anymore
    virtual int price(void) final {(*\index{\texttt{final}}*)(*\index{Modifiers!suffix!\texttt{final}}*)
       // ...
    }
};

class Apple : public Fruit
{
public:
    // (*override*) explicitly tells the compiler you are
    // overriding a virtual function
    int price(void) override {} // compiler error!{(*\index{\texttt{override}}*)(*\index{Modifiers!suffix!\texttt{override}}*)
};

// a (*final*) class doesn't allow derived classes
class Fruit final{(*\index{\texttt{final}}*)(*\index{Modifiers!suffix!\texttt{final}}*)
{
public:
    virtual int price(void) final {
       // ...
    }
};

// this function doesn't throw exceptions
int sum(int a, int b) noexcept { return a+b; }(*\index{\textbf{noexcept}|see{Modifiers}}*)(*\index{Modifiers!suffix!\textbf{noexcept}}*)
\end{lstlisting}
%
% Lambda expressions
%
\section{Lambda Expressions}
A lambda expression is an unnamed function that can be used where a function is needed as an argument or object.
It is introduced by \texttt{[ ]} which are called \emph{lambda introducers}.\index{Lambda expressions!lambda introducers} 
In the following code snippets we consider a \texttt{Function} class whose constructor requires a function as an argument:
\begin{lstlisting}
// callable object returning a (*double*)
// and requiring a (*double*) parameter
typedef function<double(double)> Function_type;(*\index{\textbf{typedef}}*)

// the function class
class Function : public Shape(*\index{Classes!\textbf{class}!\texttt{Function}}*)
{
public:
    // constructor
    Function (Function_type f,pair<double,double> rx,
        double d, pair<double,double> ry,(*\index{\texttt{pair}}*)
        Point p, int lx, double ar=1);
    // virtual destructor
    virtual ~Function();
    // ...
private:
    Function_type func;                // function
    // ...
}
\end{lstlisting}
Given the class above, the following lambda expressions are possible:
\begin{enumerate}
\item[$\Rightarrow$] Without access to local variables:\index{Lambda expressions!without access to local variables}
\end{enumerate}
\begin{lstlisting}
// Instantiates a (*Function*) object where the first argument
// is an unnamed function having one (*double*) parameter (*x*)
// and returning a (*double*). The return type is inferred
Function e_gr{[](double x){return exp(x);}, {-8.0,8.0},(*\index{\texttt{[ ]}}*)
        0.001, {-8.0,8.0}, {320,240}, 400};
\end{lstlisting}
\begin{enumerate}
\item[$\Rightarrow$] With access to local variables (copy by value):\index{Lambda expressions!with access to local variables!copy by value}
\end{enumerate}
\begin{lstlisting}
// Same as above, but the variable (*n*) inside the lambda
// introducer is available for the function to be used
int n = 5;
Function ee_gr{[n](double x) { n++; return expe(x,n);},
    {-8.0,8.0}, 0.001, {-8.0,8.0}, {320,240}, 400};
// now (*n*) is still 5
\end{lstlisting}
\begin{enumerate}
\item[$\Rightarrow$] With access to local variables (copy by reference):\index{Lambda expressions!with access to local variables!copy by reference}
\end{enumerate}
\begin{lstlisting}
// Same as above, but the variable (*n*) inside the lambda
// introducer is available for the function to be used 
// and modified
int n = 5;
Function ee_gr{[&n](double x){n++; return expe(x,n);},
    {-8.0,8.0}, 0.001, {-8.0,8.0}, {320,240}, 400};
// now (*n*) is 6!
\end{lstlisting}
\begin{enumerate}
\item[$\Rightarrow$] With access to all local variables (default copy by value):\index{Lambda expressions!with access to local variables!default copy by value}
\end{enumerate}
\begin{lstlisting}
// Same as above, but all local variables
// are available for the function to be used
int n = 5;
int m = 6;
Function ee_gr{[= ](double x){n++; m++; 
    return expe(x,n+m);}, {-8.0,8.0}, 0.001,
    {-8.0,8.0}, {320,240}, 400};(*\index{\texttt{[= ]}}*)
// (*n*) stays 5 and (*m*) stays 6 
 \end{lstlisting}
\begin{enumerate}
\item[$\Rightarrow$] With access to all local variables (default copy by reference):\index{Lambda expressions!with access to local variables!default copy by reference}
\end{enumerate}
\begin{lstlisting}
// Same as above, but all local variables
// are available for the function to be used and modified
int n = 5;
int m = 6;
Function ee_gr{[& ](double x){n++; m++;
    return expe(x,n+m);}, {-8.0,8.0},0.001,
    {-8.0,8.0}, {320,240}, 400};(*\index{\texttt{[\& ]}}*)
// now (*n*) is 6 and (*m*) is 7!
 \end{lstlisting}
%
% Pointers and references
%
\chapter{Pointers and References}
%
% Pointers
%
\section{Pointers}
\begin{enumerate}
\item[$\Rightarrow$] Simple object:
\end{enumerate}
\begin{lstlisting}
// simple pointer to (*double*)(*\index{Pointers!simple pointer}*)
double *d = new double{5.123};(*\index{\textbf{new}}*)

// read
double dd = *d;(*\index{Pointers!dereference operator \textbf{$\ast$}}*)

// write
*d = -11.234;

// delete the storage on the free store(*\index{Pointers!free store}*)
delete d;(*\index{\textbf{delete}}*)

// reassign: now (*d*) points to (*dd*)
d = &dd;(*\index{Pointers!address of operator \texttt{\&}}*)
\end{lstlisting}
\begin{enumerate}
\item[$\Rightarrow$] Exception handling for \texttt{new}: bad allocation\index{Pointers!exception handling for \texttt{new}!bad allocation}\\ \\Allocation failure throwing an exception:
\end{enumerate}
\begin{lstlisting}
// throws (*bad\_alloc*) if allocation fails
double *d;

try {
    d = new double{5.123}; 
} catch (const bad_alloc& e) {(*\index{\texttt{bad\_alloc}}*)
    cout << e.what() << '\n';
}
\end{lstlisting}
\begin{enumerate}
\item[] Allocation failure with \texttt{nothrow}:
\end{enumerate}
\begin{lstlisting}
// returns 0 if allocation fails(*\index{\texttt{nothrow}}*)
double *d = new(nothrow) double{5.123}; 

if (d) {
    // allocation successful
} else {
    // allocation failed
}
\end{lstlisting}
\begin{enumerate}
\item[$\Rightarrow$] Dynamic array:\index{Dynamic array|see{Pointers}}
\end{enumerate}
\begin{lstlisting}
// dynamic array of (*10 double*)s
double *dd = new double[10] {0,1,2,3,4,5,6,7,8,9};(*\index{Pointers!dynamic array!allocation}*)(*\index{\textbf{new}}*)(*\index{\textbf{new}}*)

// delete the storage on the free store
delete [] dd;(*\index{Pointers!dynamic array!deallocation}*)(*\index{\textbf{delete}}*)
\end{lstlisting}
\begin{enumerate}
\item[$\Rightarrow$] Dynamic matrix:\index{Dynamic bidimensional array|see{Pointers}}
\end{enumerate}
\begin{lstlisting}
// dynamic matrix of (*5 x 5 double*)s memory allocation(*\index{Pointers!dynamic matrix!allocation}*)
double **m = new double*[5];(*\index{\textbf{new}}*)(*\index{\textbf{new}}*)
for (int i=0; i<5; i++)
    m[i] = new double[5];(*\index{\textbf{new}}*)

// memory initialization    
for (int i=0; i<5; i++)
    for (int j=0; j<5; j++)
        m[i][j] = i*j;(*\index{Pointers!subscript operator\texttt{[]}}*)

// memory deallocation(*\index{Pointers!dynamic matrix!deallocation}*)
for (int i=0; i<5; i++)
    delete[] m[i];(*\index{\textbf{delete}}*)
delete[] m;(*\index{\textbf{delete}}*)
\end{lstlisting}
\begin{enumerate}
\item[$\Rightarrow$] Exception handling for \texttt{new}: invalid array length\index{Pointers!exception handling for \texttt{new}!invalid array length}
\end{enumerate}
\begin{lstlisting}
// throws (*bad\_array\_new\_length*) if allocation fails
double *d;

try {
    d = new double[-1]; // negative size 
    d = new double[1] {0,1,2,3}; // too many initializers 
    d = new double[INT_MAX]; // too large
} catch (const bad_array_new_length& e) {(*\index{\texttt{bad\_array\_new\_length}}*)
    cout << e.what() << '\n';
}
\end{lstlisting}
%
% Smart pointers
%
\section{Smart Pointers}
\begin{enumerate}
\item[$\Rightarrow$] \texttt{unique\_ptr}\\ \\
Holds exclusive ownership (cannot be copied!) of a dynamic object according to RAII, i.e. resource acquisition is initialization\index{Smart pointers!RAII}. It will 
automatically destroy the object if needed. Ownership can be transferred, i.e. move is supported but copy not. 
\end{enumerate}
\begin{lstlisting}
#include <memory>(*\index{\texttt{<memory>}}*)

// unique pointer to an (*int*) having value (*5*)
unique_ptr<int> p_int{ new int{5} };(*\index{Smart pointers!\texttt{unique\_ptr}}*)

// alternative declaration
auto p_int = make_unique<int>(5);(*\index{Smart pointers!\texttt{unique\_ptr}!\texttt{make\_unique}}*)(*\index{\texttt{make\_unique<T>}}*)

// evaluates to (*false*)
bool empty = ( p_int ) ? false : true;(*\index{\texttt{?}}*)

// empty pointer
unique_ptr<int> p_int{};

// evaluate to (*true*)
bool empty = ( p_int ) ? false : true;(*\index{\texttt{?}}*)
bool empty = ( p_int == nullptr );
bool empty = ( p_int.get() == nullptr );(*\index{Smart pointers!\texttt{get}}*)
\end{lstlisting}
\begin{enumerate}
\item[] Supports pointer semantics, i.e. \texttt{*} or \texttt{->}:
\end{enumerate}
\begin{lstlisting}
// unique pointer to an (*int*) having value (*5*)
unique_ptr<int> p_int{ new int{5} };

// prints (*5*)
cout << *p_out << endl;
\end{lstlisting}
\begin{enumerate}
\item[] Supports swapping:
\end{enumerate}
\begin{lstlisting}
unique_ptr<int> p_int{ new int{5} };
unique_ptr<int> q_int{ new int{7} };
    
p_int.swap(q_int);(*\index{Smart pointers!\texttt{swap}}*)

// prints (*7*) and (*5*)
cout << *p_int << endl;
cout << *q_int << endl;
\end{lstlisting}
\begin{enumerate}
\item[] Supports reset:
\end{enumerate}
\begin{lstlisting}
unique_ptr<int> p_int{ new int{5} };

// destroys (*p\_int*)    
p_int.reset();(*\index{Smart pointers!\texttt{reset}}*)

// evaluates to (*true*)
bool empty = ( p_int == nullptr );

unique_ptr<int> p_int{ new int{5} };

p_int.reset( new int{7} );
\end{lstlisting}
\begin{enumerate}
\item[] Supports replacement:
\end{enumerate}
\begin{lstlisting}
unique_ptr<int> p_int{ new int{5} };

// replacement
p_int.reset( new int{7} );(*\index{Smart pointers!replacement}*)

// prints (*7*)
cout << *p_int << endl;
\end{lstlisting}
\begin{enumerate}
\item[] Supports move transferability:
\end{enumerate}
\begin{lstlisting}
unique_ptr<int> p_int{ new int{5} };
unique_ptr<int> q_int{ new int{7} };

// move
p_int = move(q_int);(*\index{Smart pointers!move}*)

// prints (*7*)
cout << *p_int << endl;

// evaluates to (*true*)
bool empty = ( q_int == nullptr );
\end{lstlisting}
\begin{enumerate}
\item[$\Rightarrow$] \texttt{shared\_ptr}\\ \\
Has transferable non exclusive ownership (can be copied!) of a dynamic object according to RAII, i.e. resource acquisition is initialization\index{Smart pointers!RAII}. It will 
automatically destroy the object if needed. Ownership can be transferred, i.e. move and copy are supported. 
\end{enumerate}
\begin{lstlisting}
// shared pointer to an (*int*) having value (*5*)
shared_ptr<int> p_int{ new int{5} };(*\index{Smart pointers!\texttt{shared\_ptr}}*)

// alternative declaration
auto p_int = make_shared<int>(5);(*\index{Smart pointers!\texttt{shared\_ptr}!\texttt{make\_shared}}*)(*\index{\texttt{make\_shared<T>}}*)
\end{lstlisting}
\begin{enumerate}
\item[] Supports copying:
\end{enumerate}
\begin{lstlisting}
// shared pointer to an (*int*) having value (*5*)
shared_ptr<int> p_int{ new int{5} };(*\index{Smart pointers!\texttt{shared\_ptr}}*)

// copy
shared_ptr<int> q_int{ p_int };(*\index{Smart pointers!\texttt{shared\_ptr}!copying}*)

// assignment
q_int = p_int;(*\index{Smart pointers!\texttt{shared\_ptr}!assignment}*)
\end{lstlisting}
\begin{enumerate}
\item[] Safe deallocation even when pointers are shared:
\end{enumerate}
\begin{lstlisting}
#include <memory>
#include <map>
#include <list>

// a simple class
class Fruit {
public:
    Fruit(string s) : name{s} {
        cout << "Fruit: " << name << " created\n";
    }
    ~Fruit() {
        cout << "Fruit: " << name << " destroyed\n";
    }
    string get_name() const { return name; }
private:
    string name;
};

// allocate objects in a map
map<string,shared_ptr<Fruit>> fruits;
try {(*\index{Exceptions!\textbf{try-catch}}*)
    fruits["banana"] = make_shared<Fruit>("banana");(*\index{\texttt{make\_shared<T>}}*)
    fruits["apple"]  = make_shared<Fruit>("apple");(*\index{\texttt{make\_shared<T>}}*)
    fruits["kiwi"]   = make_shared<Fruit>("kiwi");(*\index{\texttt{make\_shared<T>}}*)
    fruits["mango"]  = make_shared<Fruit>("mango");(*\index{\texttt{make\_shared<T>}}*)
    fruits["ananas"] = make_shared<Fruit>("ananas");(*\index{\texttt{make\_shared<T>}}*)
} catch (bad_alloc& e) {
    cout << "Allocation failure: " << e.what() << '\n';
}
    
// share the object pointers in a new list
list< shared_ptr<Fruit> > fruit_list;
    
fruit_list.push_back( fruits["kiwi"] );(*\index{Smart pointers!\texttt{shared\_ptr}!copying}*)
fruit_list.push_back( fruits["banana"] );(*\index{Smart pointers!\texttt{shared\_ptr}!copying}*)
    
// print out items
cout << "\n\nItems on the fruit list:\n";
for (auto item : fruit_list)
    cout << " ->" << item->get_name() << '\n';
cout << "\n\n";

// allocated objects are destroyed safely even though some
// of them are shared
\end{lstlisting}
\begin{enumerate}
\item[$\Rightarrow$] \texttt{weak\_ptr}\\ \\
Has no ownership of the object pointed to. Tracks an existing object, allows conversion to a shared pointer only if the tracked object still exists. They are movable and copyable. 
\end{enumerate}
\begin{lstlisting}
// shared pointer to an (*int*) having value (*5*)
shared_ptr<int> p_int{ new int{5} };(*\index{Smart pointers!\texttt{shared\_ptr}}*)

// tracks the shared object
weak_ptr<int> wp_int { p_int };

// lock returns a shared pointer owning the object now
auto sh_ptr = wp_int.lock();(*\index{Smart pointers!\texttt{weak\_ptr}!\texttt{lock}}*)
\end{lstlisting}
\begin{enumerate}
\item[] If the tracked object doesn't exist, lock returns an empty pointer:
\end{enumerate}
\begin{lstlisting}
// creates an empty weak pointer
weak_ptr<int> wp_int {};

{
    // shared pointer to an (*int*) having value (*5*)
    shared_ptr<int> p_int{ new int{6} };(*\index{Smart pointers!\texttt{shared\_ptr}}*)

    // tracks the shared object
    wp_int = p_int;
    
    // (*p\_int*) expires here
}

// attempt to own an expired object returns a null pointer
auto sh_ptr = wp_int.lock();(*\index{Smart pointers!\texttt{weak\_ptr}!\texttt{lock}}*)

// evaluates to (*true*)
bool empty = ( sh_int == nullptr );
\end{lstlisting}
\begin{enumerate}
\item[$\Rightarrow$] Optional deleter parameter:
\end{enumerate}
\begin{lstlisting}
// deleter function object for a pointer to an (*int*)
auto int_deleter = [](int *n) { delete n; };

// unique pointer with deleter
unique_ptr<int, decltype(int_deleter)> p_int{ new int{5},
    int_deleter };(*\index{Smart pointers!deleter}*)(*\index{\textbf{decltype}}*)
\end{lstlisting}
%
% References
%
\section{References}
\begin{enumerate}
\item[$\Rightarrow$] A variable reference must be initialized with a variable being referred to:
\end{enumerate}
\begin{lstlisting}
// an integer amount
int amount = 12;

// reference to (*amount*)
int& ref_amount = amount;(*\index{Reference}*)

// outputs (*12*)
cout << "amount = " << amount << endl;
cout << "ref_amount = " << ref_amount << endl;
\end{lstlisting}
\begin{enumerate}
\item[$\Rightarrow$] A variable reference cannot be made referring to another variable at runtime:
\end{enumerate}
\begin{lstlisting}
// a new integer amount
int new_amount = 24;

// (*ref\_amount*) still refers to (*amount*)
ref_amount = new_amount;

// outputs (*24*)
cout << "amount = " << amount << endl;
cout << "ref_amount = " << ref_amount << endl;
\end{lstlisting}
\begin{enumerate}
\item[$\Rightarrow$] A variable reference can be used as input and output parameter of a function:
\end{enumerate}
\begin{lstlisting}
// function definition taking a reference
void do_something(int& in_out_var)
{
    in_out_var *= 2;
}

int a_variable = 24;

do_something(a_variable);

// outputs (*48*)
cout << "a_variable = " << a_variable << endl;
\end{lstlisting}
%
% Elementary data structures
%
\chapter{Elementary Data Structures}
%
% Arrays
%
\section{Arrays}
\begin{enumerate}
\item[$\Rightarrow$] Declaration and initialization:
\end{enumerate}
\begin{lstlisting}
// array of length (*4*) initialized to all zeros
int array[4]{};(*\index{Arrays!declaration and initialization}*)

// array of length (*4*) initialized to (*2,4,6,8*)
int array[]{2, 4, 6, 8};(*\index{Arrays!declaration and initialization}*)
\end{lstlisting}
\begin{enumerate}
\item[$\Rightarrow$] Length of an array using \textbf{sizeof}:
\end{enumerate}
\begin{lstlisting}
// array of length (*4*) initialized to (*2,4,6,8*)
int array[]{2, 4, 6, 8};

// works because (*array*) is not a pointer
// returns (*4*)
size_t array_size = sizeof(array) / sizeof(int);(*\index{\textbf{sizeof}}*)(*\index{\texttt{size\_t}}*)(*\index{Arrays!length calculation using \textbf{sizeof}}*)

// doesn't work because an array as a parameter is a pointer!
void func(int array[]) {
    // evaluates to (*1*)!
    size_t array_size = sizeof(array) / sizeof(int);
}
\end{lstlisting}
\begin{enumerate}
\item[$\Rightarrow$] Range-based \textbf{for} statement:
\index{range-for-loop}
\end{enumerate}
\begin{lstlisting}
// changes the values and outputs (*3579*)
int array[]{2, 4, 6, 8};(*\index{Arrays!declaration and initialization}*)

for (int& x : array)
  x++;(*\index{Arrays!modifying elements of an array}*)
\end{lstlisting}
\begin{enumerate}
\item[$\Rightarrow$] \textbf{auto} lets the compiler use the type of the elements in the container because it knows the type already:\index{range-for-loop}
\end{enumerate}
\begin{lstlisting}
for (auto x : arr)(*\index{\textbf{auto}}*)(*\index{range-for-loop}*)
  cout << x;(*\index{Arrays!printing elements of an array}*)
\end{lstlisting}
\begin{enumerate}
\item[$\Rightarrow$] Structured binding:\index{Arrays!structured binding}
\end{enumerate}
\begin{lstlisting}
int v[3] = {1,2,3};

// assigns (*1*) to (*a*), (*2*) to (*b*), (*3*) to (*c*), 
auto [a,b,c] = v;
\end{lstlisting}
%
% Vectors
%
\section{Vectors}
Vectors\index{Vectors} as supported by the C++ standard library. Vector elements are guaranteed to be stored sequentially in memory.
\begin{enumerate}
\item[$\Rightarrow$] Constructor with initialization list:
\end{enumerate}
\begin{lstlisting}
#include <vector>(*\index{\texttt{<vector>}}*)

// vector with base type (*int*)
vector<int> v = {2, 4, 6, 8};(*\index{Vectors!initialized with initializer list}*)(*\index{Containers!\texttt{vector}}*)
\end{lstlisting}
\begin{enumerate}
\item[$\Rightarrow$] Fill constructors\\ \\ Note that since \texttt{vector} has an initializer list constructor, you need to call the fill constructor with parentheses.
\end{enumerate}
\begin{lstlisting}
// vector with 10 elements all initialized to 0
vector<int> v(10);(*\index{Vectors!initialized with all elements to 0}*)(*\index{Containers!\texttt{vector}}*)

// vector with 10 elements all initialized to 2
vector<int> v(10,2);(*\index{Vectors!fill constructor}*)(*\index{Containers!\texttt{vector}}*)
\end{lstlisting}
\begin{enumerate}
\item[$\Rightarrow$] Access:
\end{enumerate}
\begin{lstlisting}
// unchecked access to the (*i*)th element
cout << v[i];(*\index{Vectors!access to an element!unchecked}*)

// checked access to the (*i*)th element
cout << v.at(i);(*\index{Vectors!access to an element!checked}*)(*\index{Containers!\texttt{vector}!\texttt{at}}*)

// access to the front element
cout << v.front();(*\index{Vectors!access to the front element}*)(*\index{Containers!\texttt{vector}!\texttt{front}}*)

// access to the last element
cout << v.back();(*\index{Vectors!access to the last element}*)(*\index{Containers!\texttt{vector}!\texttt{back}}*)

// pointer to the first element of the vector
cout << *v.data << endl;(*\index{Vectors!pointer to the first element}*)(*\index{Containers!\texttt{vector}!\texttt{data}}*)
\end{lstlisting}
\begin{enumerate}
\item[$\Rightarrow$] Add:
\end{enumerate}
\begin{lstlisting}
// add an element
v.push_back(10);(*\index{Vectors!add to the back}*)(*\index{Containers!\texttt{vector}!\texttt{push\_back}}*)
\end{lstlisting}
\begin{enumerate}
\item[$\Rightarrow$] Insertion:
\end{enumerate}
\begin{lstlisting}
// iterator pointing at the beginning of the vector
auto iter = v.begin();

// now pointing to the second element
++iter;

// inserting element (*10*) before the second element
v.insert(iter,10);(*\index{Vectors!element insertion}*)(*\index{Containers!\texttt{vector}!\texttt{insert}}*)
\end{lstlisting}
\begin{enumerate}
\item[$\Rightarrow$] Resize:
\end{enumerate}
\begin{lstlisting}
// resize to 20 elements
// new elements are initialized to 0
v.resize(20);(*\index{Vectors!resize}*)(*\index{Containers!\texttt{vector}!\texttt{resize}}*)
\end{lstlisting}
\begin{enumerate}
\item[$\Rightarrow$] Loop over:
\end{enumerate}
\begin{lstlisting}
// range-for-loop(*\index{range-for-loop}*)
for (auto x : v)(*\index{Vectors!loop over elements}*)(*\index{\textbf{auto}}*)
  cout << x << endl;(*\index{Input-output streams!\texttt{cout}}*)

// (*auto*) gives to (*x*) the same type of the element on the
// right hand side of the assignment, in this case
// a (*vector::iterator*)
for (auto x = v.begin(); x<v.end(); x++)(*\index{Vectors!loop over elements}*)(*\index{\textbf{auto}}*)(*\index{Iterators!\texttt{vector}!\texttt{iterator}}*)
  cout << *x << endl;(*\index{Input-output streams!\texttt{cout}}*)
\end{lstlisting}
\begin{enumerate}
\item[$\Rightarrow$] Size and capacity:
\end{enumerate}
\begin{lstlisting}
// size
cout << v.size();(*\index{Vectors!size}*)(*\index{Containers!\texttt{vector}!\texttt{size}}*)

// capacity: number of elements currently allocated
cout << v.capacity();(*\index{Vectors!total capacity}*)(*\index{Containers!\texttt{vector}!\texttt{capacity}}*)
\end{lstlisting}
\begin{enumerate}
\item[$\Rightarrow$] Reserve more capacity:
\end{enumerate}
\begin{lstlisting}
// reserve (reallocate) more capacity, e.g. at least
// 64 (*int*)s
v.reserve(64);(*\index{Vectors!reserve more capacity}*)(*\index{Containers!\texttt{vector}!\texttt{reserve}}*)
\end{lstlisting}
\begin{enumerate}
\item[$\Rightarrow$] Throws an \texttt{out\_of\_range} exception if accessed out of bounds:\index{Exceptions!\texttt{out\_of\_range}}
\end{enumerate}
\begin{lstlisting}
// out of bounds access
vector<int> v = {2, 4, 6, 8};(*\index{Containers!\texttt{vector}}*)

try(*\index{Exceptions!\textbf{try-catch}}*)
{
    cout << v.at(7);
} catch (out_of_range e)(*\index{Vectors!out of range exception}*)
{
    // access error!
}
\end{lstlisting}
\begin{enumerate}
\item[$\Rightarrow$] Clearing the vector:
\end{enumerate}
\begin{lstlisting}
// clear the vector
v.clear();(*\index{Vectors!clear all the elements}*)(*\index{Containers!\texttt{vector}!\texttt{clear}}*)

// the vector is now empty
if ( v.empty() )(*\index{Vectors!test if empty}*)(*\index{Containers!\texttt{vector}!\texttt{empty}}*)
    cout << "the vector is empty!" << endl;
else
    cout << "the vector is not empty" << endl;
\end{lstlisting}
%
% Strings
%
\section{Strings}
Strings as supported by the C++ standard library for ASCII character sets. Note that it allows almost the same methods as \texttt{vector}.
\begin{enumerate}
\item[$\Rightarrow$] Constructors:
\end{enumerate}
\begin{lstlisting}
#include <string>(*\index{\texttt{<string>}}*)

// initialization(*\index{Strings!initialization}*)
string s1 = "Hello";(*\index{Strings!\texttt{string}}*)
string s2(", world!");(*\index{Strings!\texttt{string}}*)
\end{lstlisting}
\begin{enumerate}
\item[$\Rightarrow$] Fill constructor:
\end{enumerate}
\begin{lstlisting}
string s3(5,'*'); // fill constructor (*"*****"*)(*\index{Strings!\texttt{string}}*)(*\index{Strings!fill constructor}*)
\end{lstlisting}
\begin{enumerate}
\item[$\Rightarrow$] Substring constructor from a certain position and a given optional length:
\end{enumerate}
\begin{lstlisting}
string hello("Hello, world!");
string s4(hello,0,5); // (*"Hello"*)(*\index{Strings!\texttt{string}}*)(*\index{Strings!substring constructor}*)
string s5(hello,7); // (*"world!"*)(*\index{Strings!\texttt{string}}*)(*\index{Strings!substring constructor}*)
\end{lstlisting}
\begin{enumerate}
\item[$\Rightarrow$] Constructor from C-style literal string of a certain length given an optional position:
\end{enumerate}
\begin{lstlisting}
string s6("Hello, world!",5); // (*"Hello"*)(*\index{Strings!\texttt{string}}*)(*\index{Strings!C-style constructor}*)
string s7("Hello, world!",7,6); // (*"world!"*)(*\index{Strings!\texttt{string}}*)(*\index{Strings!C-style constructor}*)
\end{lstlisting}
\begin{enumerate}
\item[$\Rightarrow$] Concatenation:
\end{enumerate}
\begin{lstlisting}
// concatenation(*\index{Strings!concatenation}*)
string hello = s1 + ", " + s2;(*\index{Strings!\texttt{string}}*)(*\index{Strings!\texttt{+}}*)
\end{lstlisting}
\begin{enumerate}
\item[$\Rightarrow$] Literal string constructor \texttt{operator""s}:
\end{enumerate}
\begin{lstlisting}
using namespace string_literals;(*\index{\texttt{string\_literals}}*)(*\index{Namespaces!\textbf{using namespace} directives}*)

// the literal string has type (*string*) (it is not a 
// C-style literal string!) and it can have zeros
string s8 = "hellohello\0\0hello"s;(*\index{Strings!\texttt{string}}*)(*\index{Strings!literal constructor}*)(*\index{Strings!\texttt{operator""""s}}*)
\end{lstlisting}
\begin{enumerate}
\item[$\Rightarrow$] Read a line:
\end{enumerate}
\begin{lstlisting}
// read a line(*\index{Strings!reading a line}*)
string line;(*\index{Strings!\texttt{string}}*)
getline(cin,line);(*\index{Input-output streams!\texttt{cin}}*)
\end{lstlisting}
\begin{enumerate}
\item[$\Rightarrow$] Access to a character:
\end{enumerate}
\begin{lstlisting}
// access to the ith character (no illegal index checking)(*\index{Strings!access to character!no illegal index checking}*)
s1[i];

// access to the ith character (with illegal index
// checking)(*\index{Strings!access to character!with illegal index checking}*)
s1.at(i);(*\index{Strings!\texttt{at}}*)
\end{lstlisting}
\begin{enumerate}
\item[$\Rightarrow$] Add:
\end{enumerate}
\begin{lstlisting}
// add a single character at the end
s1.push_back('!'); // (*"Hello!"*)(*\index{Strings!\texttt{push\_back}}*)
\end{lstlisting}
\begin{enumerate}
\item[$\Rightarrow$] Append:
\end{enumerate}
\begin{lstlisting}
// append
s1.append(s2); // (*"Hello, world!"*)(*\index{Strings!\texttt{append}}*)

// append with an open range
string s10("Hello");
string s11(", world!");
s10.append(s11.begin()+7,s11.end()); // (*"Hello!"*)(*\index{Strings!\texttt{begin}}*)(*\index{Strings!\texttt{end}}*)
\end{lstlisting}
\begin{enumerate}
\item[$\Rightarrow$] Remove:
\end{enumerate}
\begin{lstlisting}
// remove a single character from the end
s1.pop_back('c'); // (*"Hello"*)(*\index{Strings!\texttt{pop\_back}}*)
\end{lstlisting}
\begin{enumerate}
\item[$\Rightarrow$] Clear:
\end{enumerate}
\begin{lstlisting}
// clear the string
s1.clear();(*\index{Strings!\texttt{clear}}*)

// print (*string cleared*)
if ( s1.empty() )(*\index{Strings!\texttt{empty}}*)
    cout << "string cleared" << endl;
\end{lstlisting}
\begin{enumerate}
\item[$\Rightarrow$] Erase:
\end{enumerate}
\begin{lstlisting}
string s1("Hello, world!")(*\index{Strings!\texttt{string}}*)

// erase from position 5 until the end
s1.erase(s1.begin()+5,s1.end()); // (*"Hello*)(*\index{Strings!\texttt{erase}}*)

// erase from position (*2*) a length of (*3*) characters
s1.erase(2,3); // (*"He"*)(*\index{Strings!\texttt{erase}}*)
\end{lstlisting}
\begin{enumerate}
\item[$\Rightarrow$] Replace:
\end{enumerate}
\begin{lstlisting}
string s1("Hello!")(*\index{Strings!\texttt{string}}*)

// replace with range
// (*"Hello, world!*)
s1.replace(s1.begin()+5,s1.end(),", world!");(*\index{Strings!\texttt{replace}}*)

// replace with position and length
s1.replace(5,8," world!"); // (*"Hello world!*)(*\index{Strings!\texttt{replace}}*)
\end{lstlisting}
\begin{enumerate}
\item[$\Rightarrow$] Size and length:
\end{enumerate}
\begin{lstlisting}
// size and length
s1.size();(*\index{Strings!\texttt{size}}*)
s1.length();(*\index{Strings!\texttt{length}}*)
\end{lstlisting}
\begin{enumerate}
\item[$\Rightarrow$] Substring:
\end{enumerate}
\begin{lstlisting}
// substring from position (*6*) and length (*5*) characters
string substring;
substring = s1.substr(6,5); // (*"world"*)(*\index{Strings!\texttt{substr}}*)
\end{lstlisting}
\begin{enumerate}
\item[$\Rightarrow$] Find:
\end{enumerate}
\begin{lstlisting}
// find (returns (*string::npos*) if not found)
size_t pos;(*\index{\texttt{size\_t}}*)
pos = s1.find("world");(*\index{Strings!\texttt{find}}*)
if (pos == string::npos)
    cerr << "Error: String not found!\n";(*\index{Input-output streams!\texttt{cerr}}*)

// find starting from position (*5*)
pos = s1.find("l",5); // (*pos*) equals (*9*) (*\index{Strings!\texttt{find}}*)
\end{lstlisting}
\begin{enumerate}
\item[$\Rightarrow$] C-style null-terminated string of type \texttt{const char *}:
\end{enumerate}
\begin{lstlisting}
// C-string(*\index{Strings!\texttt{c\_str}}*)
s3.c_str();
\end{lstlisting}
\begin{enumerate}
\item[$\Rightarrow$] Conversions:
\end{enumerate}
\begin{lstlisting}
// from (*string*) to (*int*), (*long*), (*float*)
int    n = stoi("456");(*\index{Strings!from string to!integer}*)(*\index{\texttt{stoi}}*)
long   n = stol("1234567");(*\index{Strings!from string to!long integer}*)(*\index{\texttt{stol}}*)
double n = stod("12.345");(*\index{Strings!from string to!double}*)(*\index{\texttt{stod}}*)

// from numeric type to (*string*)(*\index{Strings!from numeric type to string}*)
string s = to_string(123.456);(*\index{\texttt{to\_string}}*)(*\index{Strings!\texttt{string}}*)
\end{lstlisting}
%
% C-Strings
%
\section{C-Strings}
\index{C-Strings}
\begin{enumerate}
\item[$\Rightarrow$] Legacy strings from C:
\end{enumerate}
\begin{lstlisting}
#include <cstring>(*\index{\texttt{<cstring>}}*)
#include <cstdlib>(*\index{\texttt{<cstdlib>}}*)

// (*C-string*) for max (*10*) characters(*\index{C-Strings!definition}*)
// long string + null char '\0'
const int SIZE = 10 + 1;(*\index{Constants!\textbf{const}}*)
char msg[SIZE] = "Hello!";
\end{lstlisting}
\begin{enumerate}
\item[$\Rightarrow$] Checking for end of string when looping:\index{C-Strings!end of string}
\end{enumerate}
\begin{lstlisting}
// correct looping over (*C-string*)s(*\index{C-Strings!correct looping}*)
int i = 0;
while ( msg[i] != '\0' && i < SIZE)
{
   // process msg[i]
}
\end{lstlisting}
\begin{enumerate}
\item[$\Rightarrow$] Safe C-string operations: \texttt{strncpy\_s}\index{C-Strings!safe copying}
\end{enumerate}
\begin{lstlisting}
// null terminated string
char src1[100] = "hello";
char dst1[6];
int r1;

// copy a string without the danger that the result will not be
// null terminated or that characters will be written past
// the end of the destination array
r1 = strncpy_s(dst1, sizeof(dst1), src1, sizeof(src1));(*\index{\texttt{strncpy\_s}}*)

if ( r1 != 0 ) 
   // error!
\end{lstlisting}
\begin{enumerate}
\item[$\Rightarrow$] Safe C-string operations: \texttt{strncmp}\index{C-Strings!safe compare}
\end{enumerate}
\begin{lstlisting}
// null terminated strings
char src1[100] = "hello"; 
char src2[100] = "hello, world";

// make sure strings are null terminated!
src1[sizeof(src1)-1] = '\0';
src2[sizeof(src2)-1] = '\0';

// safe string compare, at most (*12*) characters are compared
strncmp(src1, src2, strlen(src2));(*\index{\texttt{strncmp}}*)
\end{lstlisting}
\begin{enumerate}
\item[$\Rightarrow$] Safe C-string operations: \texttt{strncat\_s}\index{C-Strings!safe concatenation}
\end{enumerate}
\begin{lstlisting}
// null terminated strings
char s1[100] = "good";
char s5[1000] = "bye";
int r1;

// copy a string without the danger that the result will not be null
// terminated or that characters will be written past the
// end of the destination array.
r1 = strncat_s(s1, sizeof(s1), s5, sizeof(s5));(*\index{\texttt{strncat\_s}}*)

if ( r1 != 0 ) 
   // error!
\end{lstlisting}
\begin{enumerate}
\item[$\Rightarrow$] Safe C-string reading from \texttt{cin}:\index{C-Strings!correct reading from \texttt{cin}}
\end{enumerate}
\begin{lstlisting}
char buffer[10];

// limits the reading to 9 characters
// leaving space for the null terminator
cin.width(10);

// no overflow can happen here
cin >> buffer;
\end{lstlisting}

\begin{enumerate}
\item[$\Rightarrow$] Conversions:
\end{enumerate}
\begin{lstlisting}
// from (*C-string*) to (*int*), (*long*), (*float*)
int    n = atoi("567");(*\index{C-Strings!conversions!to integer}*)(*\index{\texttt{atoi}}*)
long   n = atol("1234567");(*\index{C-Strings!conversions!to long integer}*)(*\index{\texttt{atol}}*)
double n = atof("12.345");(*\index{C-Strings!conversions!to double}*)(*\index{\texttt{atof}}*)
\end{lstlisting}
%
% Input-output
%
\chapter{Input-Output}
%
% Input-output streams
%
\section{Input-Output Streams}
\index{Input-output streams}
\begin{enumerate}
\item[$\Rightarrow$] Global  input stream object \texttt{cin}, global output stream object \texttt{cout}, global error stream object \texttt{cerr}:
\index{Input-output streams!input stream|see{\texttt{cin}}}
\index{Input-output streams!output stream|see{\texttt{cout}}}
\index{Input-output streams!error stream|see{\texttt{cerr}}}
\end{enumerate}
\begin{lstlisting}
int number;
char ch;

// read a number followed by a character
// from standard input (keyboard)
// (ignores whitespaces, newlines, etc.)
cin >> number >> ch;(*\index{Input-output streams!reading from the keyboard}*)(*\index{Input-output streams!\texttt{cin}}*)

// write on standard output (display)
cout << number << " " <<  ch << endl;(*\index{Input-output streams!writing to the screen}*)(*\index{Input-output streams!\texttt{cout}}*)

// write error message on standard error (display)
cerr << "Wrong input!\n";(*\index{Input-output streams!writing error message to the screen}*)(*\index{Input-output streams!\texttt{cerr}}*)(*\index{\texttt{cerr}|see{Input-output streams}}*)
\end{lstlisting}
\begin{enumerate}
\item[$\Rightarrow$] Boolean format manipulators\\ \\ Once a  manipulator is set, it stays until another one is set, i.e. manipulators are sticky:
\index{Input-output streams!boolean format manipulators}
\end{enumerate}
\begin{lstlisting}
#include <iomanip>(*\index{\texttt{<iomanip>}}*)

// prints (*\texttt{true}*) as (*true*)
cout << boolalpha << "True: " << true << endl; (*\index{Input-output streams!\texttt{cout}!\texttt{boolalpha}}*)

// prints (*\texttt{true}*) as (*1*)
cout << noboolalpha << "True: " << true << endl;(*\index{Input-output streams!\texttt{cout}!\texttt{noboolalpha}}*)
\end{lstlisting}
\begin{enumerate}
\item[$\Rightarrow$] Integer format manipulators:
\index{Input-output streams!integer format manipulators}
\end{enumerate}
\begin{lstlisting}
#include <iomanip>(*\index{\texttt{<iomanip>}}*)

// set decimal, octal, or hexadecimal notation,
// and show the base, i.e. (*0*) for octal and (*0x*) for
// hexadecimal
cout << showbase;(*\index{Input-output streams!integer format manipulators!show the base}*)(*\index{Input-output streams!\texttt{cout}}*)(*\index{Input-output streams!\texttt{cout}!\texttt{showbase}}*)
cout << dec << 1974 << endl;(*\index{Input-output streams!integer format manipulators!decimal}*)(*\index{Input-output streams!\texttt{cout}!\texttt{dec}}*)
cout << oct << 1974 << endl;(*\index{Input-output streams!integer format manipulators!otctal}*)(*\index{Input-output streams!\texttt{cout}!\texttt{oct}}*)
cout << hex << 1974 << endl;(*\index{Input-output streams!integer format manipulators!hexadecimal}*)(*\index{Input-output streams!\texttt{cout}!\texttt{hex}}*)
cout << noshowbase;(*\index{Input-output streams!integer format manipulators!don't show the base}*)(*\index{Input-output streams!\texttt{cout}!\texttt{noshowbase}}*)

// values can be read from input in decimal, octal(*\index{Input-output streams!integer format manipulators!reading a value from the keyboard in any notation}*)
// or hexadecimal format previous unsetting
// of all the flags
cin.unsetf(ios::dec);(*\index{Input-output streams!\texttt{cin}!\texttt{unset}}*)(*\index{Input-output streams!\texttt{cin}!\texttt{unset}!\texttt{ios::dec}}*)
cin.unsetf(ios::oct);(*\index{Input-output streams!\texttt{cin}!\texttt{unset}!\texttt{ios::oct}}*)
cin.unsetf(ios::hex);(*\index{Input-output streams!\texttt{cin}!\texttt{unset}!\texttt{ios::hex}}*)

//  now (*val*) can be inserted in any format
cin >> val;(*\index{Input-output streams!\texttt{cin}}*)
\end{lstlisting}
\begin{enumerate}
\item[$\Rightarrow$] Floating point format manipulators\\ \\ Once a  manipulator is set, it stays until another one is set, i.e. manipulators are \emph{sticky}:
\index{Input-output streams!floating point format manipulators}
\end{enumerate}
\begin{lstlisting}
// set default, fixed, or scientific notation
cout << defaultfloat << 1023.984;(*\index{Input-output streams!floating point format manipulators!default float notation}*)(*\index{Input-output streams!\texttt{cout}!\texttt{defaultfloat}}*)
cout << fixed << 1023.984;(*\index{Input-output streams!floating point format manipulators!fixed notation}*)(*\index{Input-output streams!\texttt{cout}!\texttt{fixed}}*)
cout << scientific << 1023.984;(*\index{Input-output streams!floating point format manipulators!scientific notation}*)(*\index{Input-output streams!\texttt{cout}!\texttt{scientific}}*)

// set precision
cout << setprecision(2) << 1023.984;(*\index{Input-output streams!floating point format manipulators!precision}*)(*\index{Input-output streams!\texttt{cout}!\texttt{setprecision}}*)

// set character text width
cout << setw(10);(*\index{Input-output streams!floating point format manipulators! text width}*)(*\index{Input-output streams!\texttt{cout}!\texttt{setw}}*)

// set left or right alignment
cout << left  << 1023.984;(*\index{Input-output streams!floating point format manipulators!left aligned }*)(*\index{Input-output streams!\texttt{cout}!\texttt{left}}*)
cout << right << 1023.984;(*\index{Input-output streams!floating point format manipulators!right aligned}*)(*\index{Input-output streams!\texttt{cout}!\texttt{right}}*)

// always show decimal point and zeros
cout << showpoint << 0.532;(*\index{Input-output streams!floating point format manipulators!always show decimal point}*)(*\index{Input-output streams!\texttt{cout}!\texttt{showpoint}}*)

// always show plus sign
cout << showpos << 3.64;(*\index{Input-output streams!floating point format manipulators!always show plus sign}*)(*\index{Input-output streams!\texttt{cout}!\texttt{showpos}}*)
\end{lstlisting}
\begin{enumerate}
\item[$\Rightarrow$] Single characters read and write:
\index{Input-output streams!reading and writing characters}
\end{enumerate}
\begin{lstlisting}
// read any character from (*cin*) (doesn't skip spaces,
// newlines, etc.)(*\index{Input-output streams!reading and writing characters!read any character}*)
char nextChar;
cin.get(nextChar);(*\index{Input-output streams!\texttt{cin}!\texttt{get}}*)

// loop for keeping reading
// stops when end of line control character (control-d)(*\index{end of line control character}*)
// is inserted
while ( cin.get(nextChar) )(*\index{Input-output streams!\texttt{cin}!\texttt{get}}*)
{
    // process character
}

// write a character to (*cout*)(*\index{Input-output streams!reading and writing characters!write a single character}*)
cout.put(nextChar)(*\index{Input-output streams!\texttt{cout}!\texttt{put}}*)

// read a whole line of 80 chars(*\index{Input-output streams!reading and writing characters!read a whole line}*)
char line[80+1];
cin.getline(line,81);(*\index{Input-output streams!\texttt{cin}!\texttt{getline}}*)

// put back (*nextChar*) to (*cin*), (*nextChar*) will be the next(*\index{Input-output streams!reading and writing characters!putting a character back into the input stream}*)
// char read by (*cin.get()*)
cin.putback(nextChar);(*\index{Input-output streams!\texttt{cin}!\texttt{putback}}*)

// put back the last char got from (*cin.get()*) to (*cin*)(*\index{Input-output streams!reading and writing characters!putting the last character back into the input stream}*)
cin.unget();(*\index{Input-output streams!\texttt{cin}!\texttt{unget}}*)
\end{lstlisting}
\begin{enumerate}
\item[$\Rightarrow$] If the input pattern is unexpected, it is possible to set the state of \texttt{cin} to failed:
\index{Input-output streams!handling of unexpected input}
\end{enumerate}
\begin{lstlisting}
try(*\index{\textbf{try-catch}|see{Exceptions}}*)(*\index{Exceptions!\textbf{try-catch}}*)
{
    // check for unexpected input
    char ch;
    if ( cin >> ch && ch != expected_char )(*\index{Input-output streams!\texttt{cin}}*)
    {
        // put back last character read
        cin.unget();(*\index{Input-output streams!\texttt{cin}!\texttt{unget}}*)
        
        // set failed bit(*\index{Input-output streams!handling of unexpected input!setting explicitly the failure bit}*)
        cin.clear(ios_base::failbit);(*\index{Input-output streams!\texttt{cin}!\texttt{clear}}*)
    
        // throw an exception or deal with failed stream
        throw runtime_error("Unexpected input");(*\index{\textbf{throw}}*)(*\index{Exceptions!\texttt{runtime\_error}}*)
    }
}
catch (runtime_error e)
{
    cerr << "Error! " << e.what() << "\n";(*\index{Input-output streams!\texttt{cerr}}*)(*\index{Exceptions!\texttt{runtime\_error}!\texttt{what}}*)
            
    // check for failure
    if (cin.fail())(*\index{Input-output streams!\texttt{cin}!\texttt{fail}}*)
    {
        // clear failed bit(*\index{Input-output streams!handling of unexpected input!clearing the failed state of the input stream}*)
        cin.clear();(*\index{Input-output streams!\texttt{cin}!\texttt{clear}}*)
                
        // read wrong input
        string wrong_input;(*\index{Strings!\texttt{string}}*)
        cin >> wrong_input;(*\index{Input-output streams!\texttt{cin}}*)
                    
        cerr << "Got '" << wrong_input[0] << "'\n";(*\index{Input-output streams!\texttt{cerr}}*)
    }
     // End of file (eof) or corrupted state (bad)
    else return 1;
}
\end{lstlisting}
%
% Files
%
\section{Files}
\begin{enumerate}
\item[$\Rightarrow$] Accessed by means of \texttt{ifstream} (input) or \texttt{ofstream} (output) objects:
\end{enumerate}
\begin{lstlisting}
#include <fstream>(*\index{\texttt{<fstream>}}*)

// open input file(*\index{Files!opening as input}*)
ifstream in_stream {"infile.dat"};(*\index{\texttt{ifstream}}*)
// open output file(*\index{Files!opening as output}*)
ofstream out_stream {"outfile.dat"};(*\index{\texttt{ofstream}}*)
\end{lstlisting}
\begin{enumerate}
\item[$\Rightarrow$] Accessed both in input and output mode by means of \texttt{fstream}\index{\texttt{fstream}} objects (not recommended):
\end{enumerate}
\begin{lstlisting}
// open file in both input and output mode(*\index{Files!opening both as input and output}*)
fstream fs{"inoutfile.dat", ios_base::in | ios_base::out};(*\index{\texttt{fstream}}*)
\end{lstlisting}
\begin{enumerate}
\item[$\Rightarrow$] Opened explicitly (not recommended):
\end{enumerate}
\begin{lstlisting}
// input file 
ifstream in_stream;(*\index{\texttt{ifstream}}*)
// output file
ofstream out_stream;(*\index{\texttt{ofstream}}*)

// open files(*\index{Files!opening explicitly}*)
in_stream.open("infile.dat");
out_stream.open("outfile.dat");
\end{lstlisting}
\begin{enumerate}
\item[$\Rightarrow$] When checking for failure, the status flag needs to be cleared in order to continue working with the file:\index{Files!checking for failure}
\end{enumerate}
\begin{lstlisting}
// check for failure on input file
if ( !in_stream )
{
    if ( in_stream.bad() ) error("stream corrupted!");(*\index{Files!checking for failure!corrupted stream}*)
    
    if ( in_stream.eof() )(*\index{Files!checking for failure!end of file}*)
    { 
        // no more data available
    }
    
    if ( in_stream.fail() )(*\index{Files!checking for failure!format data error}*)
    {
        // some format data error, e.g. expected
        // an integer but a string was read
        // recovery is still possible
        
        // set back the state to good 
        // before attempting to read again
        in_stream.clear();(*\index{Files!checking for failure!setting back to good state}*)
        
        // read again
        string wrong_input;(*\index{Strings!\texttt{string}}*)
        in_stream >> wrong_input;
    }
}
\end{lstlisting}
\begin{enumerate}
\item[$\Rightarrow$] As for the standard input, if the input pattern is unexpected, it is possible to set the state of the file to failed and try to recover somewhere else, e.g. by throwing an exception:
\end{enumerate}
\begin{lstlisting}
try(*\index{Exceptions!\textbf{try-catch}}*)
{
    // check for unexpected input(*\index{Files!checking for unexpected input}*)
    char ch;
    if ( in_stream >> ch && ch != expected_char )
    {
        // put back last character read
        in_stream.unget();
        
        // set failed bit
        in_stream.clear(ios_base::failbit);
    
        // throw an exception or deal with failed stream
        throw runtime_error("Unexpected input");(*\index{\textbf{throw}}*)(*\index{Exceptions!\texttt{runtime\_error}}*)
    }
}
catch (runtime_error e)
{
    cerr << "Error! " << e.what() << "\n";(*\index{Input-output streams!\texttt{cerr}}*)(*\index{Exceptions!\texttt{runtime\_error}!\texttt{what}}*)
            
    // check for failure
    if (in_stream.fail())
    {
        // clear failed bit
        in_stream.clear();
                
        // read wrong input
        string wrong_input;(*\index{Strings!\texttt{string}}*)
        in_stream >> wrong_input;
                    
        cerr << "Got '" << wrong_input[0] << "'\n";(*\index{Input-output streams!\texttt{cerr}}*)
    }
    // end-of-file or bad state
    else return 1;
}
\end{lstlisting}
\begin{enumerate}
\item[$\Rightarrow$] Read and write:
\end{enumerate}
\begin{lstlisting}
// read/write data(*\index{Files!reading and writing}*)
in_stream >> data1 >> data2;
out_stream << data1 << data2;
\end{lstlisting}
\begin{enumerate}
\item[$\Rightarrow$] Read a line:
\end{enumerate}
\begin{lstlisting}
string line;(*\index{Strings!\texttt{string}}*)
getline(in_stream, line);(*\index{Files!reading a line}*)
\end{lstlisting}
\begin{enumerate}
\item[$\Rightarrow$] Ignore input (extract and discard):
\end{enumerate}
\begin{lstlisting}
// ignore up to a newline or 9999 characters(*\index{Files!ignoring input}*)
in_stream.ignore(9999,'\n');
\end{lstlisting}
\begin{enumerate}
\item[$\Rightarrow$] Move the file pointer:
\end{enumerate}
\begin{lstlisting}
// skip 5 characters when reading (seek get)
in_stream.seekg(5);(*\index{Files!moving the file pointer!reading with \texttt{seekg} (seek get)}*)
// skip 8 characters when writing (seek put)(*\index{Files!moving the file pointer!writing with \texttt{seekp} (seek put)}*)
out_stream.seekp(8);
\end{lstlisting}
\begin{enumerate}
\item[$\Rightarrow$] Checking for end of file:
\end{enumerate}
\begin{lstlisting}
// the failing read sets the EOF flag but avoids
// further processing
while ( in_stream >> next )(*\index{Files!loop for reading all the input}*)
{
    // process next
}

// check the EOF flag
if ( in_stream.eof() )
    cout << "EOF reached!" << endl;
\end{lstlisting}
\begin{enumerate}
\item[$\Rightarrow$] When a file object gets out of scope, the file is closed automatically, but explicit
close is also possible (not recommended):\index{Files!closing by going out of scope}
\end{enumerate}
\begin{lstlisting}
// explicitily close files(*\index{Files!closing explicitly}*)
in_stream.close();
out_stream.close()
\end{lstlisting}
%
% String streams
%
\section{String Streams}
A string is used as a source for an input stream or as a target for an output stream. \\
\begin{enumerate}
\item[$\Rightarrow$] Input string stream: \texttt{istringstream}
\end{enumerate}
\begin{lstlisting}
#include <sstream>(*\index{\texttt{<sstream>}}*)

// input string stream(*\index{String streams!input string stream}*)
istringstream data_stream{"1.234 -5643.32"};(*\index{\texttt{istringstream}}*)

// read numbers from data stream(*\index{String streams!read numbers from data stream}*)
double val;
while ( is >> val )
    cout << val << endl;
\end{lstlisting}
\begin{enumerate}
\item[$\Rightarrow$] Output string stream: \texttt{ostringstream}
\end{enumerate}
\begin{lstlisting}
// output string stream(*\index{String streams!output string stream}*)
ostringstream data_stream;(*\index{\texttt{ostringstream}}*)

// the same manipulators of input-output streams
// can be used(*\index{String streams!manipulators|see{Input-output streams!integer format manipulators}}*)(*\index{String streams!manipulators|see{Input-output streams!floating format manipulators}}*)
data_stream << fixed << setprecision(2) << showpos;
data_stream << 6.432 << " " << -313.2134 << "\n";

// the (*str()*) method returns the string in the stream(*\index{String streams!return the string in the stream}*)
cout << data_stream.str();
\end{lstlisting}
%
% Advanced topics
%
\chapter{Object-Oriented Programming}
%
% Classes
%
\section{Classes}
\begin{enumerate}
\item[$\Rightarrow$] Class using dynamic arrays:\index{Classes!example of a vector class}
\end{enumerate}
\begin{lstlisting}
#include <algorithm>(*\index{\texttt{<algorithm>}}*)

class MyVector(*\index{\textbf{class}|see{Classes}}*)(*\index{Classes!\textbf{class}!\texttt{MyVector}}*)
{
public:
    // explicit constructor (avoids type conversions)
    explicit MyVector();(*\index{\textbf{explicit}}*)
    // explicit constructor with size parameter
    explicit MyVector(size_t);(*\index{\textbf{explicit}}*)(*\index{\texttt{size\_t}}*)
    // explicit constructor with initializer list
    explicit MyVector(initializer_list<double>);(*\index{\texttt{initializer\_list}}*)(*\index{\textbf{explicit}}*)
    // copy constructor (pass by
    // reference, no copying!)
    MyVector(const MyVector&);(*\index{Constants!\textbf{const}}*)
    // move constructor
    MyVector(MyVector&&);
    // copy assignment
    MyVector& operator=(const MyVector&);(*\index{\texttt{operator=}}*)(*\index{Constants!\textbf{const}}*)
    // move assignment
    MyVector& operator=(MyVector&&);(*\index{\texttt{operator=}}*)
    // virtual destructor
    virtual ~MyVector() { if (e) delete[] e; }(*\index{Classes!virtual destructor}*)(*\index{\textbf{delete}}*)(*\index{\textbf{virtual}}*)
    // subscript operators(*\index{Classes!subscript operator}*)
    // write
    double& operator[](size_t i) {
        return e[i];
    }(*\index{Classes!subscript operator!write}*)(*\index{\texttt{size\_t}}*)
    // read
    const double& operator[](size_t i) const {
        return e[i];
    };(*\index{Classes!subscript operator!read}*)(*\index{Constants!\textbf{const}}*)(*\index{\texttt{size\_t}}*)
    // size
    size_t size() const { return n; }(*\index{Classes!constant member function}*)(*\index{Constants!\textbf{const}}*)(*\index{\texttt{size\_t}}*)
    // capacity
    size_t capacity() const { return m; }(*\index{Classes!constant member function}*)(*\index{Constants!\textbf{const}}*)(*\index{\texttt{size\_t}}*)
    // reserve
    void reserve(size_t);(*\index{\texttt{size\_t}}*)
    // resize
    void resize(size_t);(*\index{\texttt{size\_t}}*)
    // push back
    void push_back(double);
private:
    size_t n{0}; // size(*\index{\texttt{size\_t}}*)
    size_t m{0}; // capacity(*\index{\texttt{size\_t}}*)
    double *e{nullptr};(*\index{\textbf{nullptr}}*)
};
\end{lstlisting}
\begin{enumerate}
\item[$\Rightarrow$] Constructors definitions\\ \\ By using the \textbf{explicit} qualifier, undesired type conversions are avoided\index{Classes!constructors!type conversions}. If you give no constructor, the compiler will generate a default constructor that does nothing.
If you give at least one constructor, then the compiler will generate no other constructors\index{Classes!constructors}. Notice the use of \texttt{double()} as the default value (0.0)
when initializing the vector:\index{Default value!\texttt{double()}}
\end{enumerate}
\begin{lstlisting}
// constructor with member initialization list(*\index{Classes!constructors!vector size}*)(*\index{Classes!constructors!member initialization list}*)
MyVector::MyVector(size_t s) : n{s},
    m{s}, e{new double[n]}(*\index{\textbf{new}}*)(*\index{\texttt{size\_t}}*)
{
     for (int i=0; i<n; i++) e[i] = double();
}

// constructor with initializer list parameter(*\index{Classes!constructors!initializer list parameter}*)
MyVector::MyVector(initializer_list<double> l)(*\index{\texttt{initializer\_list}}*)
{
    n = m = l.size();
    e = new double[n];(*\index{\textbf{new}}*)
    copy(l.begin(),l.end(),e);(*\index{\texttt{copy}|see{Algorithms}}*)(*\index{Algorithms!\texttt{copy}}*)(*\index{Iterators!\texttt{initializer\_list<T>}!\texttt{iterator}}*)(*\index{Iterators!\texttt{initializer\_list<T>}!\texttt{iterator}!\texttt{begin}}*)(*\index{Iterators!\texttt{initializer\_list<T>}!\texttt{iterator}!\texttt{end}}*)
}
\end{lstlisting}
\begin{enumerate}
\item[$\Rightarrow$] Copy constructor\\ \\ The argument is passed by \textbf{const} reference, i.e. no copies and no changes. If not defined, C++ automatically adds the default copy
constructor. This might not be correct if dynamic variables are used, because class members are simply copied:\index{Classes!copy constructor}
\end{enumerate}
\begin{lstlisting}
// copy constructor
MyVector::MyVector(const MyVector& v)(*\index{Constants!\textbf{const}}*)
{
    n = v.n;
    m = v.m;
    e = new double[n];(*\index{\textbf{new}}*)
    copy(v.e,v.e+v.n,e);(*\index{Algorithms!\texttt{copy}}*)
}
\end{lstlisting}
\begin{enumerate}
\item[$\Rightarrow$] Move constructor:\index{Classes!move constructor}
\end{enumerate}
\begin{lstlisting}
// move constructor
MyVector::MyVector(MyVector&& v)
{
    n = v.n;
    m = v.m;
    e = v.e;
    v.n = 0;
    v.m = 0;
    v.e = nullptr;(*\index{\textbf{nullptr}}*)
}
\end{lstlisting}
\begin{enumerate}
\item[$\Rightarrow$] Copy assignment\\ \\ If not defined, C++ automatically adds  the default assignment operator.
It might not be correct if dynamic variables are used, because class members are simply copied:\index{Classes!copy assignment}
\end{enumerate}
\begin{lstlisting}
// copy assignment
MyVector& MyVector::operator=(const MyVector& rv)(*\index{Constants!\textbf{const}}*)(*\index{\texttt{operator=}}*)
{
    // check for self assignment
    if (this == &rv)(*\index{\textbf{this}}*)
        return *this;(*\index{\textbf{this}}*)
    // check if new allocation is needed
    if (rv.n > m)
    {
        if (e) delete[] e;(*\index{\textbf{delete}}*)
        e = new double[rv.n];(*\index{\textbf{new}}*)
        m = rv.n;
    }
    // copy the values
    copy(rv.e,rv.e+rv.n,e);(*\index{Algorithms!\texttt{copy}}*)
    n = rv.n;
    return *this;(*\index{\textbf{this}}*)
}
\end{lstlisting}
\begin{enumerate}
\item[$\Rightarrow$] Move assignment:\index{Classes!move assignment}
\end{enumerate}
\begin{lstlisting}
// move assignment
MyVector& MyVector::operator=(MyVector&& rv)(*\index{\texttt{operator=}}*)
{
    delete[] e;(*\index{\textbf{delete}}*)
    n = rv.n;
    m = rv.m;
    e = rv.e;
    rv.n = 0;
    rv.m = 0;
    rv.e = nullptr;(*\index{\textbf{nullptr}}*)
    return *this;(*\index{\textbf{this}}*)
}
\end{lstlisting}
\begin{enumerate}
\item[$\Rightarrow$] Reserve (reallocation), resize and push back:\index{Classes!reallocation of resources}
\end{enumerate}
\begin{lstlisting}
// reserve
void MyVector::reserve(size_t new_m)(*\index{\texttt{size\_t}}*)
{
    if (new_m <= m)
        return;
    // new allocation
    double* p = new double[new_m];(*\index{\textbf{new}}*)
    if (e)
    {
        copy(e,e+n,p);(*\index{Algorithms!\texttt{copy}}*)
        delete[] e;(*\index{\textbf{delete}}*)
    }
    e = p;
    m = new_m;
}

// resize
void MyVector::resize(size_t new_n)(*\index{\texttt{size\_t}}*)
{
    reserve(new_n);
    for (size_t i = n; i < new_n; i++) e[i] = double();(*\index{\texttt{size\_t}}*)
    n = new_n;
}

// push back
void MyVector::push_back(double d)
{
    if (m == 0)
        reserve(8);
    else if (n == m)
        reserve(2*m);
    e[n] = d;
    ++n;
}
\end{lstlisting}
\begin{enumerate}
\item[$\Rightarrow$] Constructor invocations:\index{Classes!constructor invocations}
\end{enumerate}
\begin{lstlisting}
// constructor with size
MyVector v1(4); 

// constructor with initializer list
MyVector v2{1,2,3,4};(*\index{\texttt{initializer\_list}}*)

// copy constructor
MyVector v3{v2}; 

// copy constructor
MyVector v3 = v2; 

// pass by value with copy constructor
// (prefer const reference!)
void func(MyVector v4)
{
    // do something
}
\end{lstlisting}
\begin{enumerate}
\item[$\Rightarrow$] Move invocations\\ \\ Avoids copying when moving is sufficient, e.g. when returning an object from a function:\index{Classes!move invocations}
\end{enumerate}
\begin{lstlisting}
// example of a function returning an object
MyVector func()
{
    MyVector v4{11,12,13,14,15};
    for (size_t i=0; i<v4.size(); i++) v4[i] += i;(*\index{\texttt{size\_t}}*)
    return v4;
}

// move constructor
MyVector v5 = func(); 

// move assignment
v4 = func();       
\end{lstlisting}
\begin{enumerate}
\item[$\Rightarrow$] Compiler generated methods \\ \\ If not implemented or deleted, a compiler will generate default implementations for the destructor, copy constructor, copy assignment, move constructor, move assignment (\emph{rule of five}):\index{Classes!compiler generated methods}\index{Classes!\emph{rule of five}}
\end{enumerate}
\begin{lstlisting}
// basic class with default copy and move semantics
// the compiler generates the default implementation
class Basic(*\index{Classes!\textbf{class}!\texttt{Basic}}*)
{
public:
    // default constructor and destructor
    Basic() = default;(*\index{\texttt{= default}}*)
    ~Basic() = default;(*\index{\texttt{= default}}*)
    // default copy constructor
    Basic(const Basic& b) = default;(*\index{\texttt{= default}}*)
    // default copy assignment
    Basic& operator=(const Basic& b) = default;(*\index{\texttt{= default}}*)(*\index{\texttt{operator=}}*)
     // default move constructor
    Basic(const Basic&& b) = default;(*\index{\texttt{= default}}*)
     // default move assignment
    Basic& operator=(const Basic&& b) = default;(*\index{\texttt{= default}}*)(*\index{\texttt{operator=}}*)
}

// fancy class with deleted copy and move semantics
// the compiler generates no default implementation
class Fancy(*\index{Classes!\textbf{class}!\texttt{Fancy}}*)
{
public:
    // no constructor and destructor
    Basic() = delete;(*\index{\texttt{= delete}}*)
    ~Basic() = delete;(*\index{\texttt{= delete}}*)
    // no copy constructor
    Basic(const Basic& b) = delete;(*\index{\texttt{= delete}}*)
    // no copy assignment
    Basic& operator=(const Basic& b) = delete;(*\index{\texttt{= delete}}*)(*\index{\texttt{operator=}}*)
     // no move constructor
    Basic(const Basic&& b) = delete;(*\index{\texttt{= delete}}*)
     // no move assignment
    Basic& operator=(const Basic&& b) = delete; (*\index{\texttt{= delete}}*)(*\index{\texttt{operator=}}*)
}
\end{lstlisting}

%
% Operator overloading
%
\section{Operator Overloading}
The behaviour is different if an operator is overloaded as  a class member or friend function.
\begin{enumerate}
\item[$\Rightarrow$] As class members\index{Operator overloading!as class member}
\end{enumerate}
\begin{lstlisting}
class Euro(*\index{Classes!\textbf{class}!\texttt{Euro}}*)
{
public:
    // constructor for  (*euro*)
    Euro(int euro);
    // constructor for  (*euro*) and (*cents*)
    Euro(int euro, int cents);
    Euro operator+(const Euro& amount);(*\index{\texttt{operator+}}*)(*\index{Constants!\textbf{const}}*)
private:
    int euro;
    int cents;
};
\end{lstlisting}
\begin{enumerate}
\item[$\Rightarrow$] The definition above requires a calling object:
\end{enumerate}
\begin{lstlisting}
// works, equivalent to (*Euro\{5\}.operator+( Euro\{2\} )*)
Euro result = Euro{5} + 2; 

// doesn't work, (*2*) is not a calling object of type (*Euro*) !
Euro result = 2 + Euro{5}; 
\end{lstlisting}
\begin{enumerate}
\item[$\Rightarrow$] As friend members\index{Operator overloading!as \textbf{friend} member}
\end{enumerate}
\begin{lstlisting}
#include <istream>
#include <ostream>

class Euro(*\index{Classes!\textbf{class}!\texttt{Euro}}*)
{
public:
    // constructor for  (*euro*)
    Euro(int euro);
    // constructor for (*euro*) and (*cents*)
    Euro(int euro, int cents);
    friend Euro operator+(const Euro& amount1,
        const Euro& amount2);(*\index{Constants!\textbf{const}}*)(*\index{\textbf{friend}}*)(*\index{\texttt{operator+}}*)
    // insertion and extraction operators
    friend ostream& operator<<(ostream& outs,
        const Euro& amount);(*\index{\texttt{operator<<}}*)(*\index{\textbf{friend}}*)(*\index{Constants!\textbf{const}}*)(*\index{\textbf{friend}}*)
    friend istream& operator>>(istream& ins,
        Euro& amount);(*\index{\texttt{operator>>}}*)(*\index{\textbf{friend}}*)
private:
    int euro;
    int cents;
};
\end{lstlisting}
\begin{enumerate}
\item[$\Rightarrow$] The definition above works for every combination because \textbf{int} arguments are converted by the constructor to \texttt{Euro} objects:
\index{Classes!constructors!type conversions}
\end{enumerate}
\begin{lstlisting}
// works, equivalent to (*Euro\{5\} + Euro\{2\}*)
Euro result = Euro{5} + 2; 

// works, equivalent to (*Euro\{2\} + Euro\{5\}*)
Euro result = 2 + Euro{5}; 
\end{lstlisting}
%
% Inheritance
%
\section{Inheritance}
\begin{enumerate}
\item[$\Rightarrow$] Abstract base class (excerpt):
\index{Inheritance!abstract base class}
\end{enumerate}
\begin{lstlisting}
class Shape : public Widget(*\index{Classes!\textbf{class}!\texttt{Shape}}*)
{
public:
    // no copy constructor allowed
    Shape(const Shape&) = delete;(*\index{Constants!\textbf{const}}*)(*\index{\texttt{= delete}}*)
    // no copy assignment allowed
    Shape& operator=(const Shape&) = delete;(*\index{Constants!\textbf{const}}*)(*\index{\texttt{= delete}}*)(*\index{\texttt{operator=}}*)
    // virtual destructor
    virtual ~Shape() {}(*\index{\textbf{virtual}}*)(*\index{Classes!virtual destructor}*)
    // overrides Fl_Widget::draw()
    void draw();
    // moves a shape relative to the current
    // top-left corner (call of redraw()
    // might be needed)
    void move(int dx, int dy);
    // setter and getter methods for
    // color, style, font, transparency
    // (call of redraw() might be needed)
    void set_color(Color_type c);
    void set_color(int c);
    Color_type get_color() const {
        return to_color_type(new_color);
    }(*\index{Constants!\textbf{const}}*)
    void set_style(Style_type s, int w);
    Style_type get_style() const {
        return to_style_type(line_style);
    }(*\index{Constants!\textbf{const}}*)
    void set_font(Font_type f, int s);
protected:
    // Shape is an abstract class,
    // no instances of Shape can be created!
    Shape() : Widget() {}
    // protected virtual methods to be overridden
    // by derived classes
    virtual void draw_shape() = 0;(*\index{\textbf{virtual}}*)(*\index{\texttt{= 0}}*)
    virtual void move_shape(int dx, int dy) = 0;(*\index{\textbf{virtual}}*)(*\index{\texttt{= 0}}*)
    // protected setter methods
    virtual void set_color_shape(Color_type c) {(*\index{\textbf{virtual}}*)
        new_color = to_fl_color(c);
    }
    virtual void set_color_shape(int c) {(*\index{\textbf{virtual}}*)
        new_color = to_fl_color(c);
    }
    virtual void set_style_shape(Style_type s, int w);(*\index{\textbf{virtual}}*)
    virtual void set_font_shape(Font_type f, int s);(*\index{\textbf{virtual}}*)
    // helper methods for FLTK style and font
    void set_fl_style();
    void restore_fl_style();
    void set_fl_font();
    void restore_fl_font() {
        fl_font(old_font,old_fontsize);
    }
    // test method for checking resize calls
    void draw_outline();
private:
    Fl_Color new_color{Fl_Color()};   // color
    Fl_Color old_color{Fl_Color()};   // old color
    Fl_Font new_font{0};              // font
    Fl_Font old_font{0};              // old font
    Fl_Fontsize new_fontsize{0};      // font size
    Fl_Fontsize old_fontsize{0};      // old font size
    int line_style{0};                // line style
    int line_width{0};                // line width
};
\end{lstlisting}
\begin{enumerate}
\item[$\Rightarrow$] A base class can be a derived class itself:
\index{Inheritance!base class}
\end{enumerate}
\begin{lstlisting}
// (*Shape*) is a base class for (*Line*)
// but (*Shape*) is derived from (*Widget*)
class Line : public Shape(*\index{Classes!\textbf{class}!\texttt{Line}}*)
{
    ...
};
\end{lstlisting}
\begin{enumerate}
\item[$\Rightarrow$] Disabling copy constructors and assignment\\ \\ Notice the \texttt{= delete} syntax for disabling them. If they were allowed, slicing\index{Slicing} might occur when derived objects are copied into base objects. Usually, \textbf{sizeof}\texttt{(Shape) <= }\textbf{sizeof}\texttt{(derived classes from Shape)}. By allowing copying, some attributes are not be copied, which might lead to crashes when member functions of the derived classes are called! Note that slicing is the class object equivalent of integer truncation.
\index{Inheritance!disabling copy constructors and assignment}
\end{enumerate}
\begin{lstlisting}
class Shape : public Widget(*\index{Classes!\textbf{class}!\texttt{Shape}}*)
{
public:
    // no copy constructor allowed
    Shape(const Shape&) = delete;(*\index{Constants!\textbf{const}}*)(*\index{\texttt{= delete}}*)
    // no copy assignment allowed
    Shape& operator=(const Shape&) = delete;(*\index{Constants!\textbf{const}}*)(*\index{\texttt{= delete}}*)(*\index{\texttt{operator=}}*)
    ...
};
\end{lstlisting}
\begin{enumerate}
\item[$\Rightarrow$] Virtual destructor\\ \\ Destructors should be declared \textbf{virtual}. When derived
objects are referenced by base class pointers, the destructor of the derived class is called if it is declared \textbf{virtual}.
\index{Inheritance!virtual destructor}
\end{enumerate}
\begin{lstlisting}
class Shape : public Widget(*\index{Classes!\textbf{class}!\texttt{Shape}}*)
{
public:
    ...
    // virtual destructor
    virtual ~Shape() {}(*\index{\textbf{virtual}}*)(*\index{Classes!virtual destructor}*)
    ...
};
\end{lstlisting}
\begin{enumerate}
\item[$\Rightarrow$] Protected constructor\\ \\ By declaring the constructor as \textbf{protected}, no instances of this class can be created by a user. Since \texttt{Shape} is an abstract class, it should be used only as a base class for derived classes.
\index{Inheritance!protected constructor}
\end{enumerate}
\begin{lstlisting}
class Shape : public Widget(*\index{Classes!\textbf{class}!\texttt{Shape}}*)
{
    ...
protected:
    ...
    // Shape is an abstract class
    // no instances of Shape can be created!
    Shape() : Widget() {}
    ...
};
\end{lstlisting}
\begin{enumerate}
\item[$\Rightarrow$] Protected member functions\\ \\ By declaring member functions as protected, access is restricted only to the class itself or to derived classes, a user cannot call such functions. This is useful for helper functions which are not supposed to be called directly outside the class.
\index{Inheritance!protected member functions}
\end{enumerate}
\begin{lstlisting}
class Shape : public Widget(*\index{Classes!\textbf{class}!\texttt{Shape}}*)
{
    ...
protected:
    ...
    // helper methods for FLTK style and font
    void set_fl_style();
    void restore_fl_style();
    void set_fl_font();
    void restore_fl_font() {
        fl_font(old_font,old_fontsize);
    }
    ...
};
\end{lstlisting}
\begin{enumerate}
\item[$\Rightarrow$] Pure virtual functions\\ \\ The protected member functions \texttt{draw\_shape()} and \texttt{move\_shape()} are pure virtual functions, i.e. a derived class must provide an implementation for them. Notice the syntax \index{\texttt{= 0}} which signals that the function is a pure virtual function. When a class has function members that are declared as pure virtual functions, then the class becomes an abstract class.
\index{Inheritance!pure virtual functions}
\end{enumerate}
\begin{lstlisting}
class Shape : Widget(*\index{Classes!\textbf{class}!\texttt{Shape}}*)
{
    ...
protected:
    ...
    // protected virtual methods to be overridden by 
    // derived classes
    virtual void draw_shape() = 0;(*\index{\texttt{= 0}}*)(*\index{\textbf{virtual}}*)
    virtual void move_shape(int dx, int dy) = 0;(*\index{\texttt{= 0}}*)(*\index{\textbf{virtual}}*)
    ...
};
\end{lstlisting}
\begin{enumerate}
\item[$\Rightarrow$] Virtual functions\\ \\ The protected member functions \texttt{set\_color\_shape()} is declared as a virtual function and an implementation is provided. This means that if a derived class does not override the implementation of the base class, the derived class inherits the implementation from the base class.
\index{Inheritance!virtual functions}
\end{enumerate}
\begin{lstlisting}
class Shape : Widget(*\index{Classes!\textbf{class}!\texttt{Shape}}*)
{
    ...
protected:
    ...
    // protected setter methods
    virtual void set_color_shape(Color_type c) {(*\index{\textbf{virtual}}*)
        new_color = to_fl_color(c);
    }
    virtual void set_color_shape(int c) {(*\index{\textbf{virtual}}*)
        new_color = to_fl_color(c);
    }
    ...
};
\end{lstlisting}
\begin{enumerate}
\item[$\Rightarrow$] A derived class from the base class \texttt{Shape}:
\index{Inheritance!derived class}
\end{enumerate}
\begin{lstlisting}
class Line : public Shape(*\index{Classes!\textbf{class}!\texttt{Line}}*)
{
public:
    Line(pair<Point,Point> line) : l{line} {(*\index{\texttt{pair}}*)
        resize_shape(l.first,l.second);
    }
    virtual ~Line() {}(*\index{\textbf{virtual}}*)(*\index{Classes!virtual destructor}*)
    pair<Point,Point> get_line() const { return l; }(*\index{Constants!\textbf{const}}*)(*\index{\texttt{pair}}*)
    void set_line(pair<Point,Point> line) { l = line; }(*\index{\texttt{pair}}*)
protected:
    void draw_shape() {
        fl_line(l.first.x, l.first.y, l.second.x,
            l.second.y);
    }
    void move_shape(int dx, int dy) {
        l.first.x  += dx;  l.first.y += dy;
        l.second.x += dx; l.second.y += dy;
        resize_shape(l.first,l.second);
    }
private:
    pair<Point,Point> l;(*\index{\texttt{pair}}*)
};
\end{lstlisting}
\begin{enumerate}
\item[$\Rightarrow$] \texttt{Line} is derived from \texttt{Shape}, it models the relationship that a \texttt{Line} is a \texttt{Shape}
\index{Inheritance!derived class}
\end{enumerate}
\begin{lstlisting}
class Line : public Shape(*\index{Classes!\textbf{class}!\texttt{Line}}*)
{
    ...
};
\end{lstlisting}
\begin{enumerate}
\item[$\Rightarrow$] \texttt{Line} has its own getter and setter functions for accessing its own internal private representation:
\index{Classes!getter and setter functions}
\end{enumerate}
\begin{lstlisting}
class Line : public Shape(*\index{Classes!\textbf{class}!\texttt{Line}}*)
{
public:
    ...
    pair<Point,Point> get_line() const { return l; }(*\index{Constants!\textbf{const}}*)(*\index{\texttt{pair}}*)
    void set_line(pair<Point,Point> line) { l = line; }(*\index{\texttt{pair}}*)
    ...
private:
    pair<Point,Point> l;(*\index{\texttt{pair}}*)
};
\end{lstlisting}
\begin{enumerate}
\item[$\Rightarrow$] \texttt{Line} specialises the virtual functions \texttt{draw\_shape()} and \texttt{move\_shape()} according to its representation:
\index{Inheritance!function specialisation}
\end{enumerate}
\begin{lstlisting}
class Line : public Shape(*\index{Classes!\textbf{class}!\texttt{Line}}*)
{
public:
    ...
protected:
    void draw_shape() {
        fl_line(l.first.x, l.first.y, l.second.x,
            l.second.y);
    }
    void move_shape(int dx, int dy) {
        l.first.x  += dx;  l.first.y += dy;
        l.second.x += dx; l.second.y += dy;
        resize_shape(l.first,l.second);
    }
    ...
};
\end{lstlisting}
\begin{enumerate}
\item[$\Rightarrow$] \texttt{Circle} is also derived from \texttt{Shape}, a \texttt{Circle} is also a \texttt{Shape}.
\index{Inheritance!function specialisation}
\end{enumerate}
\begin{lstlisting}
class Circle : public Shape(*\index{Classes!\textbf{class}!\texttt{Circle}}*)
{
public:
    Circle(Point a, int rr) : c{a}, r{rr} {
        resize_shape(Point{c.x-r,c.y-r},
            Point{c.x+r,c.y+r});
    }
    virtual ~Circle() {}(*\index{\textbf{virtual}}*)(*\index{Classes!virtual destructor}*)
    Point get_center() const { return c; }(*\index{Constants!\textbf{const}}*)
    void set_center(Point p) {
        c = p;
        resize_shape(Point{c.x-r,c.y-r},
            Point{c.x+r,c.y+r});
    }
    int get_radius() const { return r; }(*\index{Constants!\textbf{const}}*)
    void set_radius(int rr) {
        r = rr;
        resize_shape(Point{c.x-r,c.y-r},
            Point{c.x+r,c.y+r});
    }
protected:
    void draw_shape() {
        Point tl = get_tl();
        Point br = get_br();
        fl_arc(tl.x,tl.y,br.x-tl.x,br.y-tl.y,0,360);
    }
    void move_shape(int dx,int dy) {
        c.x += dx; c.y += dy;
        resize_shape(Point{c.x-r,c.y-r},
            Point{c.x+r,c.y+r});
    }
private:
    Point c{}; // center
    int r{0};  // radius
};
\end{lstlisting}
%
% Polymorphism
%
\section{Polymorphism}
\begin{enumerate}
\item[$\Rightarrow$] From a window perspective, it is possible to attach and draw any type of widget, and the window just needs to call the \texttt{Fl\_Widget::draw()} method:
\index{Polymorphism}
\end{enumerate}
\begin{lstlisting}
void Window::draw(Fl_Widget& w) {
    w.draw();
}
\end{lstlisting}
\begin{enumerate}
\item[$\Rightarrow$] Since \texttt{Fl\_Widget::draw()} is a pure virtual function, it is overridden by
\texttt{Shape::draw()}, which in turn calls the pure virtual function \texttt{Shape::draw\_shape()}, which gets specialised in every derived class, e.g. as in \texttt{Line} or \texttt{Circle}:
\end{enumerate}
\begin{lstlisting}
void Shape::draw() {
    set_fl_style();
    if ( is_visible() ) draw_shape();
    restore_fl_style();
}

void Circle:: draw_shape() {
    Point tl = get_tl();
    Point br = get_br();
    fl_arc(tl.x,tl.y,br.x-tl.x,br.y-tl.y,0,360);
}

void Line::draw_shape() {
    fl_line(l.first.x, l.first.y, l.second.x, l.second.y);
}
\end{lstlisting}
\begin{enumerate}
\item[$\Rightarrow$] Polymorphism is allowed by the \textbf{virtual} keyword which guarantees late binding: the call \texttt{w.draw()} inside \texttt{Windows::draw()} binds to the \texttt{draw\_shape()} function of the actual object referenced, either to a \texttt{Line} or \texttt{Circle} instance.
\index{Polymorphism!late binding}
\end{enumerate}
\begin{lstlisting}
Window win;
Line diagonal { {Point{200,200},Point{250,250}} };
Circle c1{Point{100,200},50};

win.draw(diagonal); // calls (*Line::draw\_shape()*)
win.draw(c1); // calls (*Circle::draw\_shape()*)
\end{lstlisting}
%
% Advanced topics
%
\chapter{Advanced Topics}
%
% Exceptions
%
\section{Exceptions}
\begin{enumerate}
\item[$\Rightarrow$] The value thrown by \textbf{throw} can be of any type:
\end{enumerate}
\begin{lstlisting}
// exception class
class MyException(*\index{Classes!\textbf{class}!\texttt{MyException}}*)
{
public:
    MyException(string s);(*\index{Strings!\texttt{string}}*)
    virtual ~MyException();(*\index{\textbf{virtual}}*)(*\index{Classes!virtual destructor}*)
    friend ostream& operator<<(ostream& os,
        const MyException& e);(*\index{Constants!\textbf{const}}*)(*\index{\textbf{friend}}*)(*\index{\texttt{operator<<}}*)
protected:
    string msg;(*\index{Strings!\texttt{string}}*)
};

try
{
    throw MyException("error");(*\index{\textbf{throw}}*)
}
catch (MyException& e)
{
    // error stream
    cerr << e;(*\index{Input-output streams!\texttt{cerr}}*)
}
// everything else
catch (...)
{
    exit(1);
}
\end{lstlisting}
\begin{enumerate}
\item[$\Rightarrow$] The standard library defines a hierarchy of exceptions.  For example \texttt{runtime\_error} can be thrown when runtime errors occur:
\end{enumerate}
\begin{lstlisting}
try(*\index{Exceptions!\textbf{try-catch}}*)
{
    throw runtime_error("unexpected result!");(*\index{\textbf{throw}}*)(*\index{Exceptions!\texttt{runtime\_error}}*)
}
catch (runtime_error& e)
{
    // error stream
    cerr << "runtime error: " << e.what() << "\n";(*\index{Input-output streams!\texttt{cerr}}*)(*\index{Exceptions!\texttt{runtime\_error}!\texttt{what}}*)
    return 1;
}
\end{lstlisting}
\begin{enumerate}
\item[$\Rightarrow$] Functions throwing exceptions should list the exceptions thrown in
the exception specification list. These exceptions are not caught by the function itself!
\end{enumerate}
\begin{lstlisting}
// exceptions of type (*DivideByZero*) or (*OtherException*) are(*\index{Exceptions!\texttt{DivideByZero}}\index{Exceptions!\texttt{OtherException}}*)
// to be caught outside the function. All other exceptions 
// end the program if not caught inside the function.
void my_function( ) throw (DivideByZero, OtherException);(*\index{\textbf{throw}}*)

// empty exception list, i.e. all exceptions end the
// program if thrown but not caught inside the function.
void my_function( ) throw ( );(*\index{\textbf{throw}}*)

// all exceptions of all types treated normally.
void my_function( );
\end{lstlisting}
\begin{enumerate}
\item[$\Rightarrow$] \emph{Basic guarantee}\index{Exceptions!Basic guarantee}\\ \\ Any part of your code should either succeed or throw an exception without leaking any resource:
\end{enumerate}
\begin{lstlisting}
// Does local cleanup avoiding leaking of resources
// if exception occurs
void my_function(void)
{
    void *p;(*\index{Pointers!\texttt{void $\ast$}}*) 
    socket *s;
    
    try(*\index{Exceptions!\textbf{try-catch}}*)
    {
        /* code that acquires some resource (memory, 
           socket, etc.) and might throw an exception */
    }
    catch (...)
    { 
        /* local cleanup here */
        delete p;      /* free memory */(*\index{\textbf{delete}}*)
        s.release();  /* release socket */  
        /* re-throw because function didn't succeed */
        throw()(*\index{\textbf{throw}}*)
    }    
}
\end{lstlisting}
%
% Templates
%
\section{Templates}
Types are used as parameters for a function or a class.  C++ does not need the template declaration. Always put the 
template definition in the header file directly!
\begin{enumerate}
\item[$\Rightarrow$] Function template:\index{Templates!function}
\end{enumerate}
\begin{lstlisting}
// generic swap function
template<class T>(*\index{\textbf{template}|see{Templates}}*)(*\index{Templates!\textbf{template}}*)
void generic_swap(T& a, T& b)
{
    T temp = a;
    
    a = b;
    b = temp;
}

int a, b;
char c, d;

// swaps two (*int*)s
generic_swap<int>(a, b);

// swaps two (*char*)s
generic_swap<char>(c, d);
\end{lstlisting}
\begin{enumerate}
\item[$\Rightarrow$] Template type deduction\\ \\ The compiler infers the template parameter from the usage:\index{Templates!template type deduction}
\end{enumerate}
\begin{lstlisting}
double e, f;

// swaps two (*double*)s
// compiler infers the template parameter from usage
generic_swap(a, b);
\end{lstlisting}
\begin{enumerate}
\item[$\Rightarrow$] Constrain template types with assertions and type traits:
\index{Templates!assertions}\index{Templates!type traits}
\end{enumerate}
\begin{lstlisting}
#include <type_traits>(*\index{\texttt{<type\_traits>}}*)

template<class T>
void generic_swap(T& a, T& b)
{
    static_assert(std::is_copy_constructable<T>(),
        "Type must be copy constructable");(*\index{\texttt{is\_copy\_constructable<T>()}}*)(*\index{\textbf{static\_assert}}*)(*\index{Assertions!\textbf{static\_assert}}*)
    static_assert(std::is_assignable<T&,T>(),"Type must allow T& = T");(*\index{\texttt{is\_assignable<T\&,T>()}}*)(*\index{Assertions!\textbf{static\_assert}}*)
    
    T temp = a;
    
    a = b;
    b = temp;
}
\end{lstlisting}
\begin{enumerate}
\item[$\Rightarrow$] Class templates\\ \\ Extending \texttt{MyVector} with templates. Class templates are also called \emph{type generators}:
\index{Templates!class}\index{Templates!type generator}
\end{enumerate}
\begin{lstlisting}
template<class T>(*\index{Templates!\textbf{template}}*)
class MyVector(*\index{Classes!\textbf{class}!\texttt{MyVector}}*)
{
public:
    // constructor
    explicit MyVector();(*\index{\textbf{explicit}}*)
    // constructor with size
    explicit MyVector(size_t);(*\index{\textbf{explicit}}*)(*\index{\texttt{size\_t}}*)
    // constructor with initializer list
    explicit MyVector(initializer_list<T>);(*\index{\textbf{explicit}}*)(*\index{\texttt{initializer\_list}}*)
    // copy constructor (pass by
    // reference, no copying!)
    MyVector(const MyVector&);(*\index{Constants!\textbf{const}}*)
    // move constructor
    MyVector(MyVector&&);
    // copy assignment
    MyVector& operator=(const MyVector&);(*\index{Constants!\textbf{const}}*)(*\index{\texttt{operator=}}*)
    // move assignment
    MyVector& operator=(MyVector&&);(*\index{\texttt{operator=}}*)
    // virtual destructor
    virtual ~MyVector() { if (e) delete[] e; }(*\index{\textbf{delete}}*)(*\index{\textbf{virtual}}*)(*\index{Classes!virtual destructor}*)
    // subscript operators
    // write
    T& operator[](size_t i) { return e[i]; }(*\index{\texttt{size\_t}}*)
    // read
    const T& operator[](size_t i) const { return e[i]; };(*\index{Constants!\textbf{const}}*)(*\index{\texttt{size\_t}}*)
    // size
    size_t size() const { return n; }(*\index{Constants!\textbf{const}}*)(*\index{\texttt{size\_t}}*)
    // capacity
    size_t capacity() const { return m; }(*\index{Constants!\textbf{const}}*)(*\index{\texttt{size\_t}}*)
    // reserve
    void reserve(size_t);(*\index{\texttt{size\_t}}*)
    // resize
    void resize(size_t);(*\index{\texttt{size\_t}}*)
    // push back
    void push_back(T);
private:
    size_t n{0}; // size(*\index{\texttt{size\_t}}*)
    size_t m{0}; // capacity(*\index{\texttt{size\_t}}*)
    T *e{nullptr};(*\index{\textbf{nullptr}}*)
};
\end{lstlisting}
\begin{enumerate}
\item[$\Rightarrow$] Method definition with templates:\index{Templates!method definition}
\end{enumerate}
\begin{lstlisting}
// copy assignment
template<class T>(*\index{Templates!\textbf{template}}*)
MyVector<T>& MyVector<T>::operator=(const MyVector<T>& rv)(*\index{Classes!\textbf{class}!\texttt{MyVector<T>}}*)(*\index{Constants!\textbf{const}}*)(*\index{\texttt{operator=}}*)
{
    // check for self assignment
    if (this == &rv)(*\index{\textbf{this}}*)
        return *this;(*\index{\textbf{this}}*)
    // check if new allocation is needed
    if (rv.n > m)
    {
        if (e) delete[] e;(*\index{\textbf{delete}}*)
        e = new T[rv.n];(*\index{\textbf{new}}*)
        m = rv.n;
    }
    // copy the values
    copy(rv.e,rv.e+rv.n,e);(*\index{Algorithms!\texttt{copy}}*)
    n = rv.n;
    return *this;(*\index{\textbf{this}}*)
}
\end{lstlisting}
\begin{enumerate}
\item[$\Rightarrow$] \emph{Specialisation} or \emph{template instantiation}:\index{Templates!specialisation}\index{Templates!instantiation}
\end{enumerate}
\begin{lstlisting}
// (*MyVector*) of (*double*)
MyVector<double> v4{11,12,13,14,15};

// function returning a (*MyVector*) of (*double*)
MyVector<double> func()
{
    MyVector<double> v4{11,12,13,14,15};
    for (size_t i=0; i<v4.size(); i++) v4[i] += i;(*\index{\texttt{size\_t}}*)
    return v4;
}
\end{lstlisting}
\begin{enumerate}
\item[$\Rightarrow$] Non-type template parameters\index{Templates!non-type parameters}
\end{enumerate}
\begin{lstlisting}
// Wrapper class for an array 
template<class T,size_t N>(*\index{Templates!\textbf{template}}*)
class Wrapper(*\index{Classes!\textbf{class}!\texttt{Wrapper}}*)
{
public:
    Wrapper() { for(T& e : v)  e=T(); }
    ~Wrapper() {}
    T& operator[](int n) { return v[n]; };
    const T& operator[](int n) const { return v[n]; };(*\index{Constants!\textbf{const}}*)
    size_t size() const { return N; }(*\index{Constants!\textbf{const}}*)(*\index{\texttt{size\_t}}*)
private:
    T v[N];
};

// usage
Wrapper<double,5> array;
Wrapper<char,3> array;
\end{lstlisting}
\begin{enumerate}
\item[$\Rightarrow$] Allocator as a class template parameter\index{Templates!allocator}
\end{enumerate}
\begin{lstlisting}
// Usage of an allocator as a class template parameter
// Generalises (*\texttt{MyVector}*) for data types without a default
// constructor and with customised memory management
template<class T, class A=allocator<T>>(*\index{\texttt{allocator}}*)(*\index{Templates!\textbf{template}}*)
class MyVector(*\index{Classes!\textbf{class}!\texttt{MyVector<T,A>}}*)
{
public:
    // constructor
    explicit MyVector();(*\index{\textbf{explicit}}*)
    // constructor with size and default value
    explicit MyVector(size_t,T def = T());(*\index{\textbf{explicit}}*)(*\index{\texttt{size\_t}}*)
    // constructor with initializer list
    explicit MyVector(initializer_list<T>);(*\index{\textbf{explicit}}*)(*\index{\texttt{initializer\_list}}*)
    // copy constructor (pass by
    // reference, no copying!)
    MyVector(const MyVector&);(*\index{Constants!\textbf{const}}*)
    // move constructor
    MyVector(MyVector&&);
    // copy assignment
    MyVector& operator=(const MyVector&);(*\index{Constants!\textbf{const}}*)(*\index{\texttt{operator=}}*)
    // move assignment
    MyVector& operator=(MyVector&&);(*\index{\texttt{operator=}}*)
    // virtual destructor
    virtual ~MyVector();(*\index{\textbf{virtual}}*)(*\index{Classes!virtual destructor}*)
    // subscript operators
    // write
    T& operator[](size_t i) { return e[i]; }
    // read
    const T& operator[](size_t i) const { return e[i]; };(*\index{Constants!\textbf{const}}*)
    // size
    size_t size() const { return n; }(*\index{Constants!\textbf{const}}*)(*\index{\texttt{size\_t}}*)
    // capacity
    size_t capacity() const { return m; }(*\index{Constants!\textbf{const}}*)(*\index{\texttt{size\_t}}*)
    // reserve
    void reserve(size_t);(*\index{\texttt{size\_t}}*)
    // resize
    void resize(size_t,T def = T());(*\index{\texttt{size\_t}}*)
    // push back
    void push_back(T);
private:
    A alloc;
    size_t n{0}; // size(*\index{\texttt{size\_t}}*)
    size_t m{0}; // capacity(*\index{\texttt{size\_t}}*)
    T *e{nullptr};(*\index{\textbf{nullptr}}*)
};

// reserve
template<class T,class A>(*\index{Templates!\textbf{template}}*)
void MyVector<T,A>::reserve(size_t new_m)(*\index{\texttt{size\_t}}*)
{
    if (new_m <= m)
        return;
    // new allocation
    T* p = alloc.allocate(new_m);
    if (e)
    {
        // copy
        for (size_t i=0; i<n; ++i)
            alloc.construct(&p[i],e[i]);(*\index{\texttt{size\_t}}*)
        // destroy
        for (size_t i=0; i<n; ++i)
            alloc.destroy(&e[i]);(*\index{\texttt{size\_t}}*)
        // deallocate
        alloc.deallocate(e,m);
    }
    e = p;
    m = new_m;
}
\end{lstlisting}
\begin{enumerate}
\item[$\Rightarrow$] Template friend operator:\index{Templates!friend operator}
\end{enumerate}
\begin{lstlisting}
// Note the declaration of the template friend operator.
template<class T>(*\index{Templates!\textbf{template}}*)
class SimpleNode(*\index{Classes!\textbf{class}!\texttt{SimpleNode<T>}}*)
{
    // constructor with size of the list
    SimpleNode(int size);
    // destructor
    ~SimpleNode();
    // copy constructor
    SimpleNode(ListNode<T>& b);
    // assignment operator
    SimpleNode<T>& operator=(const SimpleNode<T>& b);(*\index{Constants!\textbf{const}}*)(*\index{\texttt{operator=}}*)
    // friend insertion operator
    template <class TT>(*\index{Templates!\textbf{template}}*)
    friend ostream& operator<<(ostream& outs,
        const SimpleNode<TT>& rhs);(*\index{Constants!\textbf{const}}*)(*\index{\textbf{friend}}*)(*\index{\texttt{operator<<}}*)
private:
    T *p;
    int size;
}
\end{lstlisting}
%
% Iterators
%
\section{Iterators}
\begin{enumerate}
\item[$\Rightarrow$] An iterator is a generalisation of a pointer. It is an object that identifies an element of a sequence. Different containers have
different iterators.\index{Iterators}
\end{enumerate}
\begin{lstlisting}
#include <vector>(*\index{\texttt{<vector>}}*)

vector<int> v = {1,2,3,4,5};(*\index{Containers!\texttt{vector}}*)
// mutable iterator
vector<int>::iterator e;(*\index{\texttt{iterator}|see{Iterators}}*)(*\index{Iterators!\texttt{vector}!\texttt{iterator}}*)

// bidirectional access
e = v.begin();(*\index{Iterators!bidirectional access}*)(*\index{Iterators!\texttt{vector}!\texttt{iterator}}*)
++e;
// print (*v[1]*)
cout << *e << endl;
--e;
// print (*v[0]*)
cout << *e << endl;

// random access(*\index{Iterators!random access}*)
e = v.begin();(*\index{Iterators!\texttt{vector}!\texttt{iterator}}*)
// print (*v[3]*)
cout << e[3] << endl;

// change an element
e[3] = 9;
\end{lstlisting}
\begin{enumerate}
\item[$\Rightarrow$] Constant iterator
\end{enumerate}
\begin{lstlisting}
// constant iterator (only read)
vector<int>::const_iterator c;(*\index{\texttt{const\_iterator}|see{Iterators}}*)(*\index{Iterators!\texttt{vector}!\texttt{const\_iterator}}*)

// print out the vector content (read only)
// (*\texttt{end()}*) points one element beyond the last one!
for (c = v.begin(); c != v.end(); c++)
    cout << *c << endl;(*\index{range-for-loop}*)

// not allowed
// (*c[2] = 2;*)
\end{lstlisting}
\begin{enumerate}
\item[$\Rightarrow$] Reverse iterator
\end{enumerate}
\begin{lstlisting}
// reverse iterator
vector<int>::reverse_iterator r;(*\index{\texttt{reverse\_iterator}|see{Iterators}}*)(*\index{Iterators!\texttt{vector}!\texttt{reverse\_iterator}}*)

// print out the vector content in reverse order
for (r = v.rbegin(); r != v.rend(); r++)
    cout << *r << endl;(*\index{range-for-loop}*)
\end{lstlisting}
\begin{enumerate}
\item[$\Rightarrow$] Example iterator class for a power range:\index{Iterators!custom implementation}
\end{enumerate}
\begin{lstlisting}
#include <cstdint>(*\index{\texttt{<cstdint>}}*)

// power range class
class PowerRange(*\index{Classes!\textbf{class}!\texttt{PowerRange}}*)
{
public:
    PowerRange(uint32_t m) : max{m} {
        for (auto n=0; n<max; n++) v.push_back(n*n);
    }
    ~PowerRange() {}
    class iterator;
    iterator begin();
    iterator end();
private:
    uint32_t max{0};
    vector<uint32_t> v;
};

// iterator class
class PowerRange::iterator
{
public:
    iterator(uint32_t *p) : curr{p} { }
    // postfix
    iterator operator++(int) {
        ++curr;
        return iterator{curr-1};
    }(*\index{Operator overloading!postfix increment ++}*)(*\index{\texttt{operator++(int)}}*)
    // prefix
    iterator& operator++() {
        ++curr;
        return *this;
    }(*\index{Operator overloading!prefix increment ++}*)(*\index{\texttt{operator++}}*)
    uint32_t operator*() { return *curr; }(*\index{\texttt{operator$\ast$}}*)
    bool operator!=(const iterator& e) {
        return curr != e.curr;
    }(*\index{\texttt{operator"!=}}*)
    bool operator==(const iterator& e) {
        return curr == e.curr;
    }(*\index{\texttt{operator==}}*)
private:
    uint32_t *curr{nullptr};
};

// returns the first element
PowerRange::iterator PowerRange::begin()
{
    return PowerRange::iterator(&v[0]);
}

// returns one element beyond the end
PowerRange::iterator PowerRange::end()
{
    return PowerRange::iterator{&v[max]};
}

// example usage
PowerRange r{10};
    
// normal (*for*) loop
for (auto x = r.begin(); x != r.end(); x++)
    cout << *x << endl;

// range-based (*for*) loop
for (auto x:r)(*\index{range-for-loop}*)
    cout << x << endl;
\end{lstlisting}
\begin{enumerate}
\item[$\Rightarrow$] A linked list class:
\end{enumerate}
\begin{lstlisting}
#include <ostream>(*\index{\texttt{<ostream>}}*)
#include <algorithm>(*\index{\texttt{<algorithm>}}*)

// node of the linked list
template <class T>(*\index{Templates!\textbf{template}}*)
class LListNode(*\index{Classes!\textbf{class}!\texttt{LListNode<T>}}*)
{
public:
    // constructor for a new node
    LListNode(T new_data = T(),
        LListNode<T>* new_next = nullptr) :(*\index{\textbf{nullptr}}*)
    data(new_data), next(new_next) {};
    // friends
    friend class LList<T>;(*\index{\textbf{friend}}*)
    template <class TT>(*\index{Templates!\textbf{template}}*)
    friend ostream& operator<<(ostream& outs,
        const LList<TT>& rhs);(*\index{Constants!\textbf{const}}*)(*\index{\textbf{friend}}*)(*\index{\texttt{operator<<}}*)
private:
    // data element
    T data{T()};
    // next pointer
    LListNode<T>* next{nullptr};(*\index{\textbf{nullptr}}*)
};

// linked list declaration
template <class T>(*\index{Templates!\textbf{template}}*)
class LList(*\index{Classes!\textbf{class}!\texttt{LList<T>}}*)
{
public:
    // default constructor
    LList() : head(nullptr) {};(*\index{\textbf{nullptr}}*)
    // copy constructor
    LList(const LList<T>& rhs) { *this = rhs; };(*\index{\textbf{this}}*)
    // assignment operator
    LList<T>& operator=(const LList<T>& rhs);(*\index{Constants!\textbf{const}}*)(*\index{\texttt{operator=}}*)
    // virtual destructor
    virtual ~LList() { clear(); };(*\index{\textbf{virtual}}*)(*\index{Classes!virtual destructor}*)
    // clear (free) the list
    void clear();
    // get head
    LListNode<T>* get_head() const { return head; };(*\index{Constants!\textbf{const}}*)
    // get node
    LListNode<T>* get_node(int n=0) const;
    // insert a new data element at the head of the list
    void insert_at_head(T new_data);
    // insert a new data element at the end of the list
    void insert_at_end(T new_data);
    // insert a new element at a given pointed node
    void insert_at_point(LListNode<T>* ptr, T new_data);
    // remove the data element at the head of the list
    T remove_head();
    // test for empty list
    bool is_empty() const { return head == nullptr; };(*\index{Constants!\textbf{const}}*)(*\index{\textbf{nullptr}}*)
    // count of the elements stored in the list
    int size() const;
    // insertion operator
    template <class TT>(*\index{Templates!\textbf{template}}*)
    friend ostream& operator<<(ostream& outs,
        const LList<TT>& rhs);(*\index{Constants!\textbf{const}}*)(*\index{\textbf{friend}}*)(*\index{\texttt{operator<<}}*)
    // iterator type
    class iterator;(*\index{Iterators!\texttt{LList<T>}!\texttt{iterator}}*)
    // iterator to first element
    iterator begin() { return iterator(head); }(*\index{Iterators!\texttt{LList<T>}!\texttt{iterator}!\texttt{begin}}*)
    // iterator to one beyond last element
    iterator end() { return iterator(nullptr); }(*\index{\textbf{nullptr}}*)(*\index{Iterators!\texttt{LList<T>}!\texttt{iterator}!\texttt{end}}*)
private:
    // head pointer
    LListNode<T>* head{nullptr};(*\index{\textbf{nullptr}}*)
    // recursive copy list function
    LListNode<T>* recursive_copy(LListNode<T>* rhs);
};
\end{lstlisting}
\begin{enumerate}
\item[$\Rightarrow$] Example iterator class for the custom linked list class above\index{Iterators!custom implementation}
\end{enumerate}
\begin{lstlisting}
// iterator class for the linked list
template <class T>(*\index{Templates!\textbf{template}}*)
class LList<T>::iterator(*\index{Iterators!\texttt{LList<T>}!\texttt{iterator}}*)
{
public:
    iterator(LListNode<T>* p) : curr{p} {}
    // prefix increment, returns a reference!(*\index{Operator overloading!prefix increment ++}*)
    iterator& operator++() {
        curr = curr->next;
        return *this;
    }(*\index{\textbf{this}}*)(*\index{\texttt{operator++}}*)
    T& operator*() const {
        return curr->data;
    }(*\index{Constants!\textbf{const}}*)(*\index{\texttt{operator$\ast$}}*)
    bool operator==(const iterator& b) const {
        return curr == b.curr;
    }(*\index{Constants!\textbf{const}}*)(*\index{\texttt{operator==}}*)
    bool operator!=(const iterator& b) const {
        return curr != b.curr;
    }(*\index{Constants!\textbf{const}}*)(*\index{\texttt{operator"!=}}*)
private:
    LListNode<T>* curr{nullptr};(*\index{\textbf{nullptr}}*)
};

// example usage
LList<int> data_list;

// inserts element into the list
data_list.insert_at_head(45);
data_list.insert_at_head(-21);
data_list.insert_at_end(127);

// prints (*data\_list = (-21) -> (45) -> (127)*)
cout << "data_list = " << data_list << endl;
// prints  (*data\_list.size() = 3*)
cout << "data_list.size() = " << data_list.size();

// applies standard algorithms on the custom linked list
LList<int>::iterator p = find(data_list.begin(),
    data_list.end(),45);(*\index{\texttt{find}|see{Algorithms}}*)(*\index{Algorithms!\texttt{find}}*)(*\index{Iterators!\texttt{LList<T>}!\texttt{iterator}}*)(*\index{Iterators!\texttt{LList<T>}!\texttt{iterator}!\texttt{begin}}*)(*\index{Iterators!\texttt{LList<T>}!\texttt{iterator}!\texttt{end}}*)

// checks if the element has been found(*\index{Iterators!end of sequence convention}*)
// standard algorithms return the end of a sequence, 
// i.e. the (*end*) iterator, to indicate failure
if ( p != second_list.end() )
        cout << "found element " << *p << "\n\n";
else
        cout << "cannot find element " << 45 << "\n\n";

// write access
*p = 180;

// prints (*data\_list = (-21) -> (180) -> (127)*)
cout << "data_list = " << data_list << endl;
\end{lstlisting}
%
% Containers
%
\section{Containers}
\begin{enumerate}
\item[$\Rightarrow$] Sequential containers: \texttt{list}\index{Containers!sequential|see{\texttt{list}}}\index{Containers!sequential|see{\texttt{vector}}}
\end{enumerate}
\begin{lstlisting}
#include <list>(*\index{\texttt{<list>}}*)

list<double> data = {1.32,-2.45,5.65,-8.93,2.76};(*\index{\texttt{list}|see{Containers}}*)(*\index{Containers!\texttt{list}}*)

// adds elements
data.push_back(9.23);(*\index{Containers!\texttt{list}!\texttt{push\_back}}*)
data.push_front(-3.94);(*\index{Containers!\texttt{list}!\texttt{push\_front}}*)

// bidirectional iterator, no random access    
list<double>::iterator e;(*\index{Containers!\texttt{list}}*)(*\index{Iterators!\texttt{list}!\texttt{iterator}}*)

// (*advance*)
e = data.begin();
advance(e,2);(*\index{Algorithms!\texttt{advance}}*)

// erases element (*5.65*)
data.erase(e);(*\index{Containers!\texttt{list}!\texttt{erase}}*)

// print out the content    
for (e = data.begin(); e != data.end(); e++)
    cout << *e << endl;
    
// range-for-loop
for (auto x : data)(*\index{range-for-loop}*)
    cout << x << endl;(*\index{\textbf{auto}}*)
\end{lstlisting}
\begin{enumerate}
\item[$\Rightarrow$] Container adapters: \texttt{stack}\index{Containers!adapter|see{\texttt{stack}}}
\end{enumerate}
\begin{lstlisting}
#include <stack>(*\index{\texttt{<stack>}}*)

stack<double> numbers;(*\index{\texttt{stack}|see{Containers}}*)(*\index{Containers!\texttt{stack}}*)

// push on the stack
numbers.push(5.65);(*\index{Containers!\texttt{stack}!\texttt{push}}*)
numbers.push(-3.95);
numbers.push(6.95);

// size
cout << numbers.size()(*\index{Containers!\texttt{stack}!\texttt{size}}*)(*\index{Input-output streams!\texttt{cout}}*)

// read top data element
double d = numbers.top();(*\index{Containers!\texttt{stack}!\texttt{top}}*)

// pop top element
numbers.pop();(*\index{Containers!\texttt{stack}!\texttt{pop}}*)
\end{lstlisting}
\begin{enumerate}
\item[$\Rightarrow$] Associative containers: \texttt{set}, ordered according to its unique keys\index{Containers!associative|see{\texttt{set}}}
\end{enumerate}
\begin{lstlisting}
#include <set>(*\index{\texttt{<set>}}*)

set<char> letters;(*\index{\texttt{set}|see{Containers}}*)(*\index{Containers!\texttt{set}}*)

// inserting elements    
letters.insert('a');(*\index{Containers!\texttt{set}!\texttt{insert}}*)
letters.insert('d');
// no duplicates!
letters.insert('d');
letters.insert('g');

// erase    
letters.erase('a');(*\index{Containers!\texttt{set}!\texttt{erase}}*)

// const iterator 
set<char>::const_iterator c;(*\index{Constants!\textbf{const}}*)(*\index{Iterators!\texttt{set}!\texttt{const\_iterator}}*)
for (c = letters.begin(); c != letters.end(); c++)
    cout << *c << endl;
\end{lstlisting}
\begin{enumerate}
\item[$\Rightarrow$] Associative containers: \texttt{map}, ordered according to its key in pairs (key,value)\index{Containers!associative|see{\texttt{map}}}
\end{enumerate}
\begin{lstlisting}
#include <map>(*\index{\texttt{<map>}}*)
#include <utility>(*\index{\texttt{<utility>}}*)
#include <string>(*\index{\texttt{<string>}}*)

// initialization
map<string,int> dict = { {"one",1}, {"two",2} };(*\index{\texttt{map}|see{Containers}}*)(*\index{Containers!\texttt{map}}*)(*\index{Strings!\texttt{string}}*)
pair<string,int> three("three",3);(*\index{\texttt{pair}}*)(*\index{Strings!\texttt{string}}*)

// insertion    
dict.insert(three);(*\index{Containers!\texttt{map}!\texttt{insert}}*)
dict["four"] = 4;
dict["five"] = 5;

// (*make\_pair*)
pair<string,int> six;
six = make_pair<string,int>("six",6);(*\index{\texttt{make\_pair}}*)
dict.insert(six);

// iterator    
map<string,int>::iterator two;(*\index{Strings!\texttt{string}}*)(*\index{Iterators!\texttt{map}!\texttt{iterator}}*)

// find    
two = dict.find("two");(*\index{Containers!\texttt{map}!\texttt{find}}*)

// erase    
dict.erase(two);(*\index{Containers!\texttt{map}!\texttt{erase}}*)

// range-for-loop(*\index{range-for-loop}*)
// inside the loop (*n*) is a (*pair*)
// the key is (*n.first*) and the value is (*n.second*)
for (auto n : dict) {(*\index{\textbf{auto}}*)
    cout << "(" <<  n.first << "," <<  n.second << ")";
    cout << endl;
}
\end{lstlisting}
\begin{enumerate}
\item[$\Rightarrow$] Associative containers: \texttt{multimap}, ordered according to its key in pairs (key,value), keys can be repeated\index{Containers!associative|see{\texttt{multimap}}}
\end{enumerate}
\begin{lstlisting}
#include <map>(*\index{\texttt{<map>}}*)
#include <utility>(*\index{\texttt{<utility>}}*)
#include <string>(*\index{\texttt{<string>}}*)

multimap<string,int> mm;(*\index{\texttt{multimap}|see{Containers}}*)(*\index{Containers!\texttt{multimap}}*)
 
// insert
mm.insert(make_pair<string,int>("Mary",1));(*\index{Containers!\texttt{multimap}!\texttt{insert}}*)
mm.insert(make_pair<string,int>("Dick",6));
mm.insert(make_pair<string,int>("Mary",7));
mm.insert(make_pair<string,int>("John",1));
mm.insert(make_pair<string,int>("Mary",4));

// search for (*Mary*), returns a pair of iterators
auto pp = mm.equal_range("Mary");(*\index{Containers!\texttt{multimap}!\texttt{equal\_range}}*)

// prints out (*Mary : 1, Mary : 7, Mary : 4*)
for (auto p = pp.first; p !=pp.second; ++p)
    cout << p->first << " : " << p->second << ", ";
\end{lstlisting}
%
% Algorithms
%
\section{Algorithms}
\small
Provided by the C++ standard template library (STL)\index{STL}.
\begin{enumerate}
\item[$\Rightarrow$]  \texttt{find}
\end{enumerate}
\begin{lstlisting}
#include <algorithm>(*\index{\texttt{<algorithm>}}*)
#include <vector>(*\index{\texttt{<vector>}}*)

vector<int> v = {6,2,7,13,4,3,1};(*\index{\texttt{vector}|see{Containers}}*)(*\index{Containers!\texttt{vector}}*)
vector<int>::iterator p;(*\index{Iterators!\texttt{vector}!\texttt{iterator}}*)

// (*find*)(*\index{Algorithms!\texttt{find}}*)
// points to the first occurrence of 13 in (*v*)
p = find(v.begin(),v.end(),13);(*\index{Algorithms!\texttt{find}}*)
\end{lstlisting}
\begin{enumerate}
\item[$\Rightarrow$]  \texttt{find\_if}
\end{enumerate}
\begin{lstlisting}
bool test_greater_than_5(int x) { return x>5; }
// (*find\_if*)(*\index{Algorithms!\texttt{find\_if}}*)(*\index{\texttt{find\_if}|see{Algorithms}}*)
// general search, stops as soon as the predicate is
// satisfied points to the first occurrence of an
// element greater than 5 in (*v*)
p = find_if(v.begin(),v.end(),test_greater_than_5);(*\index{Algorithms!\texttt{find\_if}}*)
\end{lstlisting}
\begin{enumerate}
\item[$\Rightarrow$] Function object
\end{enumerate}
\begin{lstlisting}
// function object(*\index{function object}*)
class TestGreater(*\index{Classes!\textbf{class}!\texttt{TestGreater}}*)
{
public:
    TestGreater(int x) : n{x} {}
    bool operator() (const int x) const { return x>n; }(*\index{Constants!\textbf{const}}*)
private:
    int n{0};
};

// (*find\_if*)(*\index{Algorithms!\texttt{find\_if}}*)
// general search, stops as soon as the predicate is
// satisfied points to the first occurrence of an
// element greater than 7 in (*v*)
p = find_if(v.begin(),v.end(),TestGreater(7));(*\index{Algorithms!\texttt{find\_if}}*)
\end{lstlisting}
\begin{enumerate}
\item[$\Rightarrow$] \texttt{accumulate}
\end{enumerate}
\begin{lstlisting}
#include <algorithm>(*\index{\texttt{<algorithm>}}*)
#include <vector>(*\index{\texttt{<vector>}}*)
#include <list>(*\index{\texttt{<list>}}*)
#include <set>(*\index{\texttt{<set>}}*)
#include <numeric>(*\index{\texttt{<numeric>}}*)
#include <functional>(*\index{\texttt{<functional>}}*)

// (*accumulate*)
// adds the values from a sequence
// the last parameter is the initial value
// the return type is the type of the initial value!
long res = accumulate(v.begin(),v.end(),long(0));(*\index{Algorithms!\texttt{accumulate}}*)(*\index{\texttt{accumulate}|see{Algorithms}}*)

// generic accumulate performing multiplication instead(*\index{Algorithms!generic \texttt{accumulate}}*)
// of sum passes a function object
// (*multiplies<int>()*) defined in (*<functional>*)(*\index{\texttt{multiplies<int>()}}*)
double res = accumulate(v.begin(),v.end(),
    1.0,multiplies<int>());
\end{lstlisting}
\begin{enumerate}
\item[$\Rightarrow$]  Inner product
\end{enumerate}
\begin{lstlisting}
// (*inner product*)
vector<int> v1 = {-2,2,4,11,-4,3,1};(*\index{Containers!\texttt{vector}}*)
vector<int> v2 = {4,2,4,18,5,3,1,7,9,};(*\index{Containers!\texttt{vector}}*)

long res = inner_product(v1.begin(), v1.end(),
    v2.begin(),long(0));(*\index{Algorithms!\texttt{inner\_product}}*)(*\index{\texttt{inner\_product}|see{Algorithms}}*)
\end{lstlisting}
\begin{enumerate}
\item[$\Rightarrow$]  \texttt{copy} between different containers
\end{enumerate}
\begin{lstlisting}
list<double> data = {1.32,-2.45,5.65};\index{Containers!\texttt{list}}
vector<int> int_data(5);\index{Containers!\texttt{vector}}

// copy data from the (*list*) of doubles to the (*vector*)
// of integers
copy(data.begin(),data.end(),int_data.begin());(*\index{Algorithms!\texttt{copy}!between different containers}*)
\end{lstlisting}
\begin{enumerate}
\item[$\Rightarrow$]  \texttt{copy} between a container and an output stream. A container can be initialized by the elements retrieved via a pair of iterators.
\end{enumerate}
\begin{lstlisting}
// from the input character stream (*cin*) elements are 
// read as strings and used to populate a set of (*words*)

// a (*set*) doesn't allow any duplicates and keeps
// elements ordered
set<string> words{ istream_iterator<string>{cin},
                             istream_iterator<string>{} };(*\index{Iterators!initialization with a pair of iterators}*)(*\index{\texttt{istream\_iterator}}*)
                             
// copy the words from the set to the output stream (*cout*)
// and add a new line after each word
copy(words.begin(),words.end(),
    ostream_iterator<string>{cout,"\n"});(*\index{Algorithms!\texttt{copy}!between container and stream}*)(*\index{\texttt{ostream\_iterator}}*)
\end{lstlisting}
\begin{enumerate}
\item[$\Rightarrow$] Merge sort
\end{enumerate}
\begin{lstlisting}
// merge sort
sort(v.begin(),v.end());(*\index{Algorithms!\texttt{sort}}*)(*\index{\texttt{sort}|see{Algorithms}}*)
\end{lstlisting}
\begin{enumerate}
\item[$\Rightarrow$] Binary search
\end{enumerate}
\begin{lstlisting}
// binary search
bool found;
found = binary_search(v.begin(), v.end(), 3);(*\index{Algorithms!\texttt{binary\_search}}*)(*\index{\texttt{binary\_search}|see{Algorithms}}*)
\end{lstlisting}
\begin{enumerate}
\item[$\Rightarrow$] \texttt{reverse}
\end{enumerate}
\begin{lstlisting}
// reverse
reverse(v.begin(),v.end());(*\index{Algorithms!\texttt{reverse}}*)(*\index{\texttt{reverse}|see{Algorithms}}*)
\end{lstlisting}
%
% Utilities
%
\section{Utilities}
\begin{enumerate}
\item[$\Rightarrow$] Random numbers
\end{enumerate}
\begin{enumerate}
\item[] C-style:
\end{enumerate}
\begin{lstlisting}
#include <cstdlib>(*\index{\texttt{<cstdlib>}}*)
#include <ctime>(*\index{\texttt{<ctime>}}*)

// seed the generator(*\index{Utilities!Random numbers!seed the generator}*)
srand( time(0) );(*\index{\texttt{srand}}*)(*\index{\texttt{time}}*)

// integer random number between (*0*) and (*RAND\_MAX*)(*\index{Utilities!Random numbers!integer random number}*)
int n = rand();(*\index{\texttt{rand}}*)
\end{lstlisting}
\begin{enumerate}
\item[] C++ style:
\end{enumerate}
\begin{lstlisting}
#include <random>(*\index{\texttt{<random>}}*)
#include <functional>(*\index{\texttt{<functional>}}*)

// (*bind*) constructs a function object that calls its
// first argument with its second argument

// normal distribution by using the default random engine
auto gen_def = bind(normal_distribution<double>{15,4.0},(*\index{\texttt{bind}}*)(*\index{Utilities!Random numbers!normal distribution}*)
                    default_random_engine{});(*\index{Utilities!Random numbers!default random engine}*)(*\index{\texttt{default\_random\_engine}}*)
                    
// normal distribution by using the Mersenne Twister
// engine with seed (*91586*)                  
auto gen_mt = bind(normal_distribution<double>{15,4.0},(*\index{\texttt{bind}}*)(*\index{Utilities!Random numbers!normal distribution}*)(*\index{\texttt{normal\_distribution}}*)
                    mt19937_64{91586});(*\index{Utilities!Random numbers!Mersenne Twister engine}*)(*\index{\texttt{mt19937\_64}}*)

// call the function objects to get random numbers
cout << gen_def() << gen_mt() << endl;(*\index{Input-output streams!\texttt{cout}}*)
\end{lstlisting}
\begin{enumerate}
\item[$\Rightarrow$] \texttt{bitset}
\index{range-for-loop}
\end{enumerate}
\begin{enumerate}
\item[] Bits and bit operations:\index{Utilities!\texttt{bitset}!bits and bit operations}
\index{range-for-loop}
\end{enumerate}
\begin{lstlisting}
#include <bitset>(*\index{\texttt{<bitset>}}*)

bitset<8> a{87};   // (*01010111*)(*\index{\texttt{bitset<N>}}*)
bitset<8> b{0x87}; // (*10000111*)
    
cout << a << endl << b << endl;
    
// boolean and
bitset<8> c = a & b;
    
cout << c << endl; // (*00000111*)
\end{lstlisting}
\begin{enumerate}
\item[] Access to single bits:\index{Utilities!\texttt{bitset}!access to single bits}
\index{range-for-loop}
\end{enumerate}
\begin{lstlisting}
// prints out single bits reverting the order
// (*11100000*)
for (auto i=0; i<8; ++i)
     cout << c[i];
\end{lstlisting}
\begin{enumerate}
\item[$\Rightarrow$] Chrono\index{Utilities!Chrono}
\end{enumerate}
\begin{enumerate}
\item[] Run-time measurement in milliseconds:\index{Run-time measurement|see{Utilities}}\index{Utilities!Chrono!run-time measurement}
\end{enumerate}
\begin{lstlisting}
#include <chrono>(*\index{\texttt{<chrono>}}*)

using namespace std::chrono;

// returns a value of type (*time\_point<system\_clock>*)(*\index{\texttt{time\_point<system\_clock>}}*)
auto t1 = system_clock::now();(*\index{\texttt{system\_clock::now()}}*)

// ...

// returns a value of type (*time\_point<system\_clock>*)(*\index{\texttt{time\_point<system\_clock>}}*)
auto t2 = system_clock::now();(*\index{\texttt{system\_clock::now()}}*)

// run-time in milliseconds
cout << duration_cast<milliseconds>(t2-t1).count();(*\index{\texttt{duration\_cast<milliseconds>}}*)
\end{lstlisting}
\begin{enumerate}
\item[] Sleep for a certain amount of time:\index{Sleep|see{Utilities}}\index{Utilities!Chrono!sleep}
\end{enumerate}
\begin{lstlisting}
#include <thread>(*\index{\texttt{<thread>}}*)
#include <chrono>(*\index{\texttt{<chrono>}}*)

using namespace std::chrono;(*\index{\texttt{chrono}}*)(*\index{Namespaces!\textbf{using namespace} directives}*)
using namespace std::literals::chrono_literals;(*\index{\texttt{chrono\_literals}}*)(*\index{Namespaces!\textbf{using namespace} directives}*)

// returns a value of type (*time\_point<system\_clock>*)(*\index{\texttt{time\_point<system\_clock>}}*)
auto start = system_clock::now();(*\index{\texttt{system\_clock::now()}}*)

// pause thread for (*100*) ms
this_thread::sleep_for(100ms);(*\index{\texttt{this\_thread::sleep\_for()}}*)

// returns a value of type (*time\_point<system\_clock>*)(*\index{\texttt{time\_point<system\_clock>}}*)
auto end = system_clock::now();(*\index{\texttt{system\_clock::now()}}*)

// sleep time in milliseconds
cout << duration_cast<milliseconds>(end-start).count();(*\index{\texttt{duration\_cast<milliseconds>}}*)
\end{lstlisting}
\begin{enumerate}
\item[] Example stopwatch implementation for timing code execution in a given scope:\index{Utilities!Chrono!stopwatch}
\end{enumerate}
\begin{lstlisting}
#include <chrono>(*\index{\texttt{<chrono>}}*)

class Stopwatch {
public:
    Stopwatch(nanoseconds& total_time) : 
        result { total_time },
        start {high_resolution_clock::now()} {}
    ~Stopwatch() {
        result = duration_cast<nanoseconds>(
                     high_resolution_clock::now()
                     - start);
     }
private:
    nanoseconds& result;
    const time_point<high_resolution_clock> start;(*\index{\texttt{time\_point<high\_resolution\_clock>}}*)
};

// code portion to be measured
nanoseconds total_time{0};
{
    Stopwatch elapsed{ total_time };
    this_thread::sleep_for(100ms);
}

// number of nanosecond ticks
cout << "total_time.count() = " << total_time.count();
\end{lstlisting}
\begin{enumerate}
\item[$\Rightarrow$] \texttt{pair}\\ \\ Class template containing 2 objects of different types:\index{Utilities!\texttt{pair}}
\end{enumerate}
\begin{lstlisting}
#include <utilities>(*\index{\texttt{<utilities>}}*)

// temperature pairs
pair<float,string> temp1{ 23.4, " degrees C" }{(*\index{\texttt{pair}}*)

pair<float,string> temp2; 
temp2 = make_pair<float,string>( 21.7," degrees C" );(*\index{\texttt{pair}}*)(*\index{\texttt{make\_pair}}*)

// prints temperatures
cout << temp1.first << temp1.second << endl;
cout << temp2.first << temp2.second << endl;
\end{lstlisting}
\begin{enumerate}
\item[$\Rightarrow$] Regular expressions\index{Utilities!Regular expressions}
\end{enumerate}
\begin{enumerate}
\item[] First match:\index{Utilities!Regular expressions!first match}
\end{enumerate}
\begin{lstlisting}
#include <regex>(*\index{\texttt{<regex>}}*)
#include <string>(*\index{\texttt{<string>}}*)

// regular expression to look for
// in raw string format 
regex regex{ R"((\w{2})(\d{5})(-\d{4})?)" };(*\index{\texttt{regex}}*)(*\index{Strings!raw format}*)
smatch results;
string zip("The string NJ07936-3173 and NJ07936-3175 are ZIP codes");

// first match
bool found = regex_search(zip,results,regex);(*\index{\texttt{regex\_search}}*)

// this prints position (*11*)
if (found) {
    cout << "First match at position: ";
    cout << results.position(0) << endl;
}
\end{lstlisting}
\begin{enumerate}
\item[] Iterative search:\index{Utilities!Regular expressions!iterative search}
\end{enumerate}
\begin{lstlisting}
// regular expression to look for
// in raw string format 
regex regex{ R"((\w{2})(\d{5})(-\d{4})?)" };(*\index{\texttt{regex}}*)(*\index{Strings!raw format}*)
smatch results;
string zip("The string NJ07936-3173 and NJ07936-3175 are ZIP codes");

// start from the beginning
string::const_iterator start( zip.cbegin() );(*\index{Iterators!\texttt{string}!\texttt{const\_iterator}}*)

while ( regex_search( start, zip.cend(), results, regex ) )(*\index{\texttt{regex\_search}}*)
 {
     // match has been found
     cout << ( start == zip.cbegin() ? "" : " " );
     cout << results[0];
     
     // continue searching from the suffix sequence
     // after the match
     start = results.suffix().first;
 }
 cout << endl;
\end{lstlisting}
\begin{enumerate}
\item[] Replace:\index{Utilities!Regular expressions!replace}
\end{enumerate}
\begin{lstlisting}
#include <regex>(*\index{\texttt{<regex>}}*)
#include <string>(*\index{\texttt{<string>}}*)

// regular expression to look for
// in raw string format 
regex regex{ R"((\w{2})(\d{5})(-\d{4})?)" };(*\index{\texttt{regex}}*)(*\index{Strings!raw format}*)
string zip("The string NJ07936-3173 and NJ07936-3175 are ZIP codes");

// replaces the ZIP codes with (*XXX*)
string zip_hidden = regex_replace(zip,regex,"XXX");(*\index{\texttt{regex\_replace}}*)
cout << zip_hidden << endl;
\end{lstlisting}
\begin{enumerate}
\item[$\Rightarrow$] Filesystem\index{Utilities!Filesystem}\\ \\ Declaring a \texttt{path} object:
\end{enumerate}
\begin{lstlisting}
#include <filesystem>(*\index{\texttt{<filesystem>}}*)

using namespace std::filesystem;

// path object
path current_dir;(*\index{\texttt{path}}*)(*\index{Utilities!Filesystem!paths}*)
\end{lstlisting}
\begin{enumerate}
\item[] Retrieving the current folder:
\end{enumerate}
\begin{lstlisting}    
// stores the program current directory
current_dir = current_path();(*\index{\texttt{current\_path}}*)(*\index{Utilities!Filesystem!current folder}*)
\end{lstlisting}
\begin{enumerate}
\item[] Declaring a directory iterator:
\end{enumerate}
\begin{lstlisting}    
// directory iterator
directory_iterator dir{current_dir};(*\index{\texttt{directory\_iterator}}*)(*\index{Utilities!Filesystem!directory iterator}*)
\end{lstlisting}
\begin{enumerate}
\item[] Loop over directory elements:
\end{enumerate}
\begin{lstlisting}    
// loop over the current directory entries
for (auto& dir_entry : dir) {
    if ( dir_entry.is_directory() ) {
        cout << "Found directory: ";
        cout << dir_entry.path().filename() << endl;(*\index{Utilities!Filesystem!directory entry file name}*)
    }
    else if ( dir_entry.is_regular_file() ) {
        cout << "Found a file: ";
        cout << dir_entry.path().filename() << ", ";
        cout << dir_entry.file_size() << " bytes big " << endl;(*\index{Utilities!Filesystem!directory entry file size}*)
    }
}
\end{lstlisting}
\begin{enumerate}
\item[] Copy a file:
\end{enumerate}
\begin{lstlisting}
// path target object
path target_file{source_file};
target_file.replace_filename("Mycopy");
    
// copy file
copy_file(source_file,target_file);(*\index{\texttt{copy\_file}}*)(*\index{Utilities!Filesystem!copy file}*)
\end{lstlisting}
\begin{enumerate}
\item[] Read a file and print out its content:
\end{enumerate}
\begin{lstlisting}
// open the file
ifstream input_file{target_file};(*\index{Utilities!Filesystem!read file}*)
    
// read the file and print it out
char c; 
while (input_file >> c)
    cout << c;
cout << endl;
    
// close file
input_file.close();
\end{lstlisting}
\begin{enumerate}
\item[] Change permission of a file:
\end{enumerate}
\begin{lstlisting}
// removes permissions for group and others
permissions( "Mycopy", perms::group_all | perms::others_all,
    perm_options::remove);(*\index{Utilities!Filesystem!change permissions}*)
\end{lstlisting}
\begin{enumerate}
\item[] Delete a file:
\end{enumerate}
\begin{lstlisting}
// delete the file
remove(target_file);(*\index{\texttt{remove}}*)(*\index{Utilities!Filesystem!delete file}*)
\end{lstlisting}
\begin{enumerate}
\item[$\Rightarrow$] Threads\index{Utilities!Threads}\\ \\ Defining a function to be run in a separate thread:
\end{enumerate}
\begin{lstlisting}
#include <future>(*\index{\texttt{<future>}}*)
#include <thread>(*\index{\texttt{<thread>}}*)
#include <chrono>(*\index{\texttt{<chrono>}}*)
#include <vector>

using namespace std::chrono;
using namespace std::literals::chrono_literals;

// sleeps for (*5s*)
int sleeping_thread(int n) {
    cout << "Sleeping thread #" << n << " started...";
    cout << endl;
    // pause thread for 5s
    this_thread::sleep_for(5s);(*\index{\texttt{this\_thread::sleep\_for()}}*)
    cout << "Sleeping thread ended..." << endl;
    
    return 0;
}
\end{lstlisting}
\begin{enumerate}
\item[] Start an asynchronous thread and wait for its result:\index{Utilities!Threads!asynchronous task}
\end{enumerate}
\begin{lstlisting}
// start the sleeping thread with parameter (*1*)
auto sleeping_res =
    async(launch::async,sleeping_thread,1);(*\index{\texttt{async}}*)
    
// blocks until the thread finishes
if (sleeping_res.get() == 0)(*\index{Utilities!Threads!\texttt{get}}*)
    cout << "Sleeping thread #1 finished! " << endl;
\end{lstlisting}
\begin{enumerate}
\item[] Start an asynchronous thread and check periodically until result is available:\index{Utilities!Threads!asynchronous task}
\end{enumerate}
\begin{lstlisting}
// start the sleeping thread
sleeping_res = async(launch::async,sleeping_thread,1);
    
cout << "Checking on sleeping thread..." << endl;
auto sleeping_stat = sleeping_res.wait_for(1s);(*\index{Utilities!Threads!\texttt{wait\_for}}*)
while (sleeping_stat != future_status::ready) {(*\index{\texttt{future\_status::ready}}*)
    cout << "Checking on sleeping thread..." << endl;
    sleeping_stat = sleeping_res.wait_for(1s);
}

if (sleeping_stat == future_status::ready)
    cout << "Sleeping thread #1 finished! " << endl;
\end{lstlisting}
\begin{enumerate}
\item[] Usage of a mutex for synchronizing threads:\index{Utilities!Threads!mutex}\index{Mutex}
\end{enumerate}
\begin{lstlisting}
#include <mutex>(*\index{\texttt{<mutex>}}*)

// global counter to be incremented
int counter=0;

// mutex for synchronizing the access
mutex m_counter;(*\index{\texttt{mutex}}*)

// incrementing thread
template<char c>
int incrementing_thread(int n) {
    // enter infinite loop
    while (1) {
        // blocking call for getting the mutex
        m_counter.lock();(*\index{Utilities!Threads!lock mutex}*)
        // got the mutex
        // check current counter value and exit if needed
        if (counter >= n)
            break;
        cout << "incrementing_thread<" << c << ">(): ";
        cout << counter << endl;
        counter++;
        
        // release the mutex
        m_counter.unlock();(*\index{Utilities!Threads!unlock mutex}*)
        
        // yield thread execution
        this_thread::yield();(*\index{\texttt{this\_thread::yield}}*)(*\index{Utilities!Threads!yield execution}*)
    }
    cout << "incrementing_thread<" << c;
    cout << ">(): done" << endl;
    
    // release the mutex
    m_counter.unlock();
    
    return n;
}

// start the incrementing threads
auto one_res = async(launch::async,incrementing_thread<'1'>,6);
auto two_res = async(launch::async,incrementing_thread<'2'>,9);

if (one_res.get() == 6)
    cout << "Incrementing thread #1 finished! " << endl;
    
if (two_res.get() == 9)
    cout << "Incrementing thread #2 finished! " << endl;
    
cout << "Counter value: " << counter << endl;
\end{lstlisting}
\begin{enumerate}
\item[] Usage of atomic variables:\index{Utilities!Threads!atomic}\index{Atomic variables}
\end{enumerate}
\begin{lstlisting}
#include <atomic>(*\index{\texttt{<atomic>}}*)

// atomic counter
atomic_int counter_atomic = 0;(*\index{\texttt{atomic\_int}}*)

template<char c>
int atomic_incrementing_thread(int n) {
    while (1) {
        cout << "atomic_incrementing_thread<";
        cout << c << ">(): " 
        cout << counter_atomic << endl;
        
        // atomic read and increment
        if (counter_atomic < n)
            counter_atomic++;
        else break;
        
         // yield thread execution
        this_thread::yield();(*\index{\texttt{this\_thread::yield}}*)(*\index{Utilities!Threads!yield execution}*)
    }
    cout << "atomic_incrementing_thread<";
    cout << c << ">(): done" << endl;
    
    return n;
}

// start the incrementing threads
auto one_res = async(launch::async, atomic_incrementing_thread<'1'>,6);
auto two_res = async(launch::async, atomic_incrementing_thread<'2'>,9);

if (one_res.get() == 6)
    cout << "Incrementing thread #1 finished! " << endl;
    
if (two_res.get() == 9)
    cout << "Incrementing thread #2 finished! " << endl;
    
cout << "Counter value: " << counter_atomic << endl;
\end{lstlisting}
%
% Bibliography
%
\newpage

\renewcommand\refname{Bibliography}
\addcontentsline{toc}{chapter}{Bibliography}
\begin{thebibliography}{99}
\bibitem{savitch} Walter Savitch. \textsl{Problem Solving with C++}, 10th edition. Pearson Education, 2018
\bibitem{stroustrup} Bjarne Stroustrup. \textsl{Programming: Principles and Practice Using C++}, 2nd edition. Addison Wesley, 2015
\bibitem{lospinoso} Josh Lospinoso. \textsl{C++ Crash Course: A Fast-Paced Introduction}, 1st edition. No Starch Press, 2019
\bibitem{Seacord} Robert C. Seacord. \textsl{Secure Coding in C and C++}, 2nd edition. Addison Wesley, 2013
\bibitem{isoiec} ISO/IEC. \textsl{Programming languages -- C++}, Sixth edition, ISO/IEC 14882:2020(E)
\end{thebibliography}
%
% Appendix
%
\appendix
\chapter{GNU GENERAL PUBLIC LICENSE}
\begin{center}
Version 3, 29 June 2007
\end{center}
\begin{center}
{\parindent 0in

Copyright \copyright\  2007 Free Software Foundation, Inc. \texttt{https://fsf.org/}

\bigskip
Everyone is permitted to copy and distribute verbatim copies of this

license document, but changing it is not allowed.}

\end{center}

\begin{center}
{\Large \sc Preamble}
\end{center}
The GNU General Public License is a free, copyleft license for
software and other kinds of works.

The licenses for most software and other practical works are designed
to take away your freedom to share and change the works.  By contrast,
the GNU General Public License is intended to guarantee your freedom to
share and change all versions of a program--to make sure it remains free
software for all its users.  We, the Free Software Foundation, use the
GNU General Public License for most of our software; it applies also to
any other work released this way by its authors.  You can apply it to
your programs, too.

When we speak of free software, we are referring to freedom, not
price.  Our General Public Licenses are designed to make sure that you
have the freedom to distribute copies of free software (and charge for
them if you wish), that you receive source code or can get it if you
want it, that you can change the software or use pieces of it in new
free programs, and that you know you can do these things.

To protect your rights, we need to prevent others from denying you
these rights or asking you to surrender the rights.  Therefore, you have
certain responsibilities if you distribute copies of the software, or if
you modify it: responsibilities to respect the freedom of others.

For example, if you distribute copies of such a program, whether
gratis or for a fee, you must pass on to the recipients the same
freedoms that you received.  You must make sure that they, too, receive
or can get the source code.  And you must show them these terms so they
know their rights.

Developers that use the GNU GPL protect your rights with two steps:
(1) assert copyright on the software, and (2) offer you this License
giving you legal permission to copy, distribute and/or modify it.

For the developers' and authors' protection, the GPL clearly explains
that there is no warranty for this free software.  For both users' and
authors' sake, the GPL requires that modified versions be marked as
changed, so that their problems will not be attributed erroneously to
authors of previous versions.

Some devices are designed to deny users access to install or run
modified versions of the software inside them, although the manufacturer
can do so.  This is fundamentally incompatible with the aim of
protecting users' freedom to change the software.  The systematic
pattern of such abuse occurs in the area of products for individuals to
use, which is precisely where it is most unacceptable.  Therefore, we
have designed this version of the GPL to prohibit the practice for those
products.  If such problems arise substantially in other domains, we
stand ready to extend this provision to those domains in future versions
of the GPL, as needed to protect the freedom of users.

Finally, every program is threatened constantly by software patents.
States should not allow patents to restrict development and use of
software on general-purpose computers, but in those that do, we wish to
avoid the special danger that patents applied to a free program could
make it effectively proprietary.  To prevent this, the GPL assures that
patents cannot be used to render the program non-free.

The precise terms and conditions for copying, distribution and
modification follow.

\begin{center}
{\Large \sc Terms and Conditions}
\end{center}


\begin{enumerate}

\addtocounter{enumi}{-1}

\item Definitions.

``This License'' refers to version 3 of the GNU General Public License.

``Copyright'' also means copyright-like laws that apply to other kinds of
works, such as semiconductor masks.

``The Program'' refers to any copyrightable work licensed under this
License.  Each licensee is addressed as ``you''.  ``Licensees'' and
``recipients'' may be individuals or organizations.

To ``modify'' a work means to copy from or adapt all or part of the work
in a fashion requiring copyright permission, other than the making of an
exact copy.  The resulting work is called a ``modified version'' of the
earlier work or a work ``based on'' the earlier work.

A ``covered work'' means either the unmodified Program or a work based
on the Program.

To ``propagate'' a work means to do anything with it that, without
permission, would make you directly or secondarily liable for
infringement under applicable copyright law, except executing it on a
computer or modifying a private copy.  Propagation includes copying,
distribution (with or without modification), making available to the
public, and in some countries other activities as well.

To ``convey'' a work means any kind of propagation that enables other
parties to make or receive copies.  Mere interaction with a user through
a computer network, with no transfer of a copy, is not conveying.

An interactive user interface displays ``Appropriate Legal Notices''
to the extent that it includes a convenient and prominently visible
feature that (1) displays an appropriate copyright notice, and (2)
tells the user that there is no warranty for the work (except to the
extent that warranties are provided), that licensees may convey the
work under this License, and how to view a copy of this License.  If
the interface presents a list of user commands or options, such as a
menu, a prominent item in the list meets this criterion.

\item Source Code.

The ``source code'' for a work means the preferred form of the work
for making modifications to it.  ``Object code'' means any non-source
form of a work.

A ``Standard Interface'' means an interface that either is an official
standard defined by a recognized standards body, or, in the case of
interfaces specified for a particular programming language, one that
is widely used among developers working in that language.

The ``System Libraries'' of an executable work include anything, other
than the work as a whole, that (a) is included in the normal form of
packaging a Major Component, but which is not part of that Major
Component, and (b) serves only to enable use of the work with that
Major Component, or to implement a Standard Interface for which an
implementation is available to the public in source code form.  A
``Major Component'', in this context, means a major essential component
(kernel, window system, and so on) of the specific operating system
(if any) on which the executable work runs, or a compiler used to
produce the work, or an object code interpreter used to run it.

The ``Corresponding Source'' for a work in object code form means all
the source code needed to generate, install, and (for an executable
work) run the object code and to modify the work, including scripts to
control those activities.  However, it does not include the work's
System Libraries, or general-purpose tools or generally available free
programs which are used unmodified in performing those activities but
which are not part of the work.  For example, Corresponding Source
includes interface definition files associated with source files for
the work, and the source code for shared libraries and dynamically
linked subprograms that the work is specifically designed to require,
such as by intimate data communication or control flow between those
subprograms and other parts of the work.

The Corresponding Source need not include anything that users
can regenerate automatically from other parts of the Corresponding
Source.

The Corresponding Source for a work in source code form is that
same work.

\item Basic Permissions.

All rights granted under this License are granted for the term of
copyright on the Program, and are irrevocable provided the stated
conditions are met.  This License explicitly affirms your unlimited
permission to run the unmodified Program.  The output from running a
covered work is covered by this License only if the output, given its
content, constitutes a covered work.  This License acknowledges your
rights of fair use or other equivalent, as provided by copyright law.

You may make, run and propagate covered works that you do not
convey, without conditions so long as your license otherwise remains
in force.  You may convey covered works to others for the sole purpose
of having them make modifications exclusively for you, or provide you
with facilities for running those works, provided that you comply with
the terms of this License in conveying all material for which you do
not control copyright.  Those thus making or running the covered works
for you must do so exclusively on your behalf, under your direction
and control, on terms that prohibit them from making any copies of
your copyrighted material outside their relationship with you.

Conveying under any other circumstances is permitted solely under
the conditions stated below.  Sublicensing is not allowed; section 10
makes it unnecessary.

\item Protecting Users' Legal Rights From Anti-Circumvention Law.

No covered work shall be deemed part of an effective technological
measure under any applicable law fulfilling obligations under article
11 of the WIPO copyright treaty adopted on 20 December 1996, or
similar laws prohibiting or restricting circumvention of such
measures.

When you convey a covered work, you waive any legal power to forbid
circumvention of technological measures to the extent such circumvention
is effected by exercising rights under this License with respect to
the covered work, and you disclaim any intention to limit operation or
modification of the work as a means of enforcing, against the work's
users, your or third parties' legal rights to forbid circumvention of
technological measures.

\item Conveying Verbatim Copies.

You may convey verbatim copies of the Program's source code as you
receive it, in any medium, provided that you conspicuously and
appropriately publish on each copy an appropriate copyright notice;
keep intact all notices stating that this License and any
non-permissive terms added in accord with section 7 apply to the code;
keep intact all notices of the absence of any warranty; and give all
recipients a copy of this License along with the Program.

You may charge any price or no price for each copy that you convey,
and you may offer support or warranty protection for a fee.

\item Conveying Modified Source Versions.

You may convey a work based on the Program, or the modifications to
produce it from the Program, in the form of source code under the
terms of section 4, provided that you also meet all of these conditions:
  \begin{enumerate}
  \item The work must carry prominent notices stating that you modified
  it, and giving a relevant date.

  \item The work must carry prominent notices stating that it is
  released under this License and any conditions added under section
  7.  This requirement modifies the requirement in section 4 to
  ``keep intact all notices''.

  \item You must license the entire work, as a whole, under this
  License to anyone who comes into possession of a copy.  This
  License will therefore apply, along with any applicable section 7
  additional terms, to the whole of the work, and all its parts,
  regardless of how they are packaged.  This License gives no
  permission to license the work in any other way, but it does not
  invalidate such permission if you have separately received it.

  \item If the work has interactive user interfaces, each must display
  Appropriate Legal Notices; however, if the Program has interactive
  interfaces that do not display Appropriate Legal Notices, your
  work need not make them do so.
\end{enumerate}
A compilation of a covered work with other separate and independent
works, which are not by their nature extensions of the covered work,
and which are not combined with it such as to form a larger program,
in or on a volume of a storage or distribution medium, is called an
``aggregate'' if the compilation and its resulting copyright are not
used to limit the access or legal rights of the compilation's users
beyond what the individual works permit.  Inclusion of a covered work
in an aggregate does not cause this License to apply to the other
parts of the aggregate.

\item Conveying Non-Source Forms.

You may convey a covered work in object code form under the terms
of sections 4 and 5, provided that you also convey the
machine-readable Corresponding Source under the terms of this License,
in one of these ways:
  \begin{enumerate}
  \item Convey the object code in, or embodied in, a physical product
  (including a physical distribution medium), accompanied by the
  Corresponding Source fixed on a durable physical medium
  customarily used for software interchange.

  \item Convey the object code in, or embodied in, a physical product
  (including a physical distribution medium), accompanied by a
  written offer, valid for at least three years and valid for as
  long as you offer spare parts or customer support for that product
  model, to give anyone who possesses the object code either (1) a
  copy of the Corresponding Source for all the software in the
  product that is covered by this License, on a durable physical
  medium customarily used for software interchange, for a price no
  more than your reasonable cost of physically performing this
  conveying of source, or (2) access to copy the
  Corresponding Source from a network server at no charge.

  \item Convey individual copies of the object code with a copy of the
  written offer to provide the Corresponding Source.  This
  alternative is allowed only occasionally and noncommercially, and
  only if you received the object code with such an offer, in accord
  with subsection 6b.

  \item Convey the object code by offering access from a designated
  place (gratis or for a charge), and offer equivalent access to the
  Corresponding Source in the same way through the same place at no
  further charge.  You need not require recipients to copy the
  Corresponding Source along with the object code.  If the place to
  copy the object code is a network server, the Corresponding Source
  may be on a different server (operated by you or a third party)
  that supports equivalent copying facilities, provided you maintain
  clear directions next to the object code saying where to find the
  Corresponding Source.  Regardless of what server hosts the
  Corresponding Source, you remain obligated to ensure that it is
  available for as long as needed to satisfy these requirements.

  \item Convey the object code using peer-to-peer transmission, provided
  you inform other peers where the object code and Corresponding
  Source of the work are being offered to the general public at no
  charge under subsection 6d.
  \end{enumerate}

A separable portion of the object code, whose source code is excluded
from the Corresponding Source as a System Library, need not be
included in conveying the object code work.

A ``User Product'' is either (1) a ``consumer product'', which means any
tangible personal property which is normally used for personal, family,
or household purposes, or (2) anything designed or sold for incorporation
into a dwelling.  In determining whether a product is a consumer product,
doubtful cases shall be resolved in favor of coverage.  For a particular
product received by a particular user, ``normally used'' refers to a
typical or common use of that class of product, regardless of the status
of the particular user or of the way in which the particular user
actually uses, or expects or is expected to use, the product.  A product
is a consumer product regardless of whether the product has substantial
commercial, industrial or non-consumer uses, unless such uses represent
the only significant mode of use of the product.

``Installation Information'' for a User Product means any methods,
procedures, authorization keys, or other information required to install
and execute modified versions of a covered work in that User Product from
a modified version of its Corresponding Source.  The information must
suffice to ensure that the continued functioning of the modified object
code is in no case prevented or interfered with solely because
modification has been made.

If you convey an object code work under this section in, or with, or
specifically for use in, a User Product, and the conveying occurs as
part of a transaction in which the right of possession and use of the
User Product is transferred to the recipient in perpetuity or for a
fixed term (regardless of how the transaction is characterized), the
Corresponding Source conveyed under this section must be accompanied
by the Installation Information.  But this requirement does not apply
if neither you nor any third party retains the ability to install
modified object code on the User Product (for example, the work has
been installed in ROM).

The requirement to provide Installation Information does not include a
requirement to continue to provide support service, warranty, or updates
for a work that has been modified or installed by the recipient, or for
the User Product in which it has been modified or installed.  Access to a
network may be denied when the modification itself materially and
adversely affects the operation of the network or violates the rules and
protocols for communication across the network.

Corresponding Source conveyed, and Installation Information provided,
in accord with this section must be in a format that is publicly
documented (and with an implementation available to the public in
source code form), and must require no special password or key for
unpacking, reading or copying.

\item Additional Terms.

``Additional permissions'' are terms that supplement the terms of this
License by making exceptions from one or more of its conditions.
Additional permissions that are applicable to the entire Program shall
be treated as though they were included in this License, to the extent
that they are valid under applicable law.  If additional permissions
apply only to part of the Program, that part may be used separately
under those permissions, but the entire Program remains governed by
this License without regard to the additional permissions.

When you convey a copy of a covered work, you may at your option
remove any additional permissions from that copy, or from any part of
it.  (Additional permissions may be written to require their own
removal in certain cases when you modify the work.)  You may place
additional permissions on material, added by you to a covered work,
for which you have or can give appropriate copyright permission.

Notwithstanding any other provision of this License, for material you
add to a covered work, you may (if authorized by the copyright holders of
that material) supplement the terms of this License with terms:
  \begin{enumerate}
  \item Disclaiming warranty or limiting liability differently from the
  terms of sections 15 and 16 of this License; or

  \item Requiring preservation of specified reasonable legal notices or
  author attributions in that material or in the Appropriate Legal
  Notices displayed by works containing it; or

  \item Prohibiting misrepresentation of the origin of that material, or
  requiring that modified versions of such material be marked in
  reasonable ways as different from the original version; or

  \item Limiting the use for publicity purposes of names of licensors or
  authors of the material; or

  \item Declining to grant rights under trademark law for use of some
  trade names, trademarks, or service marks; or

  \item Requiring indemnification of licensors and authors of that
  material by anyone who conveys the material (or modified versions of
  it) with contractual assumptions of liability to the recipient, for
  any liability that these contractual assumptions directly impose on
  those licensors and authors.
  \end{enumerate}

All other non-permissive additional terms are considered ``further
restrictions'' within the meaning of section 10.  If the Program as you
received it, or any part of it, contains a notice stating that it is
governed by this License along with a term that is a further
restriction, you may remove that term.  If a license document contains
a further restriction but permits relicensing or conveying under this
License, you may add to a covered work material governed by the terms
of that license document, provided that the further restriction does
not survive such relicensing or conveying.

If you add terms to a covered work in accord with this section, you
must place, in the relevant source files, a statement of the
additional terms that apply to those files, or a notice indicating
where to find the applicable terms.

Additional terms, permissive or non-permissive, may be stated in the
form of a separately written license, or stated as exceptions;
the above requirements apply either way.

\item Termination.

You may not propagate or modify a covered work except as expressly
provided under this License.  Any attempt otherwise to propagate or
modify it is void, and will automatically terminate your rights under
this License (including any patent licenses granted under the third
paragraph of section 11).

However, if you cease all violation of this License, then your
license from a particular copyright holder is reinstated (a)
provisionally, unless and until the copyright holder explicitly and
finally terminates your license, and (b) permanently, if the copyright
holder fails to notify you of the violation by some reasonable means
prior to 60 days after the cessation.

Moreover, your license from a particular copyright holder is
reinstated permanently if the copyright holder notifies you of the
violation by some reasonable means, this is the first time you have
received notice of violation of this License (for any work) from that
copyright holder, and you cure the violation prior to 30 days after
your receipt of the notice.

Termination of your rights under this section does not terminate the
licenses of parties who have received copies or rights from you under
this License.  If your rights have been terminated and not permanently
reinstated, you do not qualify to receive new licenses for the same
material under section 10.

\item Acceptance Not Required for Having Copies.

You are not required to accept this License in order to receive or
run a copy of the Program.  Ancillary propagation of a covered work
occurring solely as a consequence of using peer-to-peer transmission
to receive a copy likewise does not require acceptance.  However,
nothing other than this License grants you permission to propagate or
modify any covered work.  These actions infringe copyright if you do
not accept this License.  Therefore, by modifying or propagating a
covered work, you indicate your acceptance of this License to do so.

\item Automatic Licensing of Downstream Recipients.

Each time you convey a covered work, the recipient automatically
receives a license from the original licensors, to run, modify and
propagate that work, subject to this License.  You are not responsible
for enforcing compliance by third parties with this License.

An ``entity transaction'' is a transaction transferring control of an
organization, or substantially all assets of one, or subdividing an
organization, or merging organizations.  If propagation of a covered
work results from an entity transaction, each party to that
transaction who receives a copy of the work also receives whatever
licenses to the work the party's predecessor in interest had or could
give under the previous paragraph, plus a right to possession of the
Corresponding Source of the work from the predecessor in interest, if
the predecessor has it or can get it with reasonable efforts.

You may not impose any further restrictions on the exercise of the
rights granted or affirmed under this License.  For example, you may
not impose a license fee, royalty, or other charge for exercise of
rights granted under this License, and you may not initiate litigation
(including a cross-claim or counterclaim in a lawsuit) alleging that
any patent claim is infringed by making, using, selling, offering for
sale, or importing the Program or any portion of it.

\item Patents.

A ``contributor'' is a copyright holder who authorizes use under this
License of the Program or a work on which the Program is based.  The
work thus licensed is called the contributor's ``contributor version''.

A contributor's ``essential patent claims'' are all patent claims
owned or controlled by the contributor, whether already acquired or
hereafter acquired, that would be infringed by some manner, permitted
by this License, of making, using, or selling its contributor version,
but do not include claims that would be infringed only as a
consequence of further modification of the contributor version.  For
purposes of this definition, ``control'' includes the right to grant
patent sublicenses in a manner consistent with the requirements of
this License.

Each contributor grants you a non-exclusive, worldwide, royalty-free
patent license under the contributor's essential patent claims, to
make, use, sell, offer for sale, import and otherwise run, modify and
propagate the contents of its contributor version.

In the following three paragraphs, a ``patent license'' is any express
agreement or commitment, however denominated, not to enforce a patent
(such as an express permission to practice a patent or covenant not to
sue for patent infringement).  To ``grant'' such a patent license to a
party means to make such an agreement or commitment not to enforce a
patent against the party.

If you convey a covered work, knowingly relying on a patent license,
and the Corresponding Source of the work is not available for anyone
to copy, free of charge and under the terms of this License, through a
publicly available network server or other readily accessible means,
then you must either (1) cause the Corresponding Source to be so
available, or (2) arrange to deprive yourself of the benefit of the
patent license for this particular work, or (3) arrange, in a manner
consistent with the requirements of this License, to extend the patent
license to downstream recipients.  ``Knowingly relying'' means you have
actual knowledge that, but for the patent license, your conveying the
covered work in a country, or your recipient's use of the covered work
in a country, would infringe one or more identifiable patents in that
country that you have reason to believe are valid.

If, pursuant to or in connection with a single transaction or
arrangement, you convey, or propagate by procuring conveyance of, a
covered work, and grant a patent license to some of the parties
receiving the covered work authorizing them to use, propagate, modify
or convey a specific copy of the covered work, then the patent license
you grant is automatically extended to all recipients of the covered
work and works based on it.

A patent license is ``discriminatory'' if it does not include within
the scope of its coverage, prohibits the exercise of, or is
conditioned on the non-exercise of one or more of the rights that are
specifically granted under this License.  You may not convey a covered
work if you are a party to an arrangement with a third party that is
in the business of distributing software, under which you make payment
to the third party based on the extent of your activity of conveying
the work, and under which the third party grants, to any of the
parties who would receive the covered work from you, a discriminatory
patent license (a) in connection with copies of the covered work
conveyed by you (or copies made from those copies), or (b) primarily
for and in connection with specific products or compilations that
contain the covered work, unless you entered into that arrangement,
or that patent license was granted, prior to 28 March 2007.

Nothing in this License shall be construed as excluding or limiting
any implied license or other defenses to infringement that may
otherwise be available to you under applicable patent law.

\item No Surrender of Others' Freedom.

If conditions are imposed on you (whether by court order, agreement or
otherwise) that contradict the conditions of this License, they do not
excuse you from the conditions of this License.  If you cannot convey a
covered work so as to satisfy simultaneously your obligations under this
License and any other pertinent obligations, then as a consequence you may
not convey it at all.  For example, if you agree to terms that obligate you
to collect a royalty for further conveying from those to whom you convey
the Program, the only way you could satisfy both those terms and this
License would be to refrain entirely from conveying the Program.

\item Use with the GNU Affero General Public License.

Notwithstanding any other provision of this License, you have
permission to link or combine any covered work with a work licensed
under version 3 of the GNU Affero General Public License into a single
combined work, and to convey the resulting work.  The terms of this
License will continue to apply to the part which is the covered work,
but the special requirements of the GNU Affero General Public License,
section 13, concerning interaction through a network will apply to the
combination as such.

\item Revised Versions of this License.

The Free Software Foundation may publish revised and/or new versions of
the GNU General Public License from time to time.  Such new versions will
be similar in spirit to the present version, but may differ in detail to
address new problems or concerns.

Each version is given a distinguishing version number.  If the
Program specifies that a certain numbered version of the GNU General
Public License ``or any later version'' applies to it, you have the
option of following the terms and conditions either of that numbered
version or of any later version published by the Free Software
Foundation.  If the Program does not specify a version number of the
GNU General Public License, you may choose any version ever published
by the Free Software Foundation.

If the Program specifies that a proxy can decide which future
versions of the GNU General Public License can be used, that proxy's
public statement of acceptance of a version permanently authorizes you
to choose that version for the Program.

Later license versions may give you additional or different
permissions.  However, no additional obligations are imposed on any
author or copyright holder as a result of your choosing to follow a
later version.

\item Disclaimer of Warranty.

\begin{sloppypar}
 THERE IS NO WARRANTY FOR THE PROGRAM, TO THE EXTENT PERMITTED BY
 APPLICABLE LAW.  EXCEPT WHEN OTHERWISE STATED IN WRITING THE
 COPYRIGHT HOLDERS AND/OR OTHER PARTIES PROVIDE THE PROGRAM ``AS IS''
 WITHOUT WARRANTY OF ANY KIND, EITHER EXPRESSED OR IMPLIED,
 INCLUDING, BUT NOT LIMITED TO, THE IMPLIED WARRANTIES OF
 MERCHANTABILITY AND FITNESS FOR A PARTICULAR PURPOSE.  THE ENTIRE
 RISK AS TO THE QUALITY AND PERFORMANCE OF THE PROGRAM IS WITH YOU.
 SHOULD THE PROGRAM PROVE DEFECTIVE, YOU ASSUME THE COST OF ALL
 NECESSARY SERVICING, REPAIR OR CORRECTION.
\end{sloppypar}

\item Limitation of Liability.

 IN NO EVENT UNLESS REQUIRED BY APPLICABLE LAW OR AGREED TO IN
 WRITING WILL ANY COPYRIGHT HOLDER, OR ANY OTHER PARTY WHO MODIFIES
 AND/OR CONVEYS THE PROGRAM AS PERMITTED ABOVE, BE LIABLE TO YOU FOR
 DAMAGES, INCLUDING ANY GENERAL, SPECIAL, INCIDENTAL OR CONSEQUENTIAL
 DAMAGES ARISING OUT OF THE USE OR INABILITY TO USE THE PROGRAM
 (INCLUDING BUT NOT LIMITED TO LOSS OF DATA OR DATA BEING RENDERED
 INACCURATE OR LOSSES SUSTAINED BY YOU OR THIRD PARTIES OR A FAILURE
 OF THE PROGRAM TO OPERATE WITH ANY OTHER PROGRAMS), EVEN IF SUCH
 HOLDER OR OTHER PARTY HAS BEEN ADVISED OF THE POSSIBILITY OF SUCH
 DAMAGES.

\item Interpretation of Sections 15 and 16.

If the disclaimer of warranty and limitation of liability provided
above cannot be given local legal effect according to their terms,
reviewing courts shall apply local law that most closely approximates
an absolute waiver of all civil liability in connection with the
Program, unless a warranty or assumption of liability accompanies a
copy of the Program in return for a fee.

\begin{center}
{\Large\sc End of Terms and Conditions}

\bigskip
How to Apply These Terms to Your New Programs
\end{center}

If you develop a new program, and you want it to be of the greatest
possible use to the public, the best way to achieve this is to make it
free software which everyone can redistribute and change under these terms.

To do so, attach the following notices to the program.  It is safest
to attach them to the start of each source file to most effectively
state the exclusion of warranty; and each file should have at least
the ``copyright'' line and a pointer to where the full notice is found.

{\footnotesize
\begin{verbatim}
<one line to give the program's name and a brief idea of what it does.>

Copyright (C) <textyear>  <name of author>

This program is free software: you can redistribute it and/or modify
it under the terms of the GNU General Public License as published by
the Free Software Foundation, either version 3 of the License, or
(at your option) any later version.

This program is distributed in the hope that it will be useful,
but WITHOUT ANY WARRANTY; without even the implied warranty of
MERCHANTABILITY or FITNESS FOR A PARTICULAR PURPOSE.  See the
GNU General Public License for more details.

You should have received a copy of the GNU General Public License
along with this program.  If not, see <https://www.gnu.org/licenses/>.
\end{verbatim}
}

Also add information on how to contact you by electronic and paper mail.

If the program does terminal interaction, make it output a short
notice like this when it starts in an interactive mode:

{\footnotesize
\begin{verbatim}
<program>  Copyright (C) <year>  <name of author>

This program comes with ABSOLUTELY NO WARRANTY; for details type `show w'.
This is free software, and you are welcome to redistribute it
under certain conditions; type `show c' for details.
\end{verbatim}
}

The hypothetical commands {\tt show w} and {\tt show c} should show
the appropriate
parts of the General Public License.  Of course, your program's commands
might be different; for a GUI interface, you would use an ``about box''.

You should also get your employer (if you work as a programmer) or
school, if any, to sign a ``copyright disclaimer'' for the program, if
necessary.  For more information on this, and how to apply and follow
the GNU GPL, see \texttt{https://www.gnu.org/licenses/}.

The GNU General Public License does not permit incorporating your
program into proprietary programs.  If your program is a subroutine
library, you may consider it more useful to permit linking proprietary
applications with the library.  If this is what you want to do, use
the GNU Lesser General Public License instead of this License.  But
first, please read \\ \texttt{https://www.gnu.org/licenses/why-not-lgpl.html}.

\end{enumerate}
%
% Index
%
\newpage

\

\newpage
\addcontentsline{toc}{chapter}{Index}
\printindex
% add empty pages at the end
\newpage \mbox{} \thispagestyle{empty}
\newpage \mbox{} \thispagestyle{empty}
\end{document}
